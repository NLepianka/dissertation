%%%%%%%%%%%%%%%%%%%%%%%%%%%%%%%%%%%%%%%%%%%%%%%%%%%
%
%  New template code for TAMU Theses and Dissertations starting Fall 2016.  
%
%
%  Author: Sean Zachary Roberson
%  Version 3.17.09
%  Last Updated: 9/21/2017
%
%%%%%%%%%%%%%%%%%%%%%%%%%%%%%%%%%%%%%%%%%%%%%%%%%%%
%%%%%%%%%%%%%%%%%%%%%%%%%%%%%%%%%%%%%%%%%%%%%%%%%%%%%%%%%%%%%%%%%%%%%%
%%                           SECTION III
%%%%%%%%%%%%%%%%%%%%%%%%%%%%%%%%%%%%%%%%%%%%%%%%%%%%%%%%%%%%%%%%%%%%%

\chapter{WRIGHT AND THE AMERICAN ANTIQUARIAN SOCIETY}

\section{Preface}

While the previous chapter addressed the theories, ideas, and interpretive decisions that undergird the organization of enumerative bibliographies, it remains important to investigate what can inform the decisions that affect the construction of a bibliography now that I have argued for the ultimately subjective qualities this genre possesses. The bibliographer attempting to compile and describe a list of books must, of course, endure the process of researching, finding, comparing, and collating the books they will inevitably list. But this process is iterative, littered with changes of mind, inconsistencies to be corrected, and ever more information to be found, documented, and described. This process is aided in the case of the bibliographer by the institutions that can support the exploration and documentation of bibliographic data--primarily academic research libraries and archives that have invested in amassing collections. Institutions and the people that work within them can affect the bibliographer in the course of the creation of an enumerated list. The bibliographer's work is complemented and complicated by the institutions, what they hold, their bibliographic data, and the expertise of their staff. This relationship, however, is not unidirectional, passing from the institution to the bibliographer. Institutions themselves can be affected by the work of the bibliographer, when it is revealed where their own collections stand amongst the compilation of titles that exist, particularly when such institutions have invested time into the production of a bibliography. 

This chapter will look more closely at Lyle Wright and the composition of \textit{American Fiction}. More specifically, I will address the composition of the first volume of the bibliography, \textit{American Fiction, 1774-1850} (1939), and the relationship Wright formed with the American Antiquarian Society (AAS), one of the most significant libraries of early American materials in the U.S. The AAS was an early supporter of Wright's research; they provided Wright with access to their vast catalog but also vouched for Wright's project as a valuable contribution to bibliography and literary history. Wright was both affected by and affected the American Antiquarian Society, and those effects persisted for decades after the publication of the final volume, even continuing after Wright's death in 1979. Even now, Wright's legacy persists in terms not just of the holdings of the AAS, but the digital presence of the AAS in terms of its online catalog MARC records and the North American Imprints Program (NAIP) the AAS has maintained to the present day.\footnote{MARC, or Machine Readable Cataloging Record, is a standard for recording and displaying bibliographic information about a library holding that is easily digestible and configurable for digital databases and online catalogs of library holdings.} Through the archival materials found at the AAS, I will argue that the relationship between Wright and the AAS fundamentally informed the data of \textit{American Fiction} and its representation. The AAS, in turn, had its collections informed by Wright after the publication of \textit{American Fiction}, and went so far as to use Wright's work as a guideline for the development of the institution's fiction holdings. The result of which has had an effect on the ways scholars can and do interact with the titles described by the Wright. This chapter will trace the narrative of Wright's connection to the AAS, its people, and its collections. I will begin with discussing Wright's earliest communication with the AAS, primarily its head librarian, Robert. G. Vail, when \textit{American Fiction} was simply an idea, and go to the publication of the first volume. After that, I will look at how the AAS adopted and worked with Wright's bibliography, and incorporated it into their mission and daily life. 

To begin, let us first examine the AAS and its history. The American Antiquarian Society represents one of most prominent institutions that supports American cultural research and preservation. The Society itself was founded by Isaiah Thomas in 1812, and its library is located in Worcester, MA. Thomas' original petition for the establishment of the AAS as an institution denotes a future idea of the both the purpose of the society and its values as an institution. As Philip F. Gura reports, those in favor of the establishment of an American Antiquarian Society were "influenced by a desire to contribute to the advancement of the Arts and Sciences, and to aid, by their individual, and united efforts in collecting, and preserving such materials, as may be useful in making their progress" and "wish[ed] also to assist the researches of the future historians of our country."\footnote{\autocite[19]{gura_american_2012}. Gura's source is a copy of the petition found in the AAS Archives. See \autocite[11-14]{society_account_1813} for a transcription of the petition.} The foresight of the statement recognizes the capacity of the institution to inform future scholars; historical associations, libraries, and collections have a perpetual influence on scholarship, both as the institution itself grows and as more and more scholars come to make use of the materials the institution holds. Inherent to this desire of the founders of promoting American scholarship is the fact that the labor of librarians and collectors is necessary to ensuring that future scholarship can be done.\footnote{The current mission statement presented by the AAS makes an appeal to a broad audience. It not only limits its mission to professional scholarship, but promises to serve anyone who may have an interest, professional or otherwise, in American history:
\begin{displayquote}
Our mission is to collect, preserve, and make available for study the printed record of what is now the United States of America from first European settlement through the year 1876. As a learned society, we offer a wide variety of programs for diverse audiences including: professional scholars, pre-collegiate, undergraduate and graduate level students and educators, professional artists and writers, genealogists, and the general public.\autocite{noauthor_mission_2017}
\end{displayquote}
The appeal to an audience beyond the academy not only reinforces a more humanitarian view of education in American history, but also reveals how large a possible impact an institution can have via its collections.}

In 1937, as Wright was in the initial stages of creating \textit{American Fiction}, The AAS celebrated its 125th anniversary since its founding. The annual report for 1937 includes a statement by the Director of the AAS at the time, Clarence S. Brigham, who declares the chief reason for the society's founding was the "far-seeing realization" that scholars of American history and culture would emerge in the New Republic. Brigham points to an undisclosed 1814 "Report" that says: "The philosopher and the historian, or any to whom the Library of this Society
may be useful, will not greatly regret the distance
which separates them from the objects of their pursuit,
if they can but eventually obtain in one place,
what, otherwise, they would have to seek in many."\autocite[6]{clarence_s._brigham_report_1937} 
Brigham's citation points to a key similarity between bibliographies and libraries. Both aggregate materials from many places in order to make them available in a single, more easily accessible location. The institution creates a physical space where a catalog of materials may be found while the bibliography compiles its information into a printed volume (that is then stored in the physical space of the library). In this light, the fact that the AAS would invest in the initial stages of \textit{American Fiction}'s is less surprising. Wright, as a bibliographer, and the AAS shared a similar goal in this sense of wanting to make more accessible information about American culture. The work Wright was doing aligned with the desires of the AAS in realizing both the practical use to scholarship the aggregation of materials offers to scholars. It is important to recognize here the connections between the AAS and Wright, as it helps us to understand how \textit{American Fiction} came into existence. The implied connection between the goals of Wright and the AAS helps us to understand why the early days of Wright's work were supported so enthusiastically by the AAS. And this support was crucial, because it ultimately shaped the project as a whole, and quite possibly, if not for the people of the AAS, \textit{American Fiction} may never have gotten the support it did.

\section{Wright and Vail}

One of the most significant figures in the early days of Wright's work was Robert G. Vail, the head librarian of the AAS from 1930 to 1939. Vail was a veteran in the library world and in bibliography, while Wright was relatively young, serving as an assistant bibliographer at the Huntington Library during this time. Vail's position at the AAS was relatively short, but a key period for the Society. Coming from the New York Public Library, Vail was hired to be the head librarian of the AAS in 1930. According to Gura, the allure of the AAS to Vail at the time had been the possibility of expanding collections, something that was not a priority at the time for the NYPL. The Society had a reputation for "significant acquisitions" but that would not be the reality of Vail's time at the AAS. As the Depression was affecting the nation, Vail found himself limited in his ability to acquire new materials, relying more on donations in this period rather than purchasing power.\footnote{Gura mentions that, while the Society's financial security was more precarious during the decade, the Depression did oddly benefit the Society in some ways however, as booksellers lowered their prices and seemingly desperate donations by private citizens helped to support continued collections development.\autocite[225]{gura_american_2012}} 

But these circumstances served to benefit Vail in terms of allowing him the ability to complete his own projects, and even moreso, Wright and others benefitted from Vail's ability to commit time and energy to their requests and needs.\autocite[220-3,225,231]{gura_american_2012} Thus, the correspondence that exists between Vail and Wright, considering Vail's penchant for writing longer correspondences, is a byproduct of the increased ability for Vail to commit to the more personal service aspects of the library professions over those that acquisitions would demand. Adding to Vail's prestige, not just as a head librarian, was that fact that he had taken part in several bibliographical projects, as the lead in finishing a twenty-seven volume bibliography of North American print, Sabin's \textit{Bibliotheca Americana}. Joseph Sabin, the original compiler of the \textit{Americana} had died in 1881, before completing the twentieth volume of the \textit{Americana}. The task of finishing Sabin's work had originally been assumed by Wilberforce Eames, another well-known bibliographer of American material, but Vail eventually joined the project as a collaborator and became the lead for volumes twenty-two through twenty-seven.\footnote{Sabin's \textit{Bibliotheca Americana} is still considered a major resource for American research, as it combines historical documents, philosophical treatises, and literary works into a massive collection of over 100,000 titles covering the years 1500 to 1926. The bibliography is also notable because it is multilingual and international in scope. Currently, Lindsay Van Tine, a fellow at the Library Company of Philadelphia, is working on creating a digital repository of the \textit{Americana} titles.} In addition to completing Sabin's work, Vail had previous committed time to enumerating every edition of Susanna Rowson's 1791 novel \textit{Charlotte Temple} (a resource Wright would use), and was beginning to put together a bibliography of "Indian captivity narratives" as he called them.\autocite{vail_susanna_1933,vail_voice_1949}

Given both his position at the AAS and involvement with the \textit{Bibliotheca Americana}, it was natural that Vail was one of the first people Wright reached out to when he began to think about creating \textit{American Fiction}. Over the years of correspondence, the two became not just professional acquaintances but friends as well. Wright respected Vail's input and would eventually make decisions about the structure and contents of his bibliography based on Vail's influence, the title included. Vail was a consistent supporter and encouraging force for Wright as he was composing his first volume. Wright initiated the conversation with Vail in a letter dated June 24, 1933, wherein he discussed the idea of making a bibliography. His first message is short, but reveals the fact that \textit{American Fiction} was not always Wright's ambition. He states  that in fact his original idea for a bibliography would have been the eighteenth-century British novel, which could have included American editions, however, as Wright states, the idea was "'nipped in the bud' by a Mr. Block of England."\footnote{\autocite{lyle_h._wright_letter_1933}. The work Wright alludes to would seem to be \autocite{block_english_1939}.} His first choice of subject seemingly unviable, Wright pitches the idea of a bibliography of American literature, but asks Vail whether or not anyone else is compiling a similar list, and, more importantly, how far into the nineteenth-century a bibliography of American novels should extend. At this point, Wright and Vail seemingly have no connection to each other , and it was likely just Vail's prestigious position that made him a good candidate for Wright's questions rather than any bond or previous relationship at this point.

Vail's June 29 response, however, is effusive and self-admittedly "rambling", as his reply is a dense two pages in which he states to Wright that "it is high time that someone tackled the fiction bibliography."\autocite{robert_g._vail_letter_1933} Vail points to Wright's predecessor in American Bibliography, Oscar Wegelin and his \textit{Early American Fiction, 1774-1830}, commenting that the work is "so inadequate."\autocite{robert_g._vail_letter_1933} Vail theorizes that many graduate students must be "floundering" in an attempt to compile an American fiction bibliography, but claims that Wright is well-suited to the job since he has both bibliographic training and the Huntington Library and its resources to lend support. The final notes of the letter, however, offer practical advice as well, commanding Wright that should he take up the project, to publicize it and notify the major research libraries so as to stake his claim. In response to Wright's request for an adequate range of years for the bibliography, Vail responds, 
\begin{displayquote}
I hope you will include everything in fiction form separately published, and would almost think that it should go down to 1840 or 1850, certainly to the former date...there is a mass of early attempts at novels between 1830 and 1850 which are valuable as literary history and which should not be ignored. If you feel that the task is too large, and it certainly would be a huge one in any case, you might stop at 1830.\autocite{robert_g._vail_letter_1933}
\end{displayquote}
Vail reveals his own bibliographical stance here, as he discusses what he believes Wright's bibliography could do. What Vail calls "early attempts at novels" he considers and important part of literary history regardless of their literary merit. It is the job of the bibliographer to account for and keep in scholarly consciousness these works because they are a part of the developing literary culture of early America. Vail's role as a librarian would dictate a concern for the quantity of works that exist and possess the ability to be described, listed, cataloged, and subsequently used by scholars. This is in opposition to the works of literary scholars such as Quinn and O'Neill who reduce the field to canonical names based on a metric of quality, self-admittedly making no attempt at endeavoring for a comprehensive bibliography of the subject.\citereset{\autocite{robert_g._vail_letter_1936}} Vail complements his ideal bibliography with the more practical suggestion of only covering up to 1830, which would do the main task of overwriting Wegelin's "inadequate" work that covered 1774 to 1830, replacing it with what Vail hopes is Wright's more comprehensive record of American fiction. Wright goes forward with the initial suggestion of covering up to 1850, and eventually pursuing more than even Vail idealized by extending \textit{American Fiction} to three volumes and to the year 1900.

When Wright resumed the discussion of the bibliography with Vail in his 27 November 1936 letter, he provides Vail with some of the preliminary ideas for the format of the bibliography amidst some questions about the practicalities of traveling to Worcester, MA and navigating the materials housed at the AAS. Wright tells Vail that he would indeed take the bibliography up to 1850, in concordance with Vail's ideal and despite the warnings and challenge. Wright's model in this letter described titles simply and unadorned by trickier or more advanced bibliographic detail: "Author, title, imprint, format, collation, annotation, notes (only those absolutely necessary), census." This was in stark contrast to the norms of the time for bibliographers, especially in the midst of the New Bibliography as discussed in the last chapter. While the description of books could be complicated and detailed, Wright's own model would hold back on advanced bibliographical details in favor of a more accessible form of description that gave those outside of the field of bibliography the information they needed more quickly. Wright did not anticipate including items such as the kind of type, differences in the title lines, the collation formulas, or description of the type ornaments.\footnote{These elements are common in advanced bibliographical descriptions of texts and attempt to give as replete an account of the physical properties of a text as possible. While detailed and give a precise account of the text helpful to understanding the circumstances of a text's printing, Wright did not foresee \textit{American Fiction} as necessarily performing the same role as an intricate descriptive bibliography.} These sorts of details were esoteric and readers outside of bibliography would not necessarily have the training needed to understand the information. Wright's more narrow descriptions would include only the relevant bibliographical information that he had decided upon after discussion with "several librarians and research men."\autocite{lyle_h._wright_letter_1936-1} To this Vail adds his own advice at Wright's prompting, including a few interesting rationales for Wright's decisions: 

\begin{displayquote}
Regarding the form of your bibliography I think it is admirable as you outline it. I think that it is unnecessary in these days of the use of the photostat to go into all the horrible details of minute bibliographical description. It is more important to give a brief author, title, imprint, size, and collation with locations of copies than to describe each item as though it were a piece of incunabula. However, when you find an absolutely unique item, it would naturally call for a more detailed description...I do not think that you can lay down a hard and fast set of rules for any bibliography. I think in general your format is splendid, but I would expect you to vary it for an occassional difficult entry.\footnote{\autocite{robert_g._vail_letter_1936-1} Vail also includes in this section his distaste for the Library of Congress Union Catalog codes to reference libraries. The Union Catalog Code provided a standardized list of ways to refer to libraries when, for example, listing them in a census. The codes are sometimes not entirely intuitive. For example, the code for the American Antiquarian Society is MWA, referencing geographic location, Massachusetts-Worcester-American Antiquarian with its code, rather than AAS, which Wright and other bibliographers employ.} 
\end{displayquote}

Vail agrees with Wright in terms of presenting a more simplified description than was usual at the time for bibliographic entries. Interestingly, the foundation of Vail's agreement is in the technological advancement of the photostat, which would allow, as Vail infers, the evaluation of the various items that Wright would be excluding from his descriptions (i.e. the kinds of types and ornaments) with more precision than they perhaps would for the incunabula that bibliographers whose concerns are focused on texts that precede American publishing practices. Vail's suggestion here is positive in its view of technology, proposing that a photocopy of a text is an adequate stand in for the text itself. This view would come to be criticized by G. Thomas Tanselle, who, noting the unstable nature of the text, argues that a photocopy of a text constitutes an entirely new text and so unsuitable as evidence in bibliographical description.\footnote{See \autocite{tanselle_reproductions_1989}} But Vail's point of view is directed towards the idea of an enumerative bibliography, whose aims are not that of an analytical or descriptive bibliography. Vail understands that the nature of Wright's list would be to aggregate and demonstrate the breadth of American fiction, not in attempting to describe the manufacture and construction of books, but of their existence in the first place, their accessibility (via the census), and to place them amongst each other without the weight of literary history's value judgments (such as Quinn prescribes in his work). On the other hand, Vail's approval of simplified description also suggests an acknowledgment of the labor involved in compiling a bibliography, which is understandable given his firsthand knowledge of this labor he himself had put into completing Sabin's \textit{Bibliotheca Americana} and his other, smaller bibliographies. The exclusion of more advanced bibliographical details would involve more time and labor that would get in the way of the limited time that Wright would have to visit multiple libraries while still employed by the Huntington Library. What Vail sees in Wright's method, then, is the simplicity of his descriptions as they fit into the acknowledged standard of publishing and the bibliographical information that makes accessing and finding texts feasible, with the only allowances for variation from that standard being books that are "absolutely unique" and "difficult" to describe.

Amusingly, however, these early communications also reveal that there was some misunderstanding between the two. At the time of Wright's tenure at he Huntington Library, there were in fact two L. Wrights employed by the library. The other Wright was Louis Booker Wright, a member of the Huntington Library's research staff, and who would eventually become the director of the Folger Library in 1948. Lyle Wright, however, was at the time of his writing to Vail one of the assistant bibliographers. After Vail had mentioned in a letter that Wright would meet Arthur Hobson Quinn at the Modern Language Association's 1937 annual meeting, Wright realized that Vail did not know who he was talking to.\autocite{lyle_h._wright_letter_1936-2} The mention of the MLA meeting revealed to Lyle Wright that Vail believed he was corresponding with Louis B. Wright. Thus, Vail's apparent lauding of Wright's abilities is directed moreso towards the man whom he thought he was speaking to, more of a peer, than the (at the time) less impressive Lyle Wright. Yet, oddly, when Wright corrects the misunderstanding and reveals that he a different Wright from Louis B. Wright, Vail does not retract his claims, and continues to support the idea of Lyle Wright's pursuit of American fiction. 

Wright was by no means a novice to compiling a bibliography. Previous to \textit{American Fiction}, Wright had compiled for the Huntington Library a bibliography of "sporting books", that is, titles that contained references to various sports.\autocite{wright_sporting_1937} This list was not limited to either American titles or fictional ones, though it did contain fiction. The work was the only published Wright bibliography before \textit{American Fiction} and consisted of a more narrow scope in terms of its goals though it did still contain 1344 entries. One review particularly praises the work and Wright's bibliographical ability; written by Virgil Heltzel, the review says, "The care, the accuracy, the excellent bibliographical method employed, the printing, and the general appearance of this work make it worthy of the institution from which it has come."\autocite{heltzel_review_1938} While not publishing as widely as the eminent bibliographers of his day (i.e. Bowers or Pollard), Wright's bibliographical skill was evident to Vail, given some of the early communication between them that did not concern American fiction. As assistant bibliographer, Wright provided Vail with several bibliographical descriptions, including collations, that Vail had requested as he was completing his work with Sabin. A letter from Wright to Vail dated 14 Oct. 1936 includes collations of the Huntington's two copies of \textit{Zionitischer Weyrauchs Hagel, oder Myrrhen Berg}.\footnote{A 1739 German hymn book. Printed in Germantown, PA. Sabin no. 106364.} Included with the descriptions of the text, Wright notes that both copies should be classified as variants because the signature B was, in one copy, bound incorrectly, causing the pages to be out of order. The other copy had entirely reset the type in that signature to correct the mistake.\autocite{lyle_h._wright_letter_1936-3} Wright's collations were included by Vail in the Sabin entry alongside Wright's conjectures as to the discrepancies between the two different Huntington copies of the text, pointing to the institution directly.\autocite{sabin_bibliotheca_1936} The \textit{Zionitischer} collations are the most obvious examples of Wright's involvement and bibliographic contributions to Vail's completion of the \textit{Bibliotheca Americana}, though in an 5 Oct. 1936, Wright provides six other collations for Vail's Sabin work.\footnote{The six titles are 1. Los Imparicales [pseud.], \textit{Examen del merito que puedan tener los fundamentos con que se ha declarado nulo el prestamo de ciento treinta mil libras esterlinas}, 1839 2. Young, Robert, \textit{The Dying Criminal}, n.d. 3. Young, Robert, \textit{The Last Words and Dying Speech}, n.d. 4. \textit{Yndemnisacion Plena de Don Isidoro Palacio, en la Revolucion de la Guarnicion de Apan}, 1820 5. \textit{Constution and Bye-Laws, of the York County Conference of Churches, as revised and presented to the Conference at its semi-annual meeting in Acton, Oct. 1, 1839}, 1840 6. \textit{American Husbandry : Containing an Account of the Soil, Climate, Production and Agriculture, Of The British Colonies In North-America and the West-Indies ; With Observations on the Advantages and Disadvantages of settling in them, compared with Great Britain and Ireland}, 1775} Regardless of the case of mistaken identity, Wright had proved his expertise in the field to Vail before, and this is perhaps why Vail never retracted his support for \textit{American Fiction}.

In a February 2, 1937 letter from Wright to Vail, Wright reveals where he is in the thought of titling the project that would become \textit{American Fiction}: 

\begin{displayquote}
I have been reading Professor Quinn's book. I do not agree with him on many points, but then, I am not a professor. The book has forced me to the conclusion that to avoid trouble with the purists I will have to call my work a bibliography of American fiction. This will handle nicely \underline{Amelia; or, the Faithless Britain, an original novel}, 1787, even though "it is hardly that." \autocite{lyle_h._wright_letter_1937}
\end{displayquote}

The book in question is Arthur Hobson Quinn's \textit{American Fiction: An Historical and Critical Survey} (1936). This brief paragraph is in a reply to Vail about Wright's upcoming visit to the AAS in the midst of composing what would become the first volume of \textit{American Fiction}. Wright's reading of Quinn was the responsibility of Vail himself, who had written "Prof. Quinn?" in the margin of Wright's initial 27 November 1936 letter about his AAS visit and the work he had begun on the bibliography. In two replies to Wright, dated 4 Dec. and 12 Dec. of 1936, Vail makes mention of Quinn's recent work at the AAS, but more importantly he iterates how much more suited Wright would be to the job of compiling a bibliography of American Literature than Quinn, and his associate, Edward H. O'Neill:\footnote{\autocite{robert_g._vail_letter_1936-1,robert_g._vail_letter_1936} O'Neill is thanked in the credits of Quinn's \textit{American Fiction}. O'Neill was a scholar of biography; in 1935 he published \textit{A History of American Biography, 1800-1935}. See \autocite{oneill_history_1935}. Shortly after the publication of Wright I, O'Neill contributed his own subject bibliography of American biographies: \autocite{oneill_biography_1939}}
\begin{displayquote}
It would be a shame to have two of you working on the same project. I think you would do a very much better job of it than Mr. O'Neill, who has enthusiasm but not library training for the job. He has done a good deal of work here and though he seems to accomplish a great deal, I do not think he is very accurate and any digging he does would have to be verified. This, of course is between ourselves.\footnote{\autocite{robert_g._vail_letter_1936}. While damning of O'Neill's abilities, it seems Vail would be correct in his assertion. In a review of O'Neill's \textit{Biography by Americans, 1658–1936} by Milton Halsey Thomas, he states that a bibliography of biographies by Americans could be valuable, but O'Neill's work is "neither complete nor indexed" and contains "astonishing" omissions. \autocite{thomas_biography_1940}}
\end{displayquote}

The importance of these exchanges, including Wright's mistaken identity, the reading of Quinn's work, and Vail's praise is that it reveals how much of an effect the AAS and its members had on Wright and the composition of \textit{American Fiction}. As stated, Vail's recommendation of Quinn's work to Wright became the deciding factor for the bibliography's final title. The choice of "fiction" over "novel" as Wright was initially intending demonstrates a few particular ideas about how both Wright was conceiving of his bibliography, and how he  took the feedback and information given to him that helped form the sense and purpose of his work. Quinn represented to Wright the sort of "purist" idea of what a novel was, that is, for Quinn, the definition of a novel was a work that seemingly dealt with more than one series of incidents, rather than a single event. When Quinn declares that while a work such as \textit{Amelie} has the word "novel" on its title page, "it is hardly that" because despite the estimated 7000 words Quinn notes, the story covers only a single event. Quinn's own decision to use "fiction" over "novel" in his title is justified by liminal cases such as \textit{Amelie} that can neither be, under Quinn's definitions, a novel, but also not a short story due to its "lack of characterization."\autocite[5-6]{quinn_american_1936}
 
Wright's own notice of the way literary scholars defined genres at the time demonstrates the commitment to making an intervention as a bibliographer into literary historical discussions. It was Vail, however, who noted the possibility for Quinn's book to be effective for Wright. Not simply in terms of semantics, but the recommendation also revealed to Wright the incompleteness of one of the latest bibliographical contributions of American fiction. The recommendation of Quinn as a source of reading for Wright grounded the bibliographer in the current context of American literary scholarship, specifically in how a literary historian would approach the recording of American literary titles. Vail's mention of Wegelin's "inadequate" work as well leaves the same impression. In both cases it becomes important to demonstrate to Wright the limited capacity of previous attempts, and why exactly it is "high time" that someone tackled American literary fiction. 

The decision of Wright to visit the AAS would seem, at first, obvious. In the 1930s the AAS was the foremost archive of early American materials. As Wright was beginning to work on his first volume, Vail was finishing the final volume, 29, of Joseph Sabin's \textit{Bibliotheca Americana}, as mentioned earlier, and would go on to work on finishing Charles Evans' \textit{American Bibliography}.\footnote{Evans' \textit{American Bibliography} is a fourteen-volume bibliography that focused on books, pamphlets, and periodicals published in the United States between 1639 and 1820. It has lost some prestige since its publication because of Evans' many omissions and his tendency to list titles that do not exist.} When Wright was finally able to leave the Huntington to travel to other institutions in April of 1937, he tells Vail he intends on spending most of his time with the Society, a full two months out of his anticipated three month leave. \citereset{\autocite{lyle_h._wright_letter_1937, lyle_h._wright_letter_1936-1}} Vail insists that the AAS itself would fill out most of the bibliography, suggesting that Wright would find "more titles here than in any other collection."\footnote{\autocite{robert_g._vail_letter_1937}
In Vail's 29 June 1933 reply to Wright's initial broaching of the topic of an American fiction bibliography, Vail also boasts of the AAS' collections: "As our library probably contains three-fourths of all the titles published up to 1820 or later, you would doubtless make a good many discoveries here." Vail does temper this claim by revealing the uncertainty of the then uncatalogued collections beyond 1820, which he also in the same letter was suggesting that Wright go beyond, but Vail's certainty as to the importance of the AAS to Wright's project remained consistent over the course of the compiling and publishing of Wright I.} The AAS did indeed supply Wright with many of his titles, as evidenced by a checklist sent to the AAS in the final proofing stages of the first edition of Wright I. On the checklist, the AAS marked itself as owning 737 of the 2239 titles listed in this preliminary document, or roughly 33 percent. Vail had predicted Wright would find approximately three-fourths of his pre-1820 titles at the AAS, and this estimate holds mostly true, with the gaps of the AAS mostly being after 1820.\citereset{\autocite{robert_g._vail_letter_1933}}

Wright's tone towards Vail turns significantly more casual after his travels in the east. He continues to stay in touch with Vail even after the visit and continues to receive information from Vail that would aid him in the compilation of Wright I.\footnote{While outside the scope of this chapter, Vail's help would continue after the pubication of the first volume of \textit{American Fiction} and inevitably help Wright as he began to work on the second volume a decade later.} In his first letter to Vail since arriving back in California, Wright begins the document with the informal "Back home and broke." This letter additionally contains more personal sentiments, as Wright, apparently responding to a situation that developed either in his time at the AAS or in the course of an earlier letter, writes, "I trust Mrs. Vail is well along the road to good health." These moments of congeniality between Wright and Vail temper the rigidity of the bibliographical information the two trade as Wright is finishing his first volume. Previous letters displayed a direct approach in discussing bibliographic matters, but the turn the letters take after Wright's visit to the AAS suggest a relationship that has moved beyond the strict formalities of business. 

As Wright was sending out the initial checklist for various institutions to review, he sought help from Vail in finding titles for the prolific Osgood Bradbury, who had him "going in circles."\autocite{lyle_h._wright_letter_1937-2} The phrasings shared between the two are informal and friendly as much as they are business; Vail responds to the request: "Did you ever read Mark Twain's 'Punch, Brother, Punch'? If so, you will realize what you have done to me in sicking me on Osgood Bradbury."\footnote{\autocite{robert_g._vail_letter_1937-1}Vail omits an S in the title. The correct title is "Punch, Brothers, Punch" (1876) or alternatively titled "A Literary Nightmare." The story tells of a musical jingle that gets lodged into the mind of the tormented narrator until they are able to pass it to a friend, who in turn is tortured by the continuous presence of the song in his mind until he can unload it onto a group of "poor, unthinking students" at a university. Vail, by his description, had undergone a similar monomania. The piece originally appeared in the February issue of the \textit{Atlantic Monthly}. See item 3365 in \autocite{blanck_bibliography_1955-1}} Vail writes of Bradbury, "We have not a thing in our Library by him and I can find no record of his writings outside of the LC Catalog and two entries in Williamson's Bibliography of Maine, Volume I, p. 186." Vail then lists Bradbury's \textit{Metallak} (1844) and \textit{The Four Elders of Maine} (1856) with a mention of possible title, "Empress of Beauty." What Vail can supply however is some biographical information instead of bibliographical, noting that Bradbury was a member of the Maine legislature in 1838-9, married Mary M. Dinmore, and died at "nearly 90" years old.\autocite{robert_g._vail_letter_1937-1} While Vail is remorseful that he could not supply more bibliographical information to Wright, the response he gets is both cheerful and boastful. 

\begin{figure}
\includegraphics[width=\textwidth,keepaspectratio, angle=270]{osgood}
\caption{Writing on the back of Wright's 12 Dec. 1937 letter recording the names of titles and AAS' holdings of Osgood Bradbury titles.}
\end{figure}

Wright responds on Dec. 29, 1937, opening with a colloquial "Man, oh, Man! Did you hand me a nice present in your last letter." Here was Wright's time to show off his own bibliographical skills. Wright hinges on Vail's mention of the Bradbury titles recorded in Williamson's Maine bibliography (despite the fact that he was a "Massachusetts Man") and argues the case for the fact that the AAS possesses four of Wright's sixteen total record titles by Bradbury. Narrating a story of finding an old Library of Congress catalog card for a title "The Spanish Pirate" that had been attributed to Bradbury and subsequently then looking for the title page in the copyright offices in order to verify the attribution. The title page does not name Bradbury, but instead claims "by the author of "Helen Clarence, "Julia Bicknell, "Emily Mansfield, "The eastern belle" [sic]. Following the same "by the author" statement on the \textit{The Eastern Belle} led Wright to the \textit{Mysteries of Boston}, of which the AAS did hold a copy. Since the Maine bibliography provided Wright additional titles, and one of them possessing the same characters as \textit{Mysteries of Boston}, Wright concludes from following the string of "by the author" statements on title pages a list of potential Bradbury titles, several being in the possession of the AAS.\autocite{lyle_h._wright_letter_1937-3} Wright's thesis here assumes the fact that the two titles, \textit{Mysteries of Boston} and \textit{Louise Kempton}, share the same characters means they are by the same author. With this trail of evidence presented, Wright defers to Vail's judgment as to whether such an assumption is fallacious or viable.

Vail affirms Wright's conclusions, however, andwith Vail's validation, Wright assigns the anonymous titles to Bradbury.\autocite{robert_g._vail_letter_1938} Within Wright I, Bradbury receives twenty-seven entries, all the ones included in the written list in Figure 2.1 being included aside from \textit{Ellen, the Pride of Broadway} and \textit{Julia Mansfield}, as they were published after 1850. What we can learn, however, from this is that the process and the labor by which these titles enter the bibliography come from a sustained conversation and relationship that has built up over time, rather than the isolated work of a bibliographer in an archive. The case of Bradbury, as the other decisions Wright has made in terms of composition of his work are rooted in the suggestions and advice of Vail, whom he not only trusted but also actively sought advice from as both an individual and the head librarian of a major American materials archive. Both Vail's status as well as his own bibliographical ideologies suffused the Wright bibliography, affecting some of its basic premises in terms of its definition of fiction, scope, and exclusions. In the two revisions of Wright I (1948 and 1969), the Bradbury titles listed remain unmodified save the inclusion of numerical identifiers and the addition of more Bradbury titles; at no point are the original titles appended to Bradbury by Wright removed or ascribed to different authors.\footnote{Worth noting is that the foundation of Wright's argument involves assuming the presence of the same characters between \textit{Louise Kempton} and \textit{Mysteries of Boston}. In the first revision of Wright I, a second edition of \textit{Louise Kempton} is noted as found in the collections of Yale's James T. Babb. Wright's description of this edition does not note that the edition is printed as anonymous, thereby possibly proving Wright's assumption correct should he have found a copy of \textit{Louise} that listed Bradbury as author. However, based on Yale's MARC record for this particular edition (no. 385 in Wright I 1948) the edition is also anonymous and merely gives a "by the author of" statement in place of Bradbury's name despite what Wright's description would hint.}

Wright's claims here are uncontested insofar as his attributions for Bradbury are concerned. Wright's inferential work concerning these anyonymous titles is treated as an authoritative statement. Wright was helped to his conclusion by Vail and the AAS' resources, but the AAS benefited from this exchange. Currently, the AAS respects Wright's designations, and their own cataloging recognizes Bradbury as the author of these works, but does not note that many of the texts attributed to Bradbury are printed with the author's name absent. Thus, Wright's descriptions have, for the purposes of the AAS' cataloging, superseded the physical evidence of the texts themselves for both librarians and the researchers who locate these works, and are read as declarations, rather than arguments. The AAS' cataloging encodes in their records Wright's claims and thus subjects researchers to it when they happen upon the Bradbury titles and seek more information. These researchers, if relying upon the data provided by the AAS, are able to locate the source of the MARC record's information, as the Wright number for those listed in Wright I and II are provided for most entries. Should scholars wish to expand beyond the AAS' designation, however, they would still meet with Wright's influence, as his designations would be found in Gale Cengage's dataset "Crime, Punishment, and Popular Culture, 1790-1920," which the AAS records link to for several Bradbury titles.\footnote{See for the example the AAS online catalog for Bradbury's \textit{Julia Bicknell; or, Love and Murder} (1845, Wright no. 381). The physical text is anonymous save for a "by the author of" statement. This text is, nevertheless, still attributed to Bradbury based on the logic supplied by Wright. The text is further shown to be in Gale's "Crime and Punishment" dataset and linked therein with the same bibliographical information as supplied by the AAS, thus proliferating Wright's designations beyond a singular institution. Catalog Record \#189975, accessed October 18, 2017. https://catalog.mwa.org/vwebv/holdingsInfo?bibId=189975.} Similarly, attempting to find copies of such works beyond the AAS in a repository such as Hathitrust, would bring up facsimiles of editions of works such as \textit{Louise Kempton} (1844) or the \textit{Belle and the Bowery} (1846) that are anonymous, but attributed to Osgood Bradbury in their metadata and their search results despite the lack of the author's name on the pieces.\footnote{HathiTrust is an open acccess repository of digitized texts, mostly drawn from university library holdings. Hathitrust's images of \textit{Louise Kempton} does includes Bradbury's name written in pencil at the top of the title page. See \autocite{bradbury_louise_1844}.} But while in the case of Bradbury, Vail serves as an identifiable source from which the explanation and verification of Wright's claims, and the subsequent adoption of those claims by institutions, he is not the lone actor in Wright's adoption and proliferation of his contributions to bibliography, and American literary history. The AAS formally worked with Wright's materials from the beginning as a means of incorporating his work into their own mission and means of framing their collections. Vail as one of the original supporters of Wright's work is instrumental in the creation of \textit{American Fiction}, but the organization for which he spoke, was no less interested in his work.

\section{Traces of Wright in the Reading Room}

The AAS would remain involved in the production of Lyle Wright's work after he returned to the Huntington and began the process of composing \textit{American Fiction}. Wright would rely on AAS librarians to help amend the first volume of \textit{American Fiction} as he was revising and sorting through the wealth of titles and descriptions he had accumulated in his research travels. It was as Wright was composing the first volume that a shift in the relationship between Wright and the AAS occurred. Now that Wright had acquired the information he needed and was arranging it, the AAS was able to use his work as a resource for improving their cataloging and holdings. Similar to the case of Bradbury Osgood as discussed in the last section, the AAS would continue to use Wright's work for their own benefit, just as Wright used the AAS for his. The AAS possesses several copies of \textit{American Fiction}, one set of which is present in the AAS Reading Room alongside other notable bibliographies, such as the Sabin \textit{Bibliotheca Americana} or Blanck's \textit{Bibliography of American Literature}. The significance of Wright's presence in the Reading Room is in the fact that the books present there need not be "checked out", but instead may be picked up by scholars working there at will. Most of the works physically present in the Reading Room, instead of in the Society's stacks, are multi-volume bibliographies and reference guides that will help scholars locate works pertaining to the subject of interest to the scholar. Placing Wright in the Reading Room indicates his authority, as determined by the AAS, for scholars working in American literature. But outside of the volumes at hand to scholars, more copies of \textit{American Fiction} exist within the building's interior stacks and those copies bear markings of the AAS, as an institution, reading the copies of Wright and making their own revisions, clarifications, and interpretations of Wright's work. 

This section will discuss these copies of Wright that the AAS holds and their position within the Society as tools for both scholarly and institutional use that further cements Wright's contributions as fundamental to the development of American literary collections and, thus, scholarship. The AAS' adoption of Wright as an authority first took the shape of reviewing and using the bibliography as a tool that helped the AAS to navigate its own collections. \textit{American Fiction} aided the AAS in terms of identifying texts the AAS possessed and the classification of those texts (i.e. as either fiction or not). However, Wright's work was also useful in showing how the AAS' collections could be improved and expanded by presenting a list of titles the AAS could pursue. This process began by assisting Wright shortly after he returned to the Huntington to begin compiling \textit{American Fiction}. Beyond supplying Wright access to titles that would be listed in \textit{American Fiction}, the AAS staff would also perform the role of peer-reviewing and editing \textit{American Fiction}'s rough draft, a task that required a significant amount of labor and resources.

In 1938, after Wright had returned from his east coast tour of the various libraries, including Yale and the New York Public library--suggestions of Vail--he prepared a checklist to send to the various libraries he had visited. Because Wright intended to provide a census of libraries that held a given title in his bibliography, he invited the libraries he had visited to report what titles they held. This checklist, a mimeograph of shortened titles (i.e. not full descriptions) that Wright was planning on adding to the first volume of \textit{American Fiction}, provided a means of not only collecting the information for his census in a systematic way by deferring to the institutions surveyed rather than relying solely on his own notes, but also allowed for institutions to provide feedback for Wright in regards to the titles he intended to list. The checklist includes titles which would eventually be removed for the published version, including Edgar Allan Poe's "The Balloon Hoax" (1844). The checklist as a piece of physical evidence however, shows the ways of engaging with one of the initial drafts of the document that would become \textit{American Fiction}, and demonstrates how much Wright relied upon the people that staffed the AAS in the direct composition of his list as much as he relied upon the texts held there.

The cover page of the document is not dissimilar from the preface of the published volumes of \textit{American Fiction.} It states its purpose clearly though does ask for the information from the AAS staff more explicitly. Under the "Purpose" heading, Wright lays out his expectations for the bibliography: 
\begin{displayquote}
The bibliography will contain novels and separately published short stories or collections of short stories. It is intended to omit: essays, annuals, folklore, juveniles, American editions of foreign works, jest books, those stories published by religious organizations and fictitious Indian captivities (Mr. Vail has in hand a bibliography of Indian captivities). 
\end{displayquote}
The prefaces of the published volumes add and clarify a few more categories of what is excluded in the list: giftbooks, "collections of anecdotes," periodicals, and extra numbers of periodicals are added to the list while the term "religious organizations" becomes the American Tract Society and the Sunday School Union. Wright additionally clarifies that the foreign-born authors included in the published bibliography are included because they "claimed the United States as their home." \autocite[ix]{wright_american_1939} The most important statement with regards to the AAS, however, refers to the absence of the fictitious captivity narratives that are excluded categorically from this checklist and subsequently in every volume and edition of \textit{American Fiction}. Wright claims that Vail has "in hand a bibliography of Indian captivities." This claim is disingenuous if not false, as Vail's bibliography of captivity narratives would not be published until 1949 as \textit{Voices of the Old Frontier}. Wright's reference to Vail, however, lends an explanation as to why \textit{American Fiction} would exclude a genre of Early American literature that was immmensely popular in its day. In Vail's June 29, 1933 letter to Wright, he explicitly mentions the desire to create such a bibliography after he finishes his work with Sabin and Evans bibliograpies.\autocite{robert_g._vail_letter_1933} The parenthetical statement on the checklist suggests that Wright's avoidance of describing such a prolific and popular genre of the early Republic is because it would be redundant with Vail's work. The irony, however, is the status of Wright versus Vail's bibliography. While Wright's work has remained influential as a resource, Vail's work is less prevalent. This decision to exclude the captivity narrative outright comes across in the checklist preface as more personal than intellectual, given how close of friends Vail and Wright had become. The consequence of Wright's decision is that \textit{American Fiction} possesses a significant gap in its coverage of titles that necessitates either a scholar's own work to fill in, or cross-referencing with Vail's bibliography (and in that, sorting through the nonfictional titles, poems, letters, and sermons) to fill out a more accurate list of American fiction than what Wright provides. The effect of this is that the holistic idea of American fiction and the bibliography are not coterminous and sometimes that is more for human reasons than soundly logical or objective ones. The absence of captivity narratives, given Wright's suggestion, is rooted in a desire to not make redundant the work of a friend and mentor, rather than an interpretive judgment about captivity narratives. 

Beyond the purpose of the bibliography as Wright describes it on the cover to the checklist, he also expresses what it is that he desires of the AAS in this task of reviewing his list:
\begin{displayquote}
The compiler will welcome any suggestions, criticisms, or additional material. If an author is offered for an anonymous work, the compiler would appreciate the authority for the attribution. 

If a new title is furnished, information desired is, the author's name, if known, full title and imprint, pagination and format (by foliation).

The following mimeographed list is to be checked and returned to the compiler at the Huntington Library. Needless to say, the accuracy and usefulness of a bibliography containing a census is governed considerably by the cooperation of institutions and individuals. The compiler will appreciate and gratefully acknowledge all aid rendered in this project.
\end{displayquote}
\begin{figure}
\includegraphics[width=\textwidth,keepaspectratio, angle=270]{mimeo30}
\caption{Page 30 of a mimeograph of the checklist sent to the AAS for review and to take a census of the holdings of potential works to be listed. The note for Ganilh's \textit{Mexico versus Texas} reads "with reprod. of title in Freeman auction June 1, 1938, no. 326." referring to the auction catalog of Freeman's in Philadelphia.}
\end{figure}
AAS librarians who reviewed this manuscript did what Wright asked of them, labeling each title listed with the easily discernible "AAS" marker in order to inform Wright of their holdings. Figure 2.2 shows a sample page of a copy of the checklist currently held still by the AAS. The copy accessible to Reading Room patrons is pragmatically labeled as a "Second Copy," the first obviously having been returned to Wright. Interestingly, however, this copy is still marked in pencil to annotate the document, rather than mimeographed or carbon copied. The labor originally conducted for the version returned to Wright is also performed, though perhaps not in a perfectly replicated way, on the copy retained by the AAS for its own use, indicating some personal investment on the AAS' part for recording their stake in the bibliography's composition.

From this copy, some instances of that revisions the occurred between the checklist copy and the first edition can be gleaned, several of them at the suggestions of the AAS staff. For a title by Maturin Murray Ballou, \textit{Fanny Campbell; or, the Female Pirate Captain} (1845), the checklist originally includes only an 1845 edition. Penciled under it, however, is the shorthand notation to point the existence of an 1846 edition: ---- ---- `  ` 1846. This additional title is included in the first edition of Wright I with a more complete description that notes a change of publisher and the removal of a pseudonymous author on the title page. The two descriptions for the titles are shown below: 
\begin{displayquote}
---- Fanny Campbell, the Female Pirate Captain. A Tale of \\
\hspace*{2 pc}the Revolution, by Lieutenant Murray [pseud.] Boston: F.
\\
\hspace*{2 pc}Gleason, 1845. 100 p., illus. 8vo  \hspace*{6 pc} H, LC, Y
\\
\hspace*{2 pc}{\small Printed in double columns.}
\\
\\
---- ---- Boston: United States Publishing Company,
\\
\hspace*{2 pc} 1846 100p., illus. 8vo \hspace*{6 pc}     BU, LC
\\
\hspace*{2 pc}{\small Printed in double columns.}\autocite[16]{wright_american_1939}
\end{displayquote}

The entry written on the checklist does not note why it would appear as a separate entry, but the differing details--publisher and pseudonymous signature--address why it is listed separately as an entry. Wright, for the first volume of \textit{American Fiction}, intended to list every possible edition of a work if he could locate it.\footnote{Wright abandoned this plan for the the second and third volume of \textit{American Fiction} when he discovered how much the U.S. publishing industry had grown in the mid-1800s, making it nearly impossible to list every edition in a practical manner.} Neither title, however, is marked as owned by the AAS, suggesting either external reference by the librarians for additional titles, or that some of the notes were written after the publication of Wright I. Given the document is dated December 24, 1937, however, it would suggest the former--that the librarians consulted other sources of information, either other bibliographies or catalogs, to inform their approach to the checklist. In either case, the markings also suggest a means of reading the document anticipating what Wright himself would desire in terms of the information he lays out in the purpose. Wright makes no mention of the desire for only a single, earliest possible, edition to describe an individual title (a quality that would define Wright II and III), but the evidence of another version of \textit{Fanny Campbell} would be welcome given Wright's goals of wanting a comprehensive listing of editions. 

The AAS' reviewers of the checklist also made efforts to encourage removal of titles from the final bibliography. Such examples include two anonymous titles, \textit{Rachel, a Tale} (1818) and \textit{The Warlock, A Tale of the Sea} (1836), whose entries are struck through on the checklist. The advice was taken, as evidenced by these titles' absence in Wright I. Investigation into the titles, however, reveals the reason: \textit{Rachel} was written by Jane Taylor (who also wrote the lyrics to ``Twinkle, Twinkle, Little Star'') and \textit{The Warlock} is attributed to Matthew Henry Barker, an author of nautical tales.\footnote{Taylor is named the author of \textit{Rachel} by the University of California-Berkeley. \textit{The Warlock} is attributed to Barker; the Hathitrust facsimile of the text has Barker's named pencilled onto the title page, though not encoded into the catalog. The reasoning for the attribution seems to follow the same logic as Wright's assignment of Bradbury titles--all works written by ``An Old Sailor'' are assumed to be the work of Barker. For \textit{Rachel}, see \autocite{taylor_rachel:_1817} For \textit{The Warlock}, see \autocite{old_sailor_warlock:_1836}} Both of these authors are British, not American. More than a simple suggestion of removal, what is revealing about these recommended (and ultimately adopted) changes to the bibliography is the labor required of the AAS staff to recommend these changes. Investigation of individual titles to reinforce the bibliographical accuracy, especially for a list of nearly 2000 titles, shows how much more than a catalog reference the review of this checklist was for Society's librarians. In the case of \textit{Rachel}, the AAS did possess a copy and marked themselves as owning it before striking it through to be deleted, but in the case of the \textit{The Warlock}, the AAS does not indicate that it possessed a copy, and so its suggested removal points to even larger amount of work necessary to confirm it as an inappropriate entry for Wright's bibliography. Their initial inclusion relies on the fact both texts were printed in Philadelphia, yet their authorial attributions, at the least, could have required research into British reference materials and bibliographical work to confirm the texts as not nascent American literature. 

Further hints as to the forms of effort put in by the AAS staff in the review of the checklist can be seen more specifically in Figure 2.2, with the ways the annotations of the document illustrate a means of how the data presented is reviewed. For Anthony Ganilh's \textit{Mexico Versus Texas} (1838), an annotation is supplied that reads: "with reprod. of title in Freeman auction June 1, 1938, no. 326." in which a specific location for a reproduction of the title is reported. This information is subsequently not included in Wright, but is nonetheless recorded by AAS and validates the existence of the title and partially pointing to a text that could warrant an additional description in the bibliography. The specificity of the citation though indicates, again, how committed the AAS staff at the time is to the verification of Wright's work, wherein the research takes the reviewer of the checklist to materials that are not necessarily meant as scholarly reference tools.\footnote{Though it should be noted, looking to auction and bookseller catalogs, though they are intended for commercial use rather than scholarly use is not an unheard of practice. Bookseller catalogs in general endeavor to describe books with a similar level of detail as bibliographers, though different standards of description exist between the two spheres.}

Also seen in Figure 2.2 is the strike through of the interestingly titled \textit{Ghost Stories; Collected with a Particular View to Counteract the Vulgar Belief in Ghosts} (1846). The title in question is anonymous but the edition described here contains engravings by American artist Felix O. C. Darley and was published in New York. Another edition of the text exists as \textit{Curious Stories}, published by James Miller in New York in 1865 and 1867. The 1867 Miller edition reprints the text and engravings, but also includes advertisements at the end to a list of "New and Attractive Juveniles" (Figure 2.3).\autocite[222]{the_library_of_congress_curious_1867} The presence of the juvenile context gives a clue as to how the work \textit{Ghost Stories} was read and considered in context of Wright and why it was recommended for exclusion. The tales themselves, as ghost stories, are clearly fictional, or at least fictionalized given that some feature real historical people (for instance, the French poet Antoinette du Ligier de la Garde Deshoulières is the primary character of the story "The Ghost of Larneville"). The introduction of the text asks "What is a ghost?" and proceeds to define but also critique the notion of ghosts in a didactic manner by continuing to ask  questions such as why do ghosts appear in clothing, rather than naked, and declaring such things only "exist in the imagination of the beholder."\autocite[5]{new_york_public_library_ghost_1846} But the text of book does not offer any direct mention of its audience being children readers, and in fact, mentions statements that would put it at odds with a young audience: "The best way to dissipate the inbred horror of supernatural phantoms, which almost all persons derive from nursery tales or other sources of causeless terror in early life, is to show by example how possible it is to impress upon ignorant or credulous persons the firm belief that they behold a ghost, when in point of fact no ghost is there."\autocite[6]{new_york_public_library_ghost_1846} Such a statement distances itself from juvenile readers, and instead seeks to undo the damage of juvenile superstition after the fact. The advertisements at the back of the Miller copy are the only real clue as to why the text of \textit{Ghost Stories} was excluded from list.  The description itself in the checklist is without any note from the AAS as to why they believe the item should be removed, only that they did indeed suggest it and Wright followed through, never reincorporating the text into Wright I. Given the previous effort of the AAS librarians for specific titles, their knowledge of the Miller edition that would lead them to the conclusion that the text was a "juvenile" seems probable. The AAS, at the least, became aware of the Miller edition by 1951, when in one of their annual reports they featured a discussion of titles illustrated by Felix O. C. Darley, wherein they note the change of title and publisher when the work moved from \textit{Ghost Stories} to \textit{Curious Stories}.\autocite[147]{theodore_bolton_book_1951}
  
\begin{figure}
\includegraphics[scale=.8]{curious}
\caption{The end pages of the Miller edition of \textit{Curious Stories}, also titled as \textit{Ghost Stories} in the Wright checklist. Juveniles are advertised here suggest the work itself was read as juvenile and thus unsuitable for \textit{American Fiction}. See note 43.}
\end{figure}
 
The checklist represents a level of institutional engagement in Wright's project in the time before it is published. From the markings of the checklist, we can see the commitment of the librarians to the composition of \textit{American Fiction} and to the parameters Wright set forth. Their suggestions of which titles to add or remove are based on their understanding of Wright's goals. But the AAS' commitment to Wright does not end with the publication of the first edition of \textit{American Fiction}'s first volume. The AAS reads and annotates the published forms of the bibliography much in the same way they do the checklist. With the first volume, Wright revised it twice, once for a 1948 edition, and then for a 1969 edition. The AAS possesses copies of each edition, and used them, and marked them, continuing to improve the contents. 

For Wright I, the AAS holds two copies of the 1939 edition, two copies of the 1948 edition, and one copy of the 1969 edition. One copy of the 1948 edition is placed in the reading room for easy access to patrons for reference. The other copies, however, are held in the stacks and bear traces of the AAS staff reading and referencing the titles. The most prevalent sorts of notes present in these various copies are the markings of AAS, or sometimes MWA (the Society's Union Catalog code), beside titles that are not printed by Wright as in the AAS' library, in a similar manner to the checklist the AAS originally marked.\footnote{The Union Catalog code is primarily used by libraries to identify holdings, especially when referencing other libraries or interlibrary loan. The MWA code is a reference to the location of the American Antiquarian Society--Massuchusetts, Worcester, America.} The continued marking of various editions of the titles with the idea of keeping the AAS' holdings in the context of Wright's American fiction shows the continued use of the bibliography to the AAS staff in order to determine the status of their own fiction collections. The collections of others are also noted at times, especially the holdings of Yale and the private collection of its head librarian, James T. Babb (who, coincidentally, was also a member of the AAS). These holdings too are marked in pencil in the document, comparing the developing collections of the chief rival of Wright holdings for the AAS. 

AAS librarians were not necessarily interested in only treating the text as an authority for which to identify or reconcile their own collections and holdings, but also as a space for adding and considering other American fiction titles not listed. Appendix A lists within 3 different AAS copies of Wright I the substantial additions and recommendations written physically into the bibliography. In one of the 1939 copies (cataloged as Backlog 19C 1325), AAS staff write in 159 new titles to the document in addition to the printed entries. Some of these entries are not within the parameters set by Wright for the volume; titles such as Edward Zane Carroll Judson's \textit{Clarence Rhett} (1866), Eliza Leslie's \textit{Sketches} (1854), and Maria Jane McIntosh's \textit{Violet} (1856) are all outside of the boundary of the years covered by Wright I, all having been published after 1850, but nonetheless have their descriptions written into the text. The additions as a whole are written into their proper places with regards to alphabetical ordering, but they do not wholly conform to Wright's parameters as he describes them in the preface. The three aforementioned titles have other titles by their authors listed or included by the AAS, and thus show the bibliography's parameters to not be a limiting feature of the bibliography by its readers. Instead, the AAS treated Wright's volumes as a space to organize more specific types of information, in this case, to easily see all titles discovered or noted by a specific author. Wright's work enabled the AAS in this sense to organize and understand the bibliographic information they discovered in a space that made to present American fiction titles. 

In a second copy of the 1939 edition (cataloged as RefT Fiction 05a), the handwritten annotations continue, but this time incorporating 109 of the previous texts written into the Backlog 1939 copy. New titles are added as well, further enlarging the titles possible for Wright I as well as more entries that may do not fit within the parameters. This copy of Wright I also includes 87 new titles for both reference and possible inclusion in a revised edition. Some descriptions added by the readers of the bibliography  include more in-depth bibliographical discussions beyond their entry descriptions. As an example, for an anonymous title, \textit{Errors of Education} (1810), an annotation to the description attributes an author, Jesse L. Holman, based on another 1810 title \textit{Prisoners of Niagara, Or Errors of Education}. Other descriptions perform acts of reinforcing the evidence available, such as for a title by  Justin Jones, \textit{The Doomed Ship} (Philadelphia, n.d.), which is included in the Appendix by Wright but are written into the context of the list by AAS staff, thereby suggesting the acquisition or location of a physical copy or a title page that attests to the text's existence.\footnote{Titles are placed in the Appendix by Wright when there were no copies located, but advertisements or other sorts of information may have alerted Wright to the existence of such a title.} This particular title is eventually moved from the appendix of Wright I to  an entry of Wright II, \textit{The Doomed Ship, or, Wreck of the Arctic Regions} (1964, Wright II no. 1384), with physical copies noted at the Boston Athenaeum (which lies in close proximity to the AAS) and the University of Pennsylvania.\autocite[186]{wright_american_1957} 

The third annotated copy of Wright I to be discussed is one of the 1948 revised editions (cataloged as RefT Fiction 05b), which features an additional 66 titles described by the AAS in addition to Wright's own extensive number of inclusions to the list (approximately 600 more, deletions notwithstanding). But this copy also features a shift in the tone of the annotations, as these annotations bring attention to entries Wright has \textit{omitted}. Rather than simply "not listed," the use of omitted implies the active sense of Wright's process of composition. Why Wright has chosen to omit certain titles written back into the bibliography pertain to Wright's own parameters. Titles described as omitted here come from foreign authors, were originally published as serial extras, were Sunday School publications. In one case, the staff seems more quizzical than anything about a choice of Wright; again Justin Jones is a point of contention where his 1849 title, \textit{Osmond the Avenger}, is written into the list with the annotation "why omitted?" The inclusion of the title is apparently in defiance to Wright's own parameters, as if the writer of the annotation is simply confounded by the possibility of Wright making a mistake. In another case, the entry for the \textit{Trapper's Bride} (1848), described as authored by Emerson Bennett, features annotations that point to an argument over the actual author; Charles Augustus Murray is also suggested as a possible author. This argument bleeds into the AAS' online catalog as well, which takes care to represent the argument in its records, under the "notes" heading: 
\begin{displayquote}
Probably written by Emerson Bennett, author of the novella ``Prairie flower.'' Sir Charles Augustus Murray is author of ``The prairie bird'' but unlikely to be the author of the present work.
``Very doubtful if by Murray, issued probably to take advantage of the popularity of The Prairie Bird.''--Library of Congress.
\\
``This has been ascribed to Sir Charles Murray ... But the Cincinnati imprint points toward Mr. Bennett. This is written in Emerson Bennett's style, very unlike Murray's writings ...''--Henry R. Wagner, The plains and the Rockies, 3d. ed., rev. by C.L. Camp, 1953, no. 145.\autocite{bennett_trappers_1848}
\end{displayquote}

Again, it is important to highlight the amount of labor that was invested into annotating these copies of \textit{American Fiction}. These copies, while cataloged as reference materials, were not necessarily in circulation or stored on the publicly-accessible shelves in the Society Reading Room. The copies of Wright there are clean of any markings by the AAS staff. It is those that are marked that are shelved in the stacks and are not necessarily intended for general use by patrons. These copies, instead, are meant to help the AAS staff and their collections. The data described by the ones annotating these copies of Wright eventually make their way to the researching scholar, as demonstrated by the case of the \textit{Trapper's Bride}, wherein the bibliographic controversy becomes an inherent part of the metadata of the object. Annotating copies of Wright becomes an exercise not just in referencing the information contained therein, but in using that information to build a conception of American literature that can be useful for the institution and for the scholars it serves. By adding to the text of \textit{American Fiction}, AAS librarians were trying to turn their copies of AAS (at the least) into as comprehensive a resource as possible. Over the course of the three editions, the AAS contributes a total of 312 annotations that added entirely new titles, corrected, or otherwise modified the existing entries. Each of those entries represents a task that recognizes the individual title and considers it against the whole of \textit{American Fiction}, and the ways AAS wishes to use the text. 

\begin{figure}[h]
\includegraphics[width=\textwidth,height=\textheight,keepaspectratio,angle=180]{graft3}
\caption{A portion of Smith's addenda physically grafted into the AAS' copy of Wright's 1969 edition of \textit{American Fiction, 1774-1850}. See note 54.}
\end{figure}

\begin{figure}
\includegraphics[width=\textwidth,height=\textheight,keepaspectratio,angle=270]{graft1}
\caption{A portion of Smith's addenda physically grafted into the AAS' copy of Wright's 1969 edition of \textit{American Fiction, 1774-1850}. See note 54.}
\end{figure}

The physical ways the labor of the librarians manifests in the reading of a bibliography can even extend beyond annotations. A copy of the 1969 edition of Wright I, the last revision to come out, is not simply just annotated, but also contains published addenda to Wright's work, including authorship attributions and suggested inclusions, that are physically grafted into the book itself (Figures 2.4 and 2.5). Both selections seen in the figures are drawn from Nolan E. Smith's addenda originally published in the Papers of the Bibliographical Society 65.4 (1971), but have been cut out of the journal and glued to the gutter of the book near where the original entry or the author would appear in Wright. Smith's other addenda are also found grafted into the book as well, the AAS staff taking his evidence and incorporating it to make the physical copy possess the same information as Smith argued for. 

In Figure 2.4, Smith is arguing that entry 479a, \textit{The Orphans}, described by Wright without an author should be attributed to William Samuel Cardell, alongside Smith's justification. Figure 2.5, however, is not a correction but an assertion of the need to include a title withing the context of \textit{American Fiction}; Smith claims of Abby Goddard's \textit{Gleanings} (1856) is "clearly a work that should be included" in Wright (though, it would be Wright II rather than I where the AAS has grafted it). Smith's argument, however, is not robust,  as he simply states the stories are of a kind "that would qualify it for inclusion" in Wright.\autocite[407]{nolan_e._smith_author-identification_1971}
 The AAS librarian who viewed Smith's notes here, however, seemingly agreed and committed the note to the text of Wright as a means of including the work with the full description and the justification. The AAS here has not even relied upon its own institutional holdings, knowledge, and research to further refine Wright's bibliography, but also the work of others, which they compile and reincorporate into the body of the bibliography as a whole to further build upon their referential knowledge. These graftings, though, create a document which attempts to allow for all of that knowledge to be found in a single place, much as the annotations do. The affordance of the physical bibliography at play here is further realized when the desire to see as many ideas of what constitutes "American fiction" are visible. That is, the ideas of American fiction espoused by Wright are shown to be a living concept that is susceptible to change, either clarifications or emendations. The physical bibliography itself does not easily allow for such modifications, except through the conventions of publishing revised editions as Wright does, but grafting, while not an insignificant task of labor, presents a physical revelation of the moments where those changes occur, and how often they occur at the hands of someone external to the composition of the bibliography itself.\footnote{In Wright's favor, the revised editions of \textit{American Fiction}'s first two volumes do provide evidence of the revisions themselves explicitly by announcing when a specific entry was deleted. Wright does this to preserve the numbering conventions of his list. Additions in the revised bibliographies as well are obvious because of this, as they do not take a number from another title, even if it has been deleted. Instead, they are given the same number as the preceding entry but with a letter added to the end to denote its more recent inclusion.} 
 
The grafted information of Figure 2.4 becomes encoded into AAS's MARC record entry for \textit{The Orphans}, wherein the suggestion of William Samuel Cardell becomes fact by virtue of its designation in the author field and enabling searches of Cardell to return the title. However, \textit{The Orphans} is a strange case when considered next to titles such as \textit{Ghost Stories}. The full title itself denotes it as a juvenile: \textit{The Orphans: An American Tale Addressed Chiefly to the Young.}The MARC record for the title validates the subititles claims and complicates the nature of the physical grafting in Wright's bibliography. The field reserved to index genres (655) declares the title to be in the category of "juvenile fiction."\autocite{cardell_orphans:_1825} The concept of the juvenile emerges again as a way of disrupting \textit{American Fiction}'s order. Wright, in order to keep within his own parameters that he set forth,  should have never included, or at the least should have removed \textit{The Orphans} from the list. The claims of the text's juvenile nature is irrelevant to the physical object which augments the original entry with more information about the title. In effect, Wright's parameters are ignored but the value of the list itself is retained and asserted. This is in contrast to the assertion in the mimeograph for the removal of \textit{Ghost Stories}, where its juvenile aspect is the evidence of its suggested (and enacted) removal. After \textit{American Fiction}'s publication, the authority of Wright and his list supercedes the evidence the text provides as to its nature, and the AAS, interested in expanding \textit{American Fiction} follow Wright's lead.

After the publication of the bibliography, the AAS was invested in Wright but not in a way that was entirely congruent to Wright's own desires for the bibliography's principles. Instead of an endorsement of a more complex list, a more fundamental idea that was simply attracted to a collection of American fiction titles is instead what informs the apparent actions of the AAS in terms of its participation with Wright's bibliography after publication. The discussion of Wright's continued effect and this focus on his collection as a organizing idea in itself that guides the AAS takes us beyond the Society's reading room. 

\section{Wright, Acquisitions, and Rivalries}

The cases that have been discussed thus far with Wright's association with the AAS have been in more private or little-circulated ways that depict the influence of \textit{American Fiction} as subtle in nature. This is not to suggest, however, that Wright was unacknowledged publicly or was not noted to have a more overt influence on the Society after the publication of his bibliography. In this last section, I would like to discuss one of the largest overall effects Wright and \textit{American Fiction} had on the AAS. Wright would eventually become the default point of reference for the AAS librarians to discuss their acquisitions in American fiction. These acquisitions would even become the foundation for comparison to other libraries. We have previously seen hints of comparison in the markings of the physical copies possessed by the AAS staff when they would record the holdings of other libraries in their physical copies of \textit{American Fiction}. However, the AAS' comparisons to their rivals becomes more evident in the published annual proceedings of the AAS that circulate among the Society's members and beyond.\footnote{The AAS' annual proceedings are also available digitally and to the public on the Society's website.} These more public documents make AAS' connection and reliance upon Wright more visible, but also reveal in direct terms how Wright's influence affects the AAS' collections and the way in which the institution perceives itself as a research institute providing a service.

\textit{American Fiction} gets its first mention in the AAS proceedings shortly after its publication, in the 1941 proceedings, only two years after Wright I's publication. The bibliography appears in the "Report of the Librarian" section (where Wright will become a familiar sight), written by Clifford K. Shipton. Shipton discusses in his report the receipt of a novel titled \textit{Emily Hamilton, a Novel, Founded on Incidents in Real Life. By a Young Lady of Worcester County} (1803), the first novel printed by Isaiah Thomas   Jr., the son of the AAS' founder. Wright emerges in the context of the discussion as a site of an error in the novel's author attribution. As Shipton reports from his perusing of two letters included with the novel:
\begin{displayquote}
One of these letters was written to Isaiah Thomas, February 13, 1802, concerning Emily Hamilton and Miss Vickery's desire not to be known as its author. Dr. Charles L. Nichols in his Bibliography of Worcester attributed the novel \textit{Emily Hamilton} to 'Eliza Vicery,' an error which was copied by Wegelin, Sabin, and Wright.\autocite[270]{clifford_k._shipton_report_1941}
\end{displayquote}
That Wright made an error in attributing the author is not  the primary reason to highlight this first mention of his bibliography in the AAS' records, but instead to note the company with which he becomes associated, that is, the Sabin and Wegelin bibliographies. As previously discussed, Wright's own contribution to bibliography emerged from a desire to expand the list of Wegelin, which had only covered to 1830, and was deemed too limited to be effective. The Sabin bibliography represented a major bibliographic milestone of describing American publishing and its completion was a significant achievement for former AAS librarian, Robert G. Vail. The mention of Wright in this context here immediately denotes its assignation to an eminent place in the minds of the AAS librarians, as well as his continuation and relationship to previous bibliographic works (even though it is represented by the proliferation of an error). At least for the AAS, Wright was quickly perceived of as an authority and put into use as a standard reference tool  for investigating bibliographic questions.

The proceedings are more explicit about the status of the Wright bibliography in their institution as time goes on. By the mid 1960s and into the mid-70s, the AAS' proceedings, in reports from both the librarian and the council, would frame their American fiction accessions for the year explicitly in terms of their relationship to Wright. From these reports, it becomes evident that Wright's work gave to the Society not simply an academic resource, but an achievable and quantifiable goal to pursue. Marcus McCorison, who replaced Shipton as the author of the reports, writes,

\begin{displayquote}
This past year has been especially fruitful in rare and
unrecorded books of fiction. Those not appearing in Lyle
Wright's list of American fiction were:

\textit{The Female Land Pirate; or Awful, Mysterious, and Horrible Disclosures of Amanda Bannorris, wife and accomplice of Richard Bannorris, a leader of that terrible band of robbers and murderers, known far and wide as the
Murrell men. Cincinnati: E. E. Barclay; 1847. 28p. illus.}

This marvelous story is pure humbug, but it fits in very
nicely with the two Murrell books we obtained a year ago.\autocite[243]{marcus_a._mccorison_report_1965}
\end{displayquote}

McCorison similarly notes later in the same report that an additional twenty-nine titles were received that year, one of them being an unlocated Wright I no. 488 (Emma Carra's \textit{Estelle}, 1848).\autocite[244]{marcus_a._mccorison_report_1965} Similar statements are found throughout the proceedings. In 1967, McCorison notes that of the 268 titles of literature received by the AAS that year, forty were found in Wright.\autocite[237]{marcus_a._mccorison_report_1967} The 1969-70 report of the librarian is more enumerative, stating,
\begin{displayquote}
It was an unusually good year for ``Wright fiction''—that is, novels written by American authors and listed by Lyle
Wright. They were in volume I—119, 120, 215, 241, 245,
311, 369, 414, 557, 1240, 1335a, 1535, 1701, 1731, 1905 1/2, 2063, 2093b, 2158, 2503, 2557a, 2608, 2710b, 2754; and in volume II—170, 202, 404, 487, 565, 586, 1221, 1306b, 1610, 1898, 1927.\autocite[277]{marcus_a._mccorison_report_1970}
\end{displayquote}
Even for years when the AAS had lower volumes of literature accessions, Wright becomes a means of framing the collections; McCorison reports for the 1970-71 proceedings, a total of four different texts that were added to the AAS' catalog that year and were described by Wright in \textit{American Fiction}.\footnote{The texts were as follows: two works by Emerson Bennett, \textit{Leni-Leoti} (1849, Wright I-300) and \textit{The Prairie Flower} (1849, Wright I-304); \textit{Morton: A Tale of the Reovlution} (1828, Wright I-1924); and George Wilkes' \textit{The Lives of Helen Jewett, and Richard P. Robinson} (1849, Wright I-2714). \autocite[223]{marcus_a._mccorison_report_1971}} In all of these cases, the proceedings use Wright's bibliography exclusively. Unlike the first mention of Wright by Shipton, which places \textit{American Fiction} within the context of Sabin and Wegelin, Wright supercedes previous bibliographies as a reference tool and a framing mechanism to explain their acquisitions each year. As he is deployed in the proceedings, Wright's work serves as a checklist by which an easy comparison to other institutions and to itself becomes quickly understandable. By employing the bibliography as a checklist itself, the AAS is able to monitor its development in American fiction and report the advances it makes in its holdings when compared against the backdrop of a formidable collection of titles that both illustrates what the AAS \textit{had} via the census of the institutions at the time of the bibliography's publication, and what more it can still do via the accessible list of titles that are not described by Wright as being own the Society.

The consistent use of Wright by the Society to describe its American fiction collection demonstrates a means of reading the bibliography against the institution's own mission of being a place where American writing, history, and materials are reliably housed and accessed. The Wright bibliography with its census shows the institution's failings, even without marking absences in an institutions records as such; what is not there becomes as markedly important as what is, and the task taken up by the AAS is to fill in the gaps. This way of reading the list, however, opens up a means of comparison for the AAS and the other institutions Wright surveyed. If the bibliography itself is read by an institution as a holistic collection to develop, or to aspire to, all other institutions become competitors in terms of both the cumulative number of titles they possess or the individual titles they hold but other institutions do not. It is unsurprising then that the AAS does exactly that, by imagining a rivalry with other institutions, most particularly, Yale and their head librarian, James T. Babb. 

As early as 1953, the comparisons to Yale via Wright began; Clarence S. Brigham, in his report to the council in the annual proceedings, claims Yale as "virtually [the] only competitor" in collecting the Wright titles. This statement occurs amidst Brigham's announcement of Wright's intention to create the second volume of \textit{American Fiction}, covering the years 1851 to 1875. Brigham claims in this announcement that Wright lists 2800 titles in his 1948 revision of Wright I of which the AAS possessed at that time.\autocite[9]{clarence_s._brigham_report_1953} As Shipton claims, the AAS collected Wright titles "energetically," though the impetus for this policy was Brigham himself, who saw collecting the materials as a competition with Yale.\footnote{See \autocite[364]{clifford_k._shipton_report_1960} for the comment on the energetic acquisition of Wright titles. See \autocite[271]{clifford_k._shipton_report_1954} for Shipton's designation of the project as Brigham's.} Brigham offers a concession in his 1948 report to the council. In the context of recognizing the publication of Wright's revised edition of \textit{American Fiction}'s first volume, Brigham discusses both the importance of the bibliography to the AAS' collection but also its status in comparison with Yale:
\begin{displayquote}
The field of early American fiction is one in which this
Society's library has a highly important collection. Of the 2772 editions listed, the Antiquarian Society has a total of 1520, followed by Yale with 1218, Library of Congress with 1057, New York Public Library with 1016, and Harvard with 846. Although the Society for the moment leads Yale, it recognizes the fact that when the collection formed by James T. Babb, the Yale Librarian, with its 505 titles, is turned over to his Alma Mater, we will hold second place.\autocite[197]{clarence_s._brigham_report_1948}
\end{displayquote} 
While Brigham clarifies that his point in mentioning the rivalry is in demonstrating the capacity for the Society to fill out its collection quickly, noting that fifteen years prior to the report the AAS possessed only a hundred of the titles, the sense is also clear that the effort put forth in amassing Wright fiction titles is also in an effort to be competitive with other institutions and their respective American fiction collections. 

While the sentiment of Brigham in 1948 is more guarded  and less explicit about the institutional rivalry, Shipton is more explicit. In Shipton's 1952 report, the discussion points directly to the cold war between the AAS and Yale. Shipton, discussing the idea of advertising the AAS' collections, mentions the American fiction holdings and the AAS' rivalry with Yale. What perhaps began as a suggestion about using the strength of the AAS' fiction holdings as a possible topic of outreach becomes a paragraph more focused on Yale versus the Society:
\begin{displayquote}
One of the commonest complaints of visitors is that we
do not advertise widely enough to forestall their wasting their time pursuing their material through scattered libraries in ignorance of the fact that it is here gathered into one place and backed by unique bibliographical tools. So it is with the excuse of providing some useful data that I indulge in a little boasting about our holdings. Few people realize the strength of our collections of American literature. The new edition of Wright's \textit{American Fiction} shows that we have 1818 items, Yale has 1535, and Mr. Babb has 442 not held by the University. The competition between Worcester and New Haven has been fierce, and we are content to accept second place. The bitterness of the rivalry may be judged by the fact that we have made a practice of offering Yale our duplicates, and Mr. Babb has given us some of his.\autocite[147]{clifford_k._shipton_report_1952}
\end{displayquote}

The above quote relies on information provided by Brigham, referring to James T. Babb's personal library. Babb was the head librarian of Yale at the time of these AAS proceedings' publication, and he would indeed, as Brigham predicts, leave his collection to the Yale libraries.\footnote{Babb served the Yale libraries for much of his career. He was the leading librarian from 1943 until 1965, serving as an emeritus librarian thereafter until his death in 1968.} In addition to working at Yale, Babb was also a member of the American Antiquarian Society, having been offered his membership in 1946, amid the AAS' "energetic" collection of Wright fiction titles. When Shipton mentions the "bitterness" of the rivalry, his tone is jocular, stating that the fierceness of the competitions is such that the two institutions mutually benefit each other in their race by donating their duplicate titles. On the one hand, it keeps the missions of the libraries as services in mind by reinforcing further each of their potential offerings to researchers, but to take Shipton's jest as a supplement to any actual hard feelings is to miss how serious the AAS was about acquiring Wright fiction titles. 

Even though Shipton's statement concludes with a joke that diminishes any sense of ill will between the institutions, the rivalry continues to surface in arguably inappropriate places, such as in James T. Babb's own obituary in the 1968 AAS proceedings. The obituary is typical in that it memorializes Babb as a member of the Society, head of the Yale Libraries, an instrumental part of the establishment of the Beinecke Rare Book and Manuscript Library, among other achievements. It is in the second to last paragraph however, that a break occurs in the tone just before explaining the details of Babb's death. Here, the writer, only mysteriously signed as J.E.M., quotes Brigham's 1948 report to the council, with its concession that the AAS has taken second place in the race to collect Wright titles.\autocite[230]{j.e.m._james_1968} In the context of a memorial piece of writing, the mention of the AAS' falling behind in its collections due to a donation by the deceased Babb to the Society's competitor reads as an odd lament by J.E.M. that displaces Babb in his own obituary, opting instead to take a moment for the loss sustained by the AAS. That this lament appears in the official proceedings suggests the importance of the statement, even if it is irreverent, in informing the audience of the proceedings of the status of the AAS in its goal to develop the most formidable collection of Wright titles. While Shipton may joke about the bitterness between the two institutions, the fact that Babb's obituary bears a trace of the AAS' rivalry suggests more than a friendly rivalry between the Society and Yale, but actually a real anxiety felt by the AAS in maintaining its status as the central institute that affords the most comprehensive access to the titles listed by Wright.

\begin{figure}
\includegraphics[width=\textwidth,keepaspectratio, angle=270]{add}
\caption{A handwritten note, found amongst the correspondences of Wright at the AAS. This note counts the number of AAS, Yale, and Babb holdings of Wright titles and totals them in order to see the effect of Babb's donation of his personal collection to the Yale libraries.}
\end{figure}

Amongst the correspondences of Wright held at the AAS, there is a scrap of paper in cellophane that bears witness to the AAS' interest in their rivalry with Yale. The paper, seen in Figure 2.6, records on the left hand side a count of the number of Wright titles held by the AAS and Yale as well as James T. Babb (shortened to JTB), with the total for JTB being added to Yale. The writing records, as well, a few additional details about the comparisons between the two institutions; the writing on the right-hand side reads:
\begin{displayquote}
Just after publication 

Yale had 1218 titles 

JTB '' 505 not in Yale 

L.C. '' 1057 

NYPL '' 1016 

Harv. '' 846.
\end{displayquote}

The comparisons would seem to draw a distinction between two different points in time concerning the respective collections in question. Likely, the state of these institutions' American fiction collections around the time of Babb's death and the state at the point of the publication of the revision of Wright I in 1948.\footnote{I arrive at the conclusion that the statement ``Just after publication'' refers to the revised Wright I rather than the initial 1939 publication, because this was before Babb was the head librarian for Yale.} The top right corner reads: 
\begin{displayquote}
London omitted 

jests not mentioned 

Don't include doubles
\end{displayquote}

Presumably, these notes are instructions or guidelines to the means of counting the titles that the AAS followed to arrive at its totals. The bottom right corner shows the AAS total from the left column (i.e. the 1781) with the addition of other numbers correspond to the tally marks that litter the bottom of the page. What these marks correspond to is unclear, though it is interesting in the context of a set of calculations that show AAS' shortcomings in Wright titles. The addition of the tallied numbers places the number of \textit{American Fiction} titles above the count for Yale and the Babb donation by 133 titles. The inverse side of the paper includes more markings that contest Babb's claims and describe more specific information about certain titles. Written on this side is a statement of four titles possessed by Yale that appear in Wright's appendix: \textit{Gentlemen's Daughter} (n.d.), \textit{Kate in Search of a Husband} (n.d.), \textit{The Orphan Stranger} (1839), and \textit{Theresa} (1846). Additionally, it is written, ``Babb say 2053 for Y \& JTB. I say 1977,'' apparently contesting Babb's claim of the combined total of \textit{American Fiction} titles held by him and Yale.

The annotated copies of Wright's bibliographies demonstrated a commitment to understanding the AAS' own collections against other institutions, but evidence such as the checklist and obituary statement elevate the status of Yale above other collecting institutions. Wright's bibliography was not simply a guiding principle of the AAS' collections, but also a means to place the AAS in competition with another institution. The means of their comparison here, however, is more macroscopic than those of the annotations. While the annotated bibliographies were focused on line-by-line attention to specific titles, the individual title is less interesting in the context of a holistic collection and the total number reigns. The texts that point to the Yale rivalry focus little on the quality or individual value of titles, and instead on the aggregate group as they compare to one another following a common foundation of understanding. That is, the Wright bibliography represents a holistic text, and the AAS, Yale, and Babb collections represent attempts at recreating that text itself with the bibliography as a manifesto of sorts for what to collect. In this way, the bibliography as a list of books is more egalitarian in concept, as per the AAS approach to collecting the titles does not presume any particular value to one text over another, but rather, that any text, insofar as it is listed, is of value to the AAS and its mission. This approach to collections via the bibliography as a guiding line is impersonal, but does represent a certain stance towards the AAS' as a site of research. This sort of macroscopic approach to the titles does not suppose value because that is not the job of the AAS or its librarians, rather, it facilitates access, and the bibliography represents a means of knowing what to provide access to.

All of this information, however, revolves around the labor and work produced by Lyle Wright, whose initial visit and involvement with the AAS caused significant institutional change in the years following his first volume's publication. When Wright initially asked Vail of the holdings, Vail was confident Wright would find enough to admirably fill out his bibliography, however, the state of the collections shifted dramatically once the AAS saw the potential within Wright's work to fill out its own collections. In providing Wright with a space for research and a foundation for his own work, Wright provided the Society with a map to improving its own holdings. Wright moved from a mistaken identity asking a humble question about the viability of an American fiction bibliography to one of the formative voices of the AAS' collections of American fiction. Wright effected enough of a change within the Society that he garnered an admiring note in his obituary in the proceedings: "The consistency and accuracy of Wright's work has never been questioned, and items ``not in Wright'' occur so infrequently that they merely remind us of Wright's major achievement as a bibliographer."\autocite[314-5]{roger_e._stoddard_lyle_1981} The obituary is lauding but also sure to implicate the Society in Wright's success, mentioning his close friendships with Vail and Brigham as reasoning for why Worcester, MA was the "high point" of his research travels.\autocite[314]{roger_e._stoddard_lyle_1981} For the Society however, Wright's importance in fueling their own collections goes with only a slight nod, though it concludes the obituary, claiming that Wright contributed to the "life's blood of research--bibliography."\autocite[315]{roger_e._stoddard_lyle_1981} 



