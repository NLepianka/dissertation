\documentclass[course, english]{Notes}
\usepackage{parskip}
\usepackage{graphicx}
\usepackage{csquotes}

\setlength{\parindent}{0cm}
\raggedright

\title{dissertation notes}
\author{nlepianka}
\date{10}{12}{2016}

\newcommand{\n}{\scalebox{2}{\textbf{\framebox{$\aleph$}}}}

\begin{document}


\section{basic bibliography}
\subsection{bowers-bibliography and textual criticism-1964}

"Each fresh bibliographical breakthrough only discovers more areas for technical investigation before criticism can begin to make use of the newer findings; and thus the day for practical application of bibliographical hypotheses on any large scale is continually being put off." [4]

"This is a world the critic never made, a world to which a sound classical education and a First in Greats seem to have little pertinence. The technical concepts explored and the bibliographical terminology required appear to be as remote from literary criticism as the language and concepts of linguistics or of a natural science." [5]

\n --DH and cultural analytics seemingly have the same relationship as described here, though perhaps the insistence and maintenance of this relationship has been moved from the shoulders of the technically outcast (as with the bibliographers' insistence of their work as a science) to the critic (who may now insist that DH and distant reading are worlds apart from their work and practice).
	
\n --Ultimately, if Gailey is to be believed, the Cold War effect and the continued exercise of neoliberal/capitalistic thought in the university have driven both Bibliography and DH; Stanley Fish would agree probably.
	
\n --forward progression of time has meant that bibliography, textual scholarship, and the technical work of these fields has become complicated and ever the more fruitful to the study of canonical or traditional texts. The possibility of serviceable objects in these fields expand. See the work on Stowe's \emph{Uncle Tom's Cabin} that Wesley Raabe has discussed. Many of his primary sources are editions created well after Bowers' was giving these lectures.


\subsection{"evidence"}

"...developed a new investigative method of annual studies using a combination of enumerative, analytical, and historical bibliography to produce analyses of the book trade's total annual outputin 1668 and 1644."

cites McKenzie, "The London Book Trade in 1668" \emph{Words: Wai-te-ata Studies in Literature} 4(1974): 75-92.
	\begin{displayquote}
	--"innocent assumption that the locus of bibliography as a subject was \emph{the book}--any book--as a physical object (76). 
	
	--"great disparity between what analytical bibliography might infer, and what we can establish (81). 
	\end{displayquote}

\n --enumeratve bibliography as a form of evidence to pronounce and build upon book history and questions of scale- Suarez notes this as the 2nd phase of McKenzie's contributions to bibliography.

\textbf{Suarez}-"the 'half' of my five and a half phases of McKenzie's extending the evidence is his conception, promotion, and contribution to national histories of the book. A resolute proponent of intellectual synthesis, the McKenzie hated the tendency of bibliography to split things up, rather than to connect them--what he called in one article, 'our use of division as a function of analysis', 'the separation of sheep from goats, of chalk from cheese'. Advocating 'the unity of all bibliographical  enterprise', he envisioned that, in the History of the Book in America, or Britain, or Australia, or New Zealand, the whole would be worth more than the sum of the parts. It was McKenzie's hope that, when the fruit of so much scholarly labour was presented in the aggregate, new patterns and new understandings would emerge. He also recognized that, if read together, such national histories would enable comparative scholarship across traditional boundaries, a development McKenzie considered highly salutary for the future of the field." ["Extended Evidence", 40]
	\begin{displayquote}
	--McKenzie, "What's Past is Prologue" and "Stretching a Point" (found in \emph{Studies in Bibliography} 37.
	\end{displayquote}
	
\n --McKenzie seems to be the one to point to in the rise of distant reading, rather than Franco Moretti or Matt Jockers or anyone. A comparative analysis of Moretti and McKenzie would be important. Also, whoever Moretti got his ideas from- \emph{longue duree}, etc. 

\n --McKenzie understood enumerative bibliography to be more than a resource, and in fact, useful for studying history and assembling the patterns found within the aggregation of individual titles. McKenzie's realization of this, according to Suarez, points to his non-commital stance towards bibliography being about the book. In the same way, distant reading/cultural analytics, are also not exactly about the book- they reject the idea of the holistic textual object, as Amy would say. McKenzie did similarly.

\n --McKenzie seemingly thought of texts not as holistic objects at all, but parts of a further whole thing. He argues against division, yet recognizes the field has a pragmatic association with it. 



\paragraph {\textbf{Bowers-\emph{Bibliography and Textual Criticism} on Evidence}}
\begin{itemize}
\item {"the only applicable function of bibliography in such an inquiry is to attempt to distinguish which details of the printed text are compositorial and which are not, the purpose being to analyse the manner in which the printer's copy has been turned in to print." [8]}
\item {"the bibliographical interpretation, by itself, can seldom extend to the cause for the anomalies in the underlying manuscript, although it may provide the foundation for such a critical explanation" [9]}
\begin{itemize}
\item{\n Bowers draws a line between a critical argument about authorship (e.g., a new critical or new historicist inquiry) that is founded but not defended on the basis of bibliographical evidence.}
\item {"It is a serious fallacy, I take it, to confuse the nature of the interpretation with the nature of the evidence." [9]}
\item {\n Bowers' approach would seem to digress from McKenzie's then, as the sociological view of bibliographical, especially in relation to the rise of high theory, political and identity consciousness in literary studies, and so on, would indicate that the sociological view sees more possibility in the relationship between interpretation and evidence, if it does not already conflate the two from the outset when discerning that the manner of compiling and describing evidence itself involves interpretation.}
\end{itemize}
\item{"one of the soundest doctrines of textual criticism requires an editor to begin with the nature of the whole before turning to any individual part. In other words, the whole form must be determined before one comes to test any single word." [12]}

\end{itemize}

\section{on ocr}
\n --What the articles of Derg-Kirkpatrick, Klein, and Durrett tell us is that the questions and scholarship of bibliography, textual scholarship, and book history, even when not called such, are important and relevant to those who are approaching text artifacts from different fields, and that relevance is something our fields can embrace, discuss, and contribute to settings beyond the Special Collections. 

\n --You do in fact get a double layer of metadata and context for a HathiTrust facsimile or OCR plain text document, as the circumstances for who scanned the text, at what time of day they did, the computer specs and settings, but the new edition also accretes some of its copy-text's bibliographic properties, and not just its linguistic properties. The new edition will present and demonstrate a historical texts properties at the same existing as a new text with its own properties. This makes the bibliography of such objects possess a dualistic nature of attempting to compile all that we know of the source text and the new text that was derived from it.

\subsection{trettien}
 

\subsection{crane et al}
 "Google is providing immense scale but the scholarly significance is not so great as it might be: there is at present no way to understand what subset of the world’s knowledge that seven million volumes represents. Even if there were, scholars have no way of understanding in more than the most general way how the services that extract information from that collection work — what is missed? What biases are embedded in the system? Scholarship depends upon transparency, and we must be careful that we do not, in pursuing our immediate research projects, compromise our fundamental commitment to transparency." [1]


	-- {Does this beg the question that non-consumptive research then is not transparent or not as transparent as it could be? If the information is obfuscated in any way, is it unsuitable as an object for academic study?}
	
	-- {If the material book persists or announces its existence in non-consumptive files- are we dealing with an additional layer of transparency that otherwise might be elided by those disinclined towards textual scholarship and bibliography?}
	
 "if we could understand how to build a comprehensive collection of classical scholarship from the beginning of print culture to the present, we would know how to work with centuries of print publications on every aspect of human society and in every discipline and from every corner of Europe and North America." [6]
	
	--  {am not entirely sure about this point. This is one viable option for investigating a question and is valuable as a point on a timeline, but casting a wide net in order to cover a wide range of time periods of print culture might cause more generalizations to develop at the cost of specifics that pertain to individual developments in printing, printers and publishers, or historical moments.}

"we cannot thus rely on upon a centralized editorial structure to guarantee for us the consistency of what we find." [7]
			-- {this acknowledges my point above}
			
{so called "fourth generation" collections "seamlessly integrate image-books, accurate transcriptions, and machine actionable knowledge in various formats." Their use of the word accurate, of course, frames the material as the authority as what is "accurate"} [9-10]

 --{what Crane et al seem to imply in their discussion of the apographeme is that there is a primacy of the material vessel that transmit a text, which we inherit from print culture, but also through various physical media, and that the progress of a project such as there continues to tie the data they are working with to its material vessels, rather than abstracting the data away from it.}

 --{Since Crane et al also value transparency as a fundamental commitment of scholarship, the material vessel then, is considered by this group as an important part of that transparency. This puts them, in fact, very much in the role of bibliographers and textual scholars who value more rigidly that same commitment to transparency and acknowledgement and codependency of text upon its medium.}

"This apographeme constitutes a superset of the capabilities and data that we inherit from print culture but it is a qualitatively different intellectual space. In the mature apographeme, every canonical text is a multitext, with dynamic editions linked to visual representations of the manuscripts, inscriptions, papyri and other sources. In the mature apographreme, each source is linked to the background data that we need to understand it — a transcription, information about the particular type of Greek or Latin script and its abbreviations, about the monastery, print shop or Egyptian village that produced it, etc." [10]

--- {A large portion of their "Services for the humanities in very large collections" also note the primacy of the material: pagination, a coordinate system for page imagines, transcription data, basic areas of the page, structures, etc." 

"Fourth-generation collections allow us to design corpora that go far beyond limitations that we internalized in print culture." [23]
	
	-- {So, in some way, we desire to get beyond the the book but are not really able to- getting beyond the book really just means acknowledging the plurality of data in books and aggregating the various forms text may occur to further emphasize not the text, but the multiple ways a text can be carried.}
	
	--{On the other hand, the proliferation of literary data at a non-consumptive level could be seen as a way that distorts and tears the text away from the book in order to accomplish the intended goals of the 4th generation collections by enabling ease of access and operations to the collection as a whole.}

\subsection{strange et al}

"much of the variability is associated with the historical and contemporary resources of publishing houses, meaning that major metropolitan papers typically sit at one end of the legibility spectrum and smaller, regional papers sit at the other." [16]
	 
	 --{not source this claim, but previous citation is Arlitsch, Kenning and John Herbert. "Microfilm, paper, and OCR: issues in newspaper digitization." Microform and Imaging Review, 33.2: 58-67}

	  --{is called the Walworth Murder Project}
	
--remove line breaks as a part of their OCR cleanup, meaning they do tear some of the bibliographic properties from the linguistic text}
 --a{lso fucking used voice recognition software as part of their transcription process}

"As well as typing from original scans, we transcribed texts using a voice recognition program (Dragon Naturally Speaking 12), another option for the correction and input process. Typing was predominantly used, since it tends to be quicker than transcriptions of dictation for corrections. For inputting longer sections from scratch, dictation was slightly faster and more convenient to use. However, it tends to fail "silently", in that it substitutes unrecognized words with other words, which a spell-checker cannot detect. Typographical errors, on the other hand, are more likely to form non-words that spell checkers can identify. Dictation is also more likely to fail on names and uncommon words and proper nouns, precisely those words which the study is most interested in identifying." [30]

 "Overall, the cleaning of the data was not essential to achieving results of interest on the two-phase comparison task, since many significant words could still be identified. Still, there were substantial differences between the results of the clean and original datasets, as significant words were missed and "false positives" were generated." [45]


\section{big data}

\subsection{Underwood}
 "How far do you have to back up before you start seeing patterns that were invisible at your ordinary scale of reading? That’s how big your collection needs to be." 
("A more intimate scale of distant reading.")
	
	
	
	
	
	
\section{textual scholarship}
Terms by scholar
McGann:
	Deformance
	Mutation
	Development
	
Greetham:
	Contamination
	
Greg-Bowers-Tanselle?
	Corruption

Shillingsburg: 
	Reincarnate

Tanselle: 
	physical is tractable (Rationale of Textual Criticism)

--{The use of organic terms like mutation to describe textual change would help in positioning against the idea that automatically generated texts are non the concern of bibliography and textual scholarship. Rather, the mutation of texts that is seen in automated processes is just a stage in the text's life.}
		 {use of "life" is on purpose here, to further emphasize the text's anti-complicity to its mechanical nature}
		\\-- {can cyborg theory be useful in understanding the text? Hayles has probably considered this, honestly.}
		\\-- {evolutionary theory as a part of textual history can be important.}
		
-- On the topic of the automated vs. manual debate, see Ryan Cordell and his claim that the debate tends to overestimate human autonomy in traditional book publication and underestimate the human component of digitization. 
\\--from a bibliographical standpoint, critical bibliography is adept at seeing through such a simplified debate, as it has traditionally engaged in the process of emphasizing and addressing the intersection of human and mechanical in printing. Its occupation with such things, may have given way to faulty logic that now closely identifies material publishing as humanistic, but ignores the "automated" or at least stringently mechanical aspect of printing. The opposite is then seen to occur in critiques of big data, OCR, and other "automated" processes. 
		
		

\subsection{Shillingsburg-GtG}
Nuanced vocabulary:
book vs. manuscript
scribal vs. holograph
authentic vs. forged vs. reproductions.

\emph{script act theory}: 

What happens if a collection of manuscripts are bound into a book? Or a blank book gets written in? (Chapter 3 discusses this)
	 {Important to this will be the concept of text in its 3 common definitions: 1) physical object 2) lexical symbols 3) conceptualization (sometimes called the work).}
	 {I think right now, I will decide on championing the first 2 definitions primarily: since the idea of the physical is important to make a case, but the idea of the lexical symbols is necessary from a cultural perspective to indicate the presence or absence of a physical object- the lexical symbols point to their physicality and the ways in the which the book artifact existed.}
	 {the idea of the conceptualization as a definition of text seems weak to me from a bibliographic perspective, because the page signatures, among other aspects of a book's production, especially books produced after the literal death of the author bear no connection to the conceptualization of the work, but instead the reader in role of editor, designer, and consumer' role in forming their own conceptualization of the work that becomes a different vessel bearing similar cargo of the original text.}
	 {In the case of very simple digitization efforts, or plain-text productions of texts, texts are fitted to a standard that does not consider the human reader, but the machine reader, and texts are considered on the basis of their ability to conform, at least for the most part, to a standard that ensures their depositing into the computer's operations are coherent.}
	
"For the sake of argument, we might agree that most manuscripts and most books are material objects, which occupy space and have weight, such that no two books or manuscripts could occupy the same space at the same time." (13)
	 {Kirschenbaum would agree were Shillingsburg to extend this to the digital. Each call to open a text file, or download it even, triggers different, unique, physical operations within the computer, and even its own metadata unique to that instance of the file is created and cataloged.}

"Texts seem to be special because they seem to be iterable : that is they (or at least some aspects of them) can be reproduced, copied, transmitted, articulated in a variety of mediums and have at least the chance of being considered unchanged in the process--the chance of being considered to be still the same text." (14)
	 {Shillingsburg's tone suggests some sort of privilege or great fortune in a text's ability to do this. There's a sense of awe and wonder in describing what seems to him to be a miracle of textuality and its inherent characteristics.}

"Discriminating readers need to know which text they are using and what relation it bears to the history of alter-texts of the same work." (20)
	 {Do we say, then, that texts in a distant reading corpus or a dataset lack the oversight of a discriminating reader?} 
	 {When a text is a part of a dataset, does its ability to compare against its alter-texts diminish or exacerbate? 
	
"clarify without simplifying the textual condition." (24)
	 {does plain text simplify the textual condition? Shillingsburg might say yes when it comes to a single Gutenberg Projectz text, but in the aggregate, what becomes of the textual condition?} 
	
"Does the fact that an electronic text is searchable compensate for the fact that we cannot guarantee its continued existence ten or twenty years from now?" (29)
	 {Compare this to McKenzie's assertion that computers and their capacity to model texts are the affordance of computers. McKenzie might say yes to this statement, because he might either place the ability to search a text within its capacity to model information, he may see it as a complementary affordance to modeling. In either case, the searchability of the text in an  operational way is indeed an affordance of the text in an electronic medium. This is exacerbated once you consider the capacity to model and search many thousands of texts at once.} 

"Most users of printed texts and electronic texts act as if the text in hand is THE text, as if the context is sufficiently well known or does not matter, and as if any revisions that have been made in the text were proper and inevitable regardless of who made them."
	 {a nonconsumptive text file actually butts against this notion, standing as a n obvious exception to the rule that one would hold this text in high estemm as THE text to read- it's audience, purpose, context are clear as they effect the presentation in one of the most radical dimensions possible. CONTINUE TALKING ABOUT THIS!}
	
	
\section{bibliography-aggregation}	

 Both anthologizing and enumerative bibliography of collections is a script act performed in service of an ideal of aggregation. Datasets are much the same. The question then is what exactly is this "ideal of aggregation" besides a flashy term to coin and hope to be picked up and cited in future scholarship.
	
	develop some clear ideas of what this ideal of aggregation could include

\end{document}