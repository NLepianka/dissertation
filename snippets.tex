\documentclass[course, english]{Notes}
\usepackage{parskip}
\usepackage{graphicx}
\usepackage{csquotes}
\usepackage{outlines}

\setlength{\parindent}{0cm}
\raggedright
\setcounter{secnumdepth}{5}
\title{dissertation notes}
\author{nlepianka}
\date{10}{12}{2016}

\newcommand{\n}{\scalebox{2}{\textbf{\framebox{$\aleph$} } } }

\begin{document}


\section{basic bibliography}
\subsection{TERMS}
\begin{enumerate}
\item \textbf{Signature--}symbol found at the bottom of the page, usually a letter or number combination. Indicates in what order leaves were bound.
\item \textbf{Catchword--}Word found at the bottom of the page that ideally matches the first word of the following page. 
\item \textbf{Imprint--}Usually at the lower part of the title page; gives publication information- place, date, publisher, etc.
\item \textbf{Colophon--}Publication information found at the end of the book.
\item \textbf{Register--}A list of signature-letters of the various sheets placed (generally) at the end of a book
\item \textbf{Ornamental Initials--}Sometimes just 'initials'; large capital letters occurring at the beginning of a paragraph.
\item \textbf{Factotum Initial--}Sometimes just 'factotum'; an ornamental block having space for a capital letter of an ordinary fount type.
\item \textbf{Head-Ornament or Head-Piece--}Ornament designed for the top of a page.
\item \textbf{Tail-Ornament or Tail-Piece--}Ornament for the foot of the page or end of the matter.
\item \textbf{Head-Line or Running-Title or Running-Head--}A line of type at the top of a page above the body of the text. Can consist of the title of the book or the section title.
\item \textbf{Head, Tail, Outer, and Inner Margins--}Margins of the book. From generally smallest to largest: inner--head--outer--tail.  
\item \textbf{Bolts--}folds at the outer margin or fore-edge of a sheet folded in octavo or other smaller sizes. Used less frequently as a term according to McKerrow.
\item \textbf{Conjugate--}Two leaves which belong to one another-if traced into and out of the back of the book are found to form a single piece of paper. 
\item \textbf{Chain-lines--}Usually widely spaced and horizontal lines that can be seen faintly in paper. Indicates the spacing of chains in a wire bed on which the paper was made.
\item \textbf{Wire-lines--}Usually closely spaced and vertical lines that can be seen in paper. Indicates the presence of the wires in the wire bed on which the paper was made. Sometimes 'wire-lines' or 'wire-marks' is are use interchangeably or in reference to both kinds of lines. 
\item \textbf{Laid--}Paper with chain and wire lines visible is said to be laid.
\item \textbf{Wove--}Paper without chain and wires lines is said to be wove. 
\item \textbf{Watermark--}A design faintly seen in some leaves.
\item \textbf{"in fours", "in eights", etc.--}terminology to describe how many leaves are in a gathering. 
\item \textbf{type--}
\item \textbf{facsimile--}
\item \textbf{copy-text--}the early text of a work which an editor selected as the basis of their own.
\item \textbf{accidental--}readings that affect the formal presentation, punctuation, word-division, spelling, etc.
\item \textbf{substantive--}readings that affect the meaningful/semantic aspects of the text. Not to be confused with a substantive edition (McKerrow), which is an edition that is not a reprint of any other. 

\end{enumerate}

\subsection{Bowers-Bibliography and Textual Criticism-1964}

"Each fresh bibliographical breakthrough only discovers more areas for technical investigation before criticism can begin to make use of the newer findings; and thus the day for practical application of bibliographical hypotheses on any large scale is continually being put off." [4]

"This is a world the critic never made, a world to which a sound classical education and a First in Greats seem to have little pertinence. The technical concepts explored and the bibliographical terminology required appear to be as remote from literary criticism as the language and concepts of linguistics or of a natural science." [5]

\n --DH and cultural analytics seemingly have the same relationship as described here, though perhaps the insistence and maintenance of this relationship has been moved from the shoulders of the technically outcast (as with the bibliographers' insistence of their work as a science) to the critic (who may now insist that DH and distant reading are worlds apart from their work and practice).
	
\n --Ultimately, if Gailey is to be believed, the Cold War effect and the continued exercise of neoliberal/capitalistic thought in the university have driven both Bibliography and DH; Stanley Fish would agree probably.
	
\n --forward progression of time has meant that bibliography, textual scholarship, and the technical work of these fields has become complicated and ever the more fruitful to the study of canonical or traditional texts. The possibility of serviceable objects in these fields expand. See the work on Stowe's \emph{Uncle Tom's Cabin} that Wesley Raabe has discussed. Many of his primary sources are editions created well after Bowers' was giving these lectures.


\subsection{"evidence"}

"...developed a new investigative method of annual studies using a combination of enumerative, analytical, and historical bibliography to produce analyses of the book trade's total annual outputin 1668 and 1644."

cites McKenzie, "The London Book Trade in 1668" \emph{Words: Wai-te-ata Studies in Literature} 4(1974): 75-92.
	\begin{displayquote}
	--"innocent assumption that the locus of bibliography as a subject was \emph{the book}--any book--as a physical object (76). 
	
	--"great disparity between what analytical bibliography might infer, and what we can establish (81). 
	\end{displayquote}

\n --enumeratve bibliography as a form of evidence to pronounce and build upon book history and questions of scale- Suarez notes this as the 2nd phase of McKenzie's contributions to bibliography.

\textbf{Suarez}-"the 'half' of my five and a half phases of McKenzie's extending the evidence is his conception, promotion, and contribution to national histories of the book. A resolute proponent of intellectual synthesis, the McKenzie hated the tendency of bibliography to split things up, rather than to connect them--what he called in one article, 'our use of division as a function of analysis', 'the separation of sheep from goats, of chalk from cheese'. Advocating 'the unity of all bibliographical  enterprise', he envisioned that, in the History of the Book in America, or Britain, or Australia, or New Zealand, the whole would be worth more than the sum of the parts. It was McKenzie's hope that, when the fruit of so much scholarly labour was presented in the aggregate, new patterns and new understandings would emerge. He also recognized that, if read together, such national histories would enable comparative scholarship across traditional boundaries, a development McKenzie considered highly salutary for the future of the field." ["Extended Evidence", 40]
	\begin{displayquote}
	--McKenzie, "What's Past is Prologue" and "Stretching a Point" (found in \emph{Studies in Bibliography} 37.
	\end{displayquote}
	
\n --McKenzie seems to be the one to point to in the rise of distant reading, rather than Franco Moretti or Matt Jockers or anyone. A comparative analysis of Moretti and McKenzie would be important. Also, whoever Moretti got his ideas from- \emph{longue duree}, etc. 

\n --McKenzie understood enumerative bibliography to be more than a resource, and in fact, useful for studying history and assembling the patterns found within the aggregation of individual titles. McKenzie's realization of this, according to Suarez, points to his non-commital stance towards bibliography being about the book. In the same way, distant reading/cultural analytics, are also not exactly about the book- they reject the idea of the holistic textual object, as Amy would say. McKenzie did similarly.

\n --McKenzie seemingly thought of texts not as holistic objects at all, but parts of a further whole thing. He argues against division, yet recognizes the field has a pragmatic association with it. 



\subsubsection{Bowers-Bibliography and Textual Criticism}

\begin{itemize}
\item {"the only applicable function of bibliography in such an inquiry is to attempt to distinguish which details of the printed text are compositorial and which are not, the purpose being to analyse the manner in which the printer's copy has been turned in to print." [8]}
\item {"the bibliographical interpretation, by itself, can seldom extend to the cause for the anomalies in the underlying manuscript, although it may provide the foundation for such a critical explanation" [9]}
\begin{itemize}
\item{\n Bowers draws a line between a critical argument about authorship (e.g., a new critical or new historicist inquiry) that is founded but not defended on the basis of bibliographical evidence.}
\item {"It is a serious fallacy, I take it, to confuse the nature of the interpretation with the nature of the evidence." [9]}
\item {\n Bowers' approach would seem to digress from McKenzie's then, as the sociological view of bibliographical, especially in relation to the rise of high theory, political and identity consciousness in literary studies, and so on, would indicate that the sociological view sees more possibility in the relationship between interpretation and evidence, if it does not already conflate the two from the outset when discerning that the manner of compiling and describing evidence itself involves interpretation.}
\end{itemize}
\item{"one of the soundest doctrines of textual criticism requires an editor to begin with the nature of the whole before turning to any individual part. In other words, the whole form must be determined before one comes to test any single word." [12]}

\end{itemize}

\subsubsection{McKerrow-An Introduction to Bibliography for Literary Students}
\begin{itemize}
\item{\n First page of the first part, McKerrow already admits that between the author and publication, numerous other people have intervened that could have an effect on transmission. The important thing, I suppose, then, is that the book was transmitted in the first place, though McKerrow still insist on the primacy of the author.}
\item{"We are all now for 'bibliographical methods', keenly on the watch for every least indication of disturbance in the accurate transmission of a text, sorting out by many subtle and ingenious methods the first, second, or third stage of the composition, the original draft, the first completed form, the revision for this, that, and the other purpose, and so on." [2]}
\item{"The virtue of bibliography as we used to count it was its definiteness, that it gave little scope for differences of opinion, that two persons of reasonable intelligence following the same line of bibliographical argument would inevitably arrive at the same conclusion, and that it therefore offered a very pleasant relief from critical investigations of the more 'literary' kind." [2]}
\begin{itemize}
\item \n McKenzie really drove a train through that line of thinking.
\item \n John Bryant, McGann, and Greetham as well would not entertain that kind of nonsense as a descriptor of bibliography, nor the purpose or even the summary of its methodology
\item \n Gailey would label it a Cold War symptomatic description.
\item McKerrow doubles down on this statement in the conclusion, calling discoveries "not mere matters of opinion, provable things that no amount of after-investigation can shake, that lends such a fascination to bibliographical research." [5]
\end{itemize}
\item \begin{outline} \1 \n McKerrow defends the idea that bibliography's relation to critical literary analysis is in bibliography's ability to maintain the honesty and certainty of the text as it is faced by the literary scholar. In this idea, it is reasonable to look at texts that have been manipulated for a certain purpose (i.e. edited with the idea of teaching, close reading, or scholarly activity otherwise) as the textual scholar and bibliographer as bureaucrat in the chain of literary command. 
\2 Non-consumptive text files, or databases of texts function in a similar manner- they are crafted for a purpose- and in this way the bibliographer can be argued to  have a stake in the production and evaluation of the composition of these texts in order to assure accuracy and certainty in the files. 
\3 HOWEVER - I had a oppositional idea and forgot in between figuring out a formatting issue here.
\3 McKerrow relegates these concerns to textual criticism- what he means by the difference, I do not yet know. 
\3 also a strange note- McKerrow's telegram example on page 3 posits as mutually exclusive the economic and bibliographical as properties of a word count. This would also be lambasted by McKenzie, McGann, et al. 
\end{outline}
\item \begin{outline} \1 "...so that he sees this [a book] not only from the point of view of the reader interested in it as literature, but also from the points of view of those who composed, corrected, printed, folded, and bound it..."[4]
\2 OCR'd files make this even more prevalent, and actually easier of an exercise due to their distinct ability to screw up the standardized reading process. 
\3 moreso in nonconsumptive files where the bibliographical will affect (infect?) the literary.
\end{outline}
\end{itemize}

\subsubsection{Tanselle}
\textbf{Rationale of Textual Criticism--}"And when the details are of a kind that can potentially be corroborated or denied by reference to the actual books, the archives provide only secondary evidence: the archives say something \textit{about} the books, but if the books survive they can speak for themselves." (57)

\subsection{Tanselle-Rationale of Textual Criticism (1989)}
\begin{outline}
\1 "If, for example, we think not of "works" (a term that implies previously created entries) but only of sequences of words that have come our way, links in the endless chain of language, the question of authenticity is meaningless (a point I shall return to later). (13)
	\2 \n "works" vs. data or bag of words.
\1 "one must be able to distinguish the work itself from attempts to reproduce it. A work, at each point in its life, is an ineluctable entity, which one can admire or deplore but cannot alter without becoming a collaborator with its creator (or creators). (13-14)
	\2 Tanselle further discusses reproductions after this.
\1 Pg. 16- Tanselle decries literary study's mostly uncritical attitude towards the text as a historical witness. Claims naivety in trusting the text in front of one. 
\1 "Because a literary work can be transmitted only indirectly, by processes that may alter it, no responsible description, interpretation, or evaluation of a literary work as a product of a past moment can avoid considering the relative reliability of the available texts and the nature of the connections among them. (18)
	\2 Tanselle uses the term "innovation" to describe one way a text may change- this could reflect well on a stance about non-consumptive data, and 
\1 "Various theories of literature have arisen from the premise that the \\textit{meaning}of verbal statements is indeterminate; but such theories remain superficial unless they confront the indeterminacy of the \textit{texts} of those statements." (24)
	\2 \n i.e. my response to Laura during orals. 
\1 "a piece of paper with a text of poem written on it does not constitute a work of literature, and therefore any alterations one makes in the manuscript do not automatically alter the work. If one cleans a dirty spot on a manuscript and reveals a word not legible before, the word is unquestionably a part of the text of the document, but it is not necessarily a part of the literary work."
	\2  \n what McGann and McKenzie (and I) might say in response: what he may mean is that the historical and popular knowledge of the work excludes the "new" word discovered in his example. Thus, it is not part of the work, because the society in which the text exists does not recognize it (yet). Thus may begin a process of (re)- inscribing that word into cultural memory. 
\1 \textbf{Pg. 43- there is an "inherent uncertainty" to all works and their documents that is derived from the matters of transmission, not only in print, but as conceived of in the author's head.}
	\2 Tanselle, is always vigilant about this and is probably his guiding principle. -see "Reproductions and Scholarship"
	\2 This is why he privileges the matter of authority.
\1 "there can be no identical copies of printed books, just as there can be no identical copies of manuscripts or of any other object;" (51)
	\2 See Kirschenbaum's .txtual condition. The catechism: "Access is thus duplication, duplication is preservation, and preservation is creation — and recreation." (16)
		\3 "In the terms I put forth in Mechanisms, each access engenders a new logical entity that is forensically individuated at the level of its physical representation on some storage medium." (16)
		\3 ON this point, Kirschenbaum and Tanselle agree then.  
\1 "A reproduction, whether produced photographically, xerographically, or in some other way, can no more be a substitute for the thing reproduced than another printed copy from the same press run can be." (54)
	\2 Tanselle is responding the critics he names in "Reproduction and Scholarship."
	\2 Also, assumption here that people already know that print runs cannot be compared, which I suppose is part of the previous discussion of this chapter.
\1 pg. 55 two axiomatic points: 
	\2 first, every text has been affected in one way or another by the physical means of its transmission.
	\2 second, that every copy of a text is a separate piece of documentary evidence.
		\2 \n This begins me questioning the idea of OCR, and how the process itself needs to be understood by scholars- not every OCR run will be the same, not every system for OCR will produce the same results- this is the purpose of the eMOP project.
			\3 It probably needs to be demonstrated that different quality OCR, of even the same edition of a text will reproduce different versions of the text- regardless of copy-text.
\1 pg. 56 Tanselle complains that some scholars who are trying to get to a broad picture ignore the minute. In this sense, he probably could be referencing big data/quantitative efforts.
\1 "And when the details are of a kind that can potentially be corroborated or denied by reference to the actual books, the archives provide only secondary evidence: the archives say something \textit{about} the books, but if the books survive they can speak for themselves." (57)
	\2 He is talking about metadata
	\2 The physical book as evidence is primary or metadata or other records or witnesses that may have some sort of testimony to give about the book. 
\1 Tanselle defends editions while at the same time pointing out that one can alter the text, because the edition will never be a substitute for the original anyways. (58-9)
	\2 editions (and other types of copies) are useful so long as their limitations are understood, i.e. what was lost in transmission.
	\2 editorial expertise can help casual or unfamiliar/untrained readers.
	\2 \n Tanselle, here seems to open up some logic to the idea that the non-consumptive file is an edition, which has benefited from some level of expertise. Ted Underwood et al. have certainly not be unskilled in their preparation of documents. 
\1 "The messiness of such documents is an essential part of their nature, and a documentary edition is not performing its function if it does not report as much of that untidiness as photographic reproduction, typographic transcription, and supplementary discussion can convey." (62)
 \end{outline}
 
\subsection{Greg- Rationale of the Copy-Text}
\begin{outline}
\1  McKerrow coins the term copy-text to mean "the earlier text of the work an editor selects as the basis of their edition." Greg's description- pg. 19
	\2 McKerrow coins it in his \textit{Works of Thomas Nashe}
\1 Greg locates the idea of choosing a copy-text, "treating a text as possessing over-riding authority" as rooted in Classical and Biblical scholarship. (19)
\1 Lachmann- genealogical classfication as a key method of textual criticism
\1 Greg's point: "authority is never absolute, but only relative."
\1 Greg seems the first to coin accidentals and substantives- pg. 21
	\2 \n OCR as a process of edition-making deals primarily with accidentals on the surface- it is dedicated mostly to formatting and presentation. Of course, its rate of error will naturally make substantive changes to the text that come to light in the movement from OCR copy to another format. The nature of the new format itself will also introduce significant substantive changes (i.e. OCR text to non-consumptive data file) will radically change the expressive capacity.
\1 Greg also seems to heavily lean on the subject of expertise (Tanselle as well)- \textbf{so then can what counts as expertise in a given moment be decided and expanded in the same way the role of the editor needs be by Bryant?}
\1 Ultimately, authorial intent is the MO of Greg, and expertise, perhaps, is how one is able to determine it. 
	\2 But intent is dead, so is expertise, or is expertise required as well to navigate the new paths opened by the author's death? 
 
\end{outline}

\section{on ocr}
\n --What the articles of Derg-Kirkpatrick, Klein, and Durrett tell us is that the questions and scholarship of bibliography, textual scholarship, and book history, even when not called such, are important and relevant to those who are approaching text artifacts from different fields, and that relevance is something our fields can embrace, discuss, and contribute to settings beyond the Special Collections. 

\n --You do in fact get a double layer of metadata and context for a HathiTrust facsimile or OCR plain text document, as the circumstances for who scanned the text, at what time of day they did, the computer specs and settings, but the new edition also accretes some of its copy-text's bibliographic properties, and not just its linguistic properties. The new edition will present and demonstrate a historical texts properties at the same existing as a new text with its own properties. This makes the bibliography of such objects possess a dualistic nature of attempting to compile all that we know of the source text and the new text that was derived from it.

\n one thing that does stand out about the OCR process, if we are to defend it as a form of transmission and edition-making, is that the role of the editor, in whatever loose capacity one has decided it to be, is not self-consciously assumed with the same critical lense as textual scholars may presume, and is not self-consciously recognized. In fact, it's possible it may ignored altogether.
	-- John Bryant, on editor, defines it as "anyone--friends, family, professional and scholarly editors, publishers, even adapters--who in the course of the history of a given work lays hands upon that text to shape it in new ways." (6, \textit{The Fluid Text}) 

\subsection{tanselle- reproductions and scholarship}
\begin{outline}
\1 \textbf{reproduction--}by this, Tanselle means the product of any chemical or electrostatic process that aims to represent with exactness (though perhaps on an enlarged or diminished scale) not only the text of a given document but also the details of its presentation, insofar as they can be duplicated on a different surface." (26)
\1 \textsc{Tanselle's position is on photocopying, xerographic, and microfilm reproduction- NOT OCR, but is relevant. }
\1 \n Tanselle is perplexed by the perceived lack of discussion of the limits of reproduction (as he means it, see above), that is, admitting that errors can in fact occur, or things can be missed. These are always possibilities, and we must keep our feet on the ground in respect to recognizing an emerging technology.
\1 Quotes Greg: "no process but in some measure obscures what it reproduces." (321, "Type Facsimiles and Others," \emph{The Library} 4.6
\1 On Omission, Laurence A. Cummings: "photographers omit material through oversight or the assumption that it could not be significant enough for the customer to wish to pay for ("Pitfalls of Photocopy Research", \emph{Bulletin of the New York Public Library} 65 (1961) 97-101)
\2 \n This point is one of the few cases where economics comes into play in Tanselle's discussion.
\1 On page 31, Tanselle expresses fear over the fact that photocopy images can also be changed- manipulated, not just through omission but through airbrushing. This is ultimately also gonna manifest as a fear of Photoshop, of course. 
\2 Reams Gabler's \emph{Ulysses} on this, since it used as its copy-text photo facsimiles.
\1 \n My takeaway of \textsc{section 1}: In relation to the Hathitrust, their page images are one layer of transmission, one layer where errors could be introduced. The next layer, then, is the OCR (where we know errors will be introduced--we do not entertain a fantasy that they will not, no matter the accuracy rating). There is a question, however, of how many layers exist between the OCR and non-consumptive files exist.
\1 Tanselle: Reproductions are always unsatisfactory, even if they are perfect, because:
	\2 there is no reason to trust them in the first place to let them serve as substitutes. The "essential fact" is that "every reproduction is a new document with characteristics of its own and no artifact can be a substitute for another artifact." (33-34)
		\3 For Tanselle, this rationale seems to be the \emph{prima facie} of his stance.
\1 Argues that facsimile editions, and photocopy editions should come with a record of how their representation of a text vary from other editions- extends to both, I think, how the photocopy differs from its copytext but also how the copy-text differs from other copies of the same edition. 
	\2 \n Could this extend to the OCR- document how the transmission works, how the objects morph as they are transmitted, but also, source the copy-text with a more robust description. 
		\3 It may also beneficial to record, either through crowdsourcing or automatically, the number of OCR errors that occur in the text, and their relative significance. 
	\2 Tanselle actually endorses the collation of multiple photocopies of multiple editions to reproduce a more ideal text. \textbf{he would}
		\3 \n this is conceivably easier for OCR, non-consumptive, and digital works, even with pdf scans, if resources are available. \emph{of course Tanselle will not mention cost}
\1 Pg. 44- mention of the books being guillotined. Violence done to the text, wherein the spine is removed during the microfilming process. 
	\2 \n Is OCR any better for the life of the book, or is the creature more Frankenstein in its pdf, OCR, or non-consumptive state? If the book is executed, is the pdf its zombie, the ocr the ghost, and the non-consumptive the writhing sinews after its been blasted by a digital Tommy Lee Jones?
		\3 "spines are still being split and pages pulped as books disappear into \textsf{information}." (44) Emphasis mine.
		\3 \textsc{the bibliographer is a necromancer}
		
\1 \n On Authorial Intent and things such as an OCR, intent takes on a very odd situation, mainly because it becomes a ontological, epistemological, and teleological impossibility.
	\2 teleological because there was not a way of preparing a text to be OCR'd, and no standards in place that would facilitate it at the time of a text's composition.
	\2 ontological because computers did not exist.
	\2 epistemological because you'd have to prove the author was of the frame of mind to anticipate such a possiblity as OCR, pdfs, non-consumptive data, and text-mining. 
\end{outline}

\subsection{trettien}
\begin{outline}
\1 remediation between multiple formats- letterpress printed paper to celluloid film to metadata-encrusted digital scan to database entry and back to paper. For Trettien, a text created this way is a Frankenstein like creation with a metadata skeleton. 

\1 "Yet, as Parikka and Sampson point out in relation to junk mail and computer viruses, artifacts that seem to skirt the peripheries of the average user's experience in fact occupy a central position within the digital marketplace, exposing the processes of mediation and communication circuits upon which network capitalism depends. " (2)
	\2 Non-consumptive data is totally a capitalistic product given all the legal/copyright issues that are buttressing its production. 
\1 Mentions Tanselle's overview of how reproductions affect scholarship.
	\2 returning to Tanselle, his "inherent uncertainty" of the text is ignored by the legal classification of what constitutes IP/copyright violation. The physical reality, the ontological and epistemological reality,of the text's reproduction are still viewed as threatening to the legal status of copyright. 
		\3 It's only in the radical deformation of non-consumptive data that it becomes non-threatening. 

\1 "It is far from radical to suggest that humanities scholarship as we know it today emerged alongside and partly depends upon this machinery of scholarly editions, and that this apparatus is itself tied to print and other technologies of mass reproduction" (16)
	\2 OCR does the same- emop project, wright american fiction both show canonical devotion, even when the canon is zoomed to the level of canons of subjects.

\1 "Yet tracing the fault line that separates canonical (and therefore acceptable) Milton from a (monstrous) POD reprint gives us a better perspective from which to understand our own historically-constructed assumptions about plain text and facsimile image, printed book and electronic file." (17)
	\2 Here, Trettien is addressing what is ultimately an institutional thing: what gets money to have nice editions made or bought; what ends up as detritus in need of POD versions? Why can't everything have a nice edition it?

\1 "Thus even as eighteenth- and nineteenth-century scholarly editions aimed to modernize Milton, the reprint industry of Arber's Areopagitica churned out "Everyman" books made to sophisticate the shelves of middle-class English families [Howsam 2009, 20]. The former practice analyzed the artifacts of early printings in order to produce a clean, abstracted text, portable across multiple platforms; the latter emphasized the dissemination of texts, of ideas, while in fact focusing on the production of material objects. Put another way, scholarly editions began institutionalizing literature, recruiting texts into, and normalizing them according to, the emerging standards of academic research. Importantly, while electronic forms of communication are now forcing scholars to rethink the assumptions undergirding these standards, the discourse of digital humanities still largely operates within the terms set out in these Enlightenment-era projects, privileging a model of texts as information, as data, and the use of digital technologies as tools to help clarify or interpret them. By contrast, popularizing reprints brought "the classics" closer to the average reader precisely by rendering them more materially other than the nineteenth-century texts with which they circulated, an otherness made explicit when Arber's Milton sits alongside, for instance, Blackburne's. Although both scholarly editions and reprints fed (and continue to feed) on a rhetoric of accessibility, the latter's highly stylized design exposes the former's pretenses of presenting dematerialized (or "plain") text. It is precisely this mediated deformance of history — here, I am invoking McGann and Lisa Samuels -— which illumines the historically constructed nature of textuality, as well as the technologically mediated strangeness of all scholarly editions." 

\1 "breathe new life into books"
	\2 not the same life it had before. 
		\3 critical editing does the same

\1 a POD book with OCR errors has its own presentness associated with it that declares itself as a 21st century object- a product of its time. 

\1 "And whither formatting? Nothing — not even a hard return — separates the various documents contained therein. By stripping the page of standard visual cues, this POD Areopagitica denies (or simply ignores) the book's role as a map orienting the reader along the multidimensional curvature of its text and, in doing so, denatures its own apparent clarity; for "plain text" without the formal materiality of formatting is far from plain." (24)
	\2 Of course, a non-consumptive file may possess that map, since it is delineated to facilitate machine readability 

\end{outline}

\subsection{crane et al}
 "Google is providing immense scale but the scholarly significance is not so great as it might be: there is at present no way to understand what subset of the world’s knowledge that seven million volumes represents. Even if there were, scholars have no way of understanding in more than the most general way how the services that extract information from that collection work — what is missed? What biases are embedded in the system? Scholarship depends upon transparency, and we must be careful that we do not, in pursuing our immediate research projects, compromise our fundamental commitment to transparency." [1]


	-- {Does this beg the question that non-consumptive research then is not transparent or not as transparent as it could be? If the information is obfuscated in any way, is it unsuitable as an object for academic study?}
	
	-- {If the material book persists or announces its existence in non-consumptive files- are we dealing with an additional layer of transparency that otherwise might be elided by those disinclined towards textual scholarship and bibliography?}
	
 "if we could understand how to build a comprehensive collection of classical scholarship from the beginning of print culture to the present, we would know how to work with centuries of print publications on every aspect of human society and in every discipline and from every corner of Europe and North America." [6]
	
	--  {am not entirely sure about this point. This is one viable option for investigating a question and is valuable as a point on a timeline, but casting a wide net in order to cover a wide range of time periods of print culture might cause more generalizations to develop at the cost of specifics that pertain to individual developments in printing, printers and publishers, or historical moments.}

"we cannot thus rely on upon a centralized editorial structure to guarantee for us the consistency of what we find." [7]
			-- {this acknowledges my point above}
			
{so called "fourth generation" collections "seamlessly integrate image-books, accurate transcriptions, and machine actionable knowledge in various formats." Their use of the word accurate, of course, frames the material as the authority as what is "accurate"} [9-10]

 --{what Crane et al seem to imply in their discussion of the apographeme is that there is a primacy of the material vessel that transmit a text, which we inherit from print culture, but also through various physical media, and that the progress of a project such as there continues to tie the data they are working with to its material vessels, rather than abstracting the data away from it.}

 --{Since Crane et al also value transparency as a fundamental commitment of scholarship, the material vessel then, is considered by this group as an important part of that transparency. This puts them, in fact, very much in the role of bibliographers and textual scholars who value more rigidly that same commitment to transparency and acknowledgement and codependency of text upon its medium.}

"This apographeme constitutes a superset of the capabilities and data that we inherit from print culture but it is a qualitatively different intellectual space. In the mature apographeme, every canonical text is a multitext, with dynamic editions linked to visual representations of the manuscripts, inscriptions, papyri and other sources. In the mature apographreme, each source is linked to the background data that we need to understand it — a transcription, information about the particular type of Greek or Latin script and its abbreviations, about the monastery, print shop or Egyptian village that produced it, etc." [10]

--- {A large portion of their "Services for the humanities in very large collections" also note the primacy of the material: pagination, a coordinate system for page imagines, transcription data, basic areas of the page, structures, etc." 

"Fourth-generation collections allow us to design corpora that go far beyond limitations that we internalized in print culture." [23]
	
	-- {So, in some way, we desire to get beyond the the book but are not really able to- getting beyond the book really just means acknowledging the plurality of data in books and aggregating the various forms text may occur to further emphasize not the text, but the multiple ways a text can be carried.}
	
	--{On the other hand, the proliferation of literary data at a non-consumptive level could be seen as a way that distorts and tears the text away from the book in order to accomplish the intended goals of the 4th generation collections by enabling ease of access and operations to the collection as a whole.}

\subsection{strange et al}

"much of the variability is associated with the historical and contemporary resources of publishing houses, meaning that major metropolitan papers typically sit at one end of the legibility spectrum and smaller, regional papers sit at the other." [16]
	 
	 --{not source this claim, but previous citation is Arlitsch, Kenning and John Herbert. "Microfilm, paper, and OCR: issues in newspaper digitization." Microform and Imaging Review, 33.2: 58-67}

	  --{is called the Walworth Murder Project}
	
--remove line breaks as a part of their OCR cleanup, meaning they do tear some of the bibliographic properties from the linguistic text}
 --a{lso fucking used voice recognition software as part of their transcription process}

"As well as typing from original scans, we transcribed texts using a voice recognition program (Dragon Naturally Speaking 12), another option for the correction and input process. Typing was predominantly used, since it tends to be quicker than transcriptions of dictation for corrections. For inputting longer sections from scratch, dictation was slightly faster and more convenient to use. However, it tends to fail "silently", in that it substitutes unrecognized words with other words, which a spell-checker cannot detect. Typographical errors, on the other hand, are more likely to form non-words that spell checkers can identify. Dictation is also more likely to fail on names and uncommon words and proper nouns, precisely those words which the study is most interested in identifying." [30]

 "Overall, the cleaning of the data was not essential to achieving results of interest on the two-phase comparison task, since many significant words could still be identified. Still, there were substantial differences between the results of the clean and original datasets, as significant words were missed and "false positives" were generated." [45]

\section{the editor}
\begin{outline}
\1 \textbf{Greg-Rationale of Copy-Text--}"...an editor who declines or is unable to exercise his judgement and falls back on some arbitrary canon, such as authority of the copy-text, is in fact abdicating his editorial function" (28)
	\2 \textbf{question--}can one abdicate an editorial function one has not recognized or assumed?
	\2 the editorial function, or the title of editor, is not always consciously assumed- In the case of Bryant (\textit{fluid text}) the role of editor is very wide (I agree with his definition--I'll be applying it to creators of datasets). In these cases, the people who lay hands on the manuscript do not always do so as people knowing they are an editor, but perhaps they know they are editing. 
		\3 To expound on this, see works by Hill, Eggert, Spevack and Werner. 
		

\end{outline}


\section{big data}

\subsection{Underwood}
 "How far do you have to back up before you start seeing patterns that were invisible at your ordinary scale of reading? That’s how big your collection needs to be." 
("A more intimate scale of distant reading.")
	
	
	
	
	
	
\section{textual scholarship}
Terms by scholar
McGann:
	Deformance
	Mutation
	Development
	
Greetham:
	Contamination
	
Greg-Bowers-Tanselle?
	Corruption

Shillingsburg: 
	Reincarnate

Tanselle: 
	physical is tractable (Rationale of Textual Criticism)

--{The use of organic terms like mutation to describe textual change would help in positioning against the idea that automatically generated texts are non the concern of bibliography and textual scholarship. Rather, the mutation of texts that is seen in automated processes is just a stage in the text's life.}
		 {use of "life" is on purpose here, to further emphasize the text's anti-complicity to its mechanical nature}
		\\-- {can cyborg theory be useful in understanding the text? Hayles has probably considered this, honestly.}
		\\-- {evolutionary theory as a part of textual history can be important.}
		
-- On the topic of the automated vs. manual debate, see Ryan Cordell and his claim that the debate tends to overestimate human autonomy in traditional book publication and underestimate the human component of digitization. 
\\--from a bibliographical standpoint, critical bibliography is adept at seeing through such a simplified debate, as it has traditionally engaged in the process of emphasizing and addressing the intersection of human and mechanical in printing. Its occupation with such things, may have given way to faulty logic that now closely identifies material publishing as humanistic, but ignores the "automated" or at least stringently mechanical aspect of printing. The opposite is then seen to occur in critiques of big data, OCR, and other "automated" processes. 
		
		

\subsection{Shillingsburg-GtG}
Nuanced vocabulary:
book vs. manuscript
scribal vs. holograph
authentic vs. forged vs. reproductions.

\emph{script act theory}: 

What happens if a collection of manuscripts are bound into a book? Or a blank book gets written in? (Chapter 3 discusses this)
	 {Important to this will be the concept of text in its 3 common definitions: 1) physical object 2) lexical symbols 3) conceptualization (sometimes called the work).}
	 {I think right now, I will decide on championing the first 2 definitions primarily: since the idea of the physical is important to make a case, but the idea of the lexical symbols is necessary from a cultural perspective to indicate the presence or absence of a physical object- the lexical symbols point to their physicality and the ways in the which the book artifact existed.}
	 {the idea of the conceptualization as a definition of text seems weak to me from a bibliographic perspective, because the page signatures, among other aspects of a book's production, especially books produced after the literal death of the author bear no connection to the conceptualization of the work, but instead the reader in role of editor, designer, and consumer' role in forming their own conceptualization of the work that becomes a different vessel bearing similar cargo of the original text.}
	 {In the case of very simple digitization efforts, or plain-text productions of texts, texts are fitted to a standard that does not consider the human reader, but the machine reader, and texts are considered on the basis of their ability to conform, at least for the most part, to a standard that ensures their depositing into the computer's operations are coherent.}
	
"For the sake of argument, we might agree that most manuscripts and most books are material objects, which occupy space and have weight, such that no two books or manuscripts could occupy the same space at the same time." (13)
	 {Kirschenbaum would agree were Shillingsburg to extend this to the digital. Each call to open a text file, or download it even, triggers different, unique, physical operations within the computer, and even its own metadata unique to that instance of the file is created and cataloged.}
	 {Tanselle would agree as well}

"Texts seem to be special because they seem to be iterable : that is they (or at least some aspects of them) can be reproduced, copied, transmitted, articulated in a variety of mediums and have at least the chance of being considered unchanged in the process--the chance of being considered to be still the same text." (14)
	 {Shillingsburg's tone suggests some sort of privilege or great fortune in a text's ability to do this. There's a sense of awe and wonder in describing what seems to him to be a miracle of textuality and its inherent characteristics.}

"Discriminating readers need to know which text they are using and what relation it bears to the history of alter-texts of the same work." (20)
	 {Do we say, then, that texts in a distant reading corpus or a dataset lack the oversight of a discriminating reader?} 
	 {When a text is a part of a dataset, does its ability to compare against its alter-texts diminish or exacerbate? 
	
"clarify without simplifying the textual condition." (24)
	 {does plain text simplify the textual condition? Shillingsburg might say yes when it comes to a single Gutenberg Projectz text, but in the aggregate, what becomes of the textual condition?} 
	
"Does the fact that an electronic text is searchable compensate for the fact that we cannot guarantee its continued existence ten or twenty years from now?" (29)
	 {Compare this to McKenzie's assertion that computers and their capacity to model texts are the affordance of computers. McKenzie might say yes to this statement, because he might either place the ability to search a text within its capacity to model information, he may see it as a complementary affordance to modeling. In either case, the searchability of the text in an  operational way is indeed an affordance of the text in an electronic medium. This is exacerbated once you consider the capacity to model and search many thousands of texts at once.} 

"Most users of printed texts and electronic texts act as if the text in hand is THE text, as if the context is sufficiently well known or does not matter, and as if any revisions that have been made in the text were proper and inevitable regardless of who made them."
	 {a nonconsumptive text file actually butts against this notion, standing as a n obvious exception to the rule that one would hold this text in high estemm as THE text to read- it's audience, purpose, context are clear as they effect the presentation in one of the most radical dimensions possible. CONTINUE TALKING ABOUT THIS!}
	
\subsection{Meredith McGill, "Echocriticism"}
\begin{outline}
\1 McGill claims texts are more fixed than ever because their reprinting is wrapped in digital distribution and these things (though she does not use the word specifically) rely on the rationale of the copy-text. Digital distribution is indebted to copy-texts, copy-right, etc. 
	\2 \n How might she discuss the non-consumptive data? On the hand she may defend her fixity statement by saying it reifies the 1838 edition as the \textit{prima facie} copy-text.
	\2 \n OR, she may buckle under the radical deformation of the text. 
		\3 The radical deformation cannot be ignored, I suppose, but it is still as much reliant on the rationale of the copy-text for McGill's argument and thus she may say the non-consumptive may not differ as much as it may seem from the static text. 
\1 McGill is also interested in what is stable about the transmission of texts, and not simply variance. Stability is actually quite a different frame of how texts are imagined. 
	\2 "...reprinting also emphasizes what remains the same despite the variety of forms and formats in which texts appear."
		\3 \n perhaps my interest in what remains the same is driven by the fact that in the most radical cases fo deformation, the tracking/recording of variance becomes an exercise in approaching the sublime- it's far too much to count and perhaps too obvious to warrant an intellectual critique. 

\1 \n McGill has caused to ask about the fact that first edition of texts were chosen for the HathiTrust dataset. Rather than something like CEAA editions of texts, that would invoke modern copyright and be more readily achievable via non-consumptive data. Why then are first editions given the privilege give traditional textual scholarship's favoritism towards more authoritative versions? Does Ted not care? Is that not a standard that can be upheld? Is there an editorial preference for the texts as they existed at the time versus how they are edited/idealized? 
	\2 First editions may be better given issues that we may have with something like \textit{Uncle Tom's Cabin}, which according to Raabe is proliferating issues. 
	\2 McGill also envisions volumes (e.g. datasets) that have organizing principles that are outside traditional standards, yet are easily attained via metadata. 

\end{outline}
	
\section{bibliography-aggregation}	

 Both anthologizing and enumerative bibliography of collections is a script act performed in service of an ideal of aggregation. Datasets are much the same. The question then is what exactly is this "ideal of aggregation" besides a flashy term to coin and hope to be picked up and cited in future scholarship.
	
	develop some clear ideas of what this ideal of aggregation could include

\end{document}
