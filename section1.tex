%%%%%%%%%%%%%%%%%%%%%%%%%%%%%%%%%%%%%%%%%%%%%%%%%%%
%
%  New template code for TAMU Theses and Dissertations starting Fall 2016.  
%
%
%  Author: Sean Zachary Roberson
%  Version 3.17.09
%  Last Updated: 9/21/2017
%
%%%%%%%%%%%%%%%%%%%%%%%%%%%%%%%%%%%%%%%%%%%%%%%%%%%

%%%%%%%%%%%%%%%%%%%%%%%%%%%%%%%%%%%%%%%%%%%%%%%%%%%%%%%%%%%%%%%%%%%%%%
%%                           SECTION I
%%%%%%%%%%%%%%%%%%%%%%%%%%%%%%%%%%%%%%%%%%%%%%%%%%%%%%%%%%%%%%%%%%%%%


\pagestyle{plain} % No headers, just page numbers
\pagenumbering{arabic} % Arabic numerals
\setcounter{page}{1}

\chapter{\uppercase {Introduction}}

\section{Beginnings}

This dissertation seeks to place the idea of data and datasets into the realm of editorial, textual, and bibliographic theory. As "big data" (or, "distant reading," "cultural analytics," and "algorithmic criticism" to suggest a few alternative terms) has become a more prominent concept in humanities (and specifically, digital humanities), we have seen increased awareness of the structures that hold literary and cultural data. Scholars have brought attention to the databases, websites, research portals, etc. that all facilitate access for scholars to materials that may otherwise be impossible to see (i.e. manuscripts, first editions, or unknown works) and the logic by which they operate (or, at least, how scholars \textit{think} they operate). These digital repositories have enabled large-scale, quantitative research of literary and historical materials because they enable access to these materials at a scale that was, if not impossible, impractical to humans. Methods that seek to read a large number of texts can do so at a scale of tens or hundreds of thousands in less time than it would take a scholar to read one (if it were, say, a novel). 

My interest, however, is not in the methods themselves, nor the processes that scholars use to read a significant number of materials. Instead, my focus is on those tens of thousands of texts. For a dataset of approximately 10,000 texts of American literature, it would be expected that such works as Hawthorne's \textit{Scarlet Letter}, Poe's \textit{Narrative of Arthur Gordon Pym}, or Cooper's \textit{Last of the Mohicans} would be present. But what populates the rest of that hypothetical list, the other 9,997 works? How did they get there, and how do we find such works that do not have the benefit of canonicity, prestige, and decades of literary scholarship that keep them foregrounded in academic and popular culture? Wright's \textit{American Fiction}, here, is a case study, but not unique in terms of its role, composition, and presentation as a bibliography. In the process of composing \textit{American Fiction}, Wright needed to answer questions inherent to bibliographical work in the early twentieth-century, where bibliography was understood as an empirical process that required explicit principles and standards in order to execute. Those standards, I will argue, are what present the interpretative positions Wright held in composing \textit{American Fiction} and inform the consequences of Wright's work since its publication. 

Answering questions about how a corpus such as Wright's came to be requires investigating the history of a dataset, and the culture from which a dataset emerged, and what contingencies enabled the inclusion of a text in a collection. There is a need for this sort of work, and it is desired by data-focused scholars. As distant reading and work with textual materials at a large scale has solidified themselves as viable methods of humanistic research, scholars have become increasingly aware that the data employed in such research itself has a history. In a 2016 provocation, Sarah Allison advocates for what she calls a "turn to the \textit{byproducts} of cultural analytics--to more project-specific tools, documentation, and discoveries." The "byproducts" Allison refers to are the data used in the course of research, data that are pulled from digital databases, print bibliographies, or others scholars' personal work. In the course of her provocation in the \textit{Journal of Cultural Analytics}, she likens the idea of finding someone else's data as "like a new manuscript: an unexplored object that deserves attention in its own right." This metaphor, as Allison later explains via a conversation with Andrew Goldstone, also demands the asking of certain inherently bibliographical questions: "is this an authoritative source? what process created it? what is the chain of transmission by which it reaches us?" and so on. In short, what Allison is in fact advocating for in the study of data, is textual scholarship and bibliography; to see the knowledge of these fields applied to this conception of data as a text.\autocite{sarah_allison_other_2016}

Tracing the history of a collection of data, in regards to literary materials, inevitably leads to a time before the digital turn. A digital repository of literary materials that represent printed text point to sources and a history outside of the repository itself. A digital copy of a nineteenth-century text, of course, has a physical copy with its own history, from its conception and production to its circulation that places it in the hands of those who would digitize it. This circulation may involve preservation in libraries and archives, auctioning at a rare book sale to a private collector, or any number of circumstances that have affected the text's movements and life up the point of its digitization. The life of a dataset is similar, and a dataset of textual materials may be derived from a single source that has history of composition, printing, and circulation. Demonstrating that fact is the purpose of this dissertation, and for that purpose we will need a suitable case study that can demonstrate not just how data can have a history, but the moments in those data's history that affect their reception and interpretation.

\section{American Fiction: A Contribution To A Bibliography}

In the previous section, I mentioned a hypothetical collection of American materials. A collection such as this could be found in several places: Gale Cengage, a commercial academic resource provider, makes available a digital collection of American texts, including individual facsimiles of texts, and the dataset as a whole for computational humanities work.\autocite{noauthor_american_nodate} Similarly, ProQuest offers a digital American fiction corpus for the same purposes.\autocite{noauthor_early_nodate-2} The \textit{Wright American Fiction Project}, a repository based at a public university rather than a private commercial entity, makes available American fiction titles for research, both traditional and computational.\autocite{noauthor_wright_nodate} All of these examples (discussed further in Chapter 4) have a shared history in the enumerative bibliography appropriately titled \textit{American Fiction}, compiled by Lyle H. Wright.

Thus, the not-so-hypothetical collection of American titles I previously mentioned began originally as a print bibliography. As a field, bibliography, or the collection and description of printed titles, is not as prevalent as it was during the early twentieth-century, when formative scholars such as W. W. Greg, Frederick Bowers, D. F. McKenzie, and Alfred Pollard began to theorize and codify the study of books and their production into terms that would evolve into contemporary fields we are more familiar with: book history, textual scholarship, and material culture. In the twentieth-century however, what was known as the New Bibliography emerged (discussed in detail in Chapter 1), and the field saw the production of a large number of resources that recorded and described titles, regardless of their  perceived aesthetic or historical value, and made their existence known to a wider audience. It was during this time that Pollard created the \textit{English Short-Title Catalog} (ESTC), and Jacob Blanck created the \textit{Bibliography of American Literature} (BAL), alongside a multitude of other, smaller and more narrow works that endeavored to provide a resource to the expansive world of print beyond what scholars would call the canon. 

It is during this period that Lyle H. Wright would compose the primary case study of this dissertation: a three volume bibliography titled \textit{American Fiction}. This bibliography, which constitutes Wright's lifelong work, covers the years 1775 to 1900, and lists 11,799 titles that fit under Wright's definition of American fiction in that span of time. From 1936 to 1966, Wright described and compiled these 11,000 texts in order to present them as a resource to American literary scholars, collectors, librarians, and students. His work was thorough, as he traveled to multiple libraries, delved into card catalogs, title-page collections, and auction listings in order to find texts that warranted inclusion in \textit{American Fiction}. 

\textit{American Fiction} has its deficiencies, however; some of them are by design and others are erroneous. As Wright notes in his preface, "In general, it has been intended to omit annuals and gift books, publications of the American Tract Society and the Sunday School Union, juveniles, Indian captivities, jestbooks, folklore, anthologies, collections of anecdotes, periodicals, and extra numbers of periodicals"\autocite[vii-viii]{wright_american_1939}. Wright's parameters for the bibliography purposefully excluded materials that were published in serial extras, leading to the exclusion of Walt Whitman's \textit{Franklin Evans} (1842) or Edgar Allan Poe's "The Balloon Hoax" (1844) to name canonical exclusions, alongside an untold number of non-canonical works that have gone undescribed. Wright was also interested in listing primarily fiction meant for adults and not juveniles. This means that some authors have absences in their lists that may seem odd to a human reader, such as Louisa May Alcott, whose \textit{Hospital Sketches} (1863) is listed, but \textit{Little Women} (1868), \textit{Little Men} (1871), and \textit{Jo's Boys} (1886) are absent.\autocite[7]{wright_american_1957} Wright's inclusion of autobiographical slave narratives such as Harriet Jacobs' \textit{Incidents in the Life of a Slave Girl} (1861) and Solomon Northup's \textit{Twelve Years a Slave} (1853) presents one of the more standout errors that denotes these works as fictional and contradicts the claims of those titles, potentially causing a modern reader to view \textit{American Fiction} with some skepticism as to its accuracy.\autocite[179, 242]{wright_american_1957} The previous errors are compounded by the fact that other fictional works by black authors, such as Martin Delany's \textit{Blake, or the Huts of American} (1859-1861), are absent in a seeming oversight. All of these individual cases reflect on the bibliography as a whole, asserting the conditions of its creation and the validity as a comprehensive source of American fiction titles. Each of these individual cases reflects an interpretive stance towards these items and posits an argument about them as to their apparent necessity in being recorded in a list.

It becomes apparent then, that the composition of a bibliography is more subjective a process than it might first be assumed. The creation of a bibliographer involves conscious decisions on the part of the bibliographer, and a list of American literary titles will necessitate interpretation. Wright, for his part, is clear about the standards he set for compiling a list of American fiction, but his process involved consulting with friends and colleagues, reading literary history, and revision, in addition to his own intellectual work and expertise in bibliography. This subject will be continued in chapter one, where I will argue for the interpretive capacity of data specifically in regards to the field of bibliography.

\section{Lyle Henry Wright}

Before beginning the critical discussion of Wright's work, it is also necessary to give a brief biographical summary of Wright's professional life, both to humanize Wright, and to explain further the context from which \textit{American Fiction} emerges.

Lyle Henry Wright (1903-1979) was a Huntington Library employee and California resident for most of his life. He was born in 1903 and graduated high school in 1921 before enrolling in the Southern Branch of the University of California (now UCLA). He began working part time at the Huntington Library in high school, but by 1923, when he was still a junior in college, began to work full-time at the Huntington in the photostat department. By 1928, at the age of 25, he began to work professionally in bibliography as an assistant bibliographer. Wright's only break from employment at the Huntington was during World War II, when in 1942 he enlisted in U.S. Army Air Corps. When Wright returned to the Huntington in 1945, he was promoted to bibliographer and the acting head of the reference department, positions he held until his retirement in 1966. Even after his retirement he continued his relationship with the Huntington, serving as a consultant to aid in the Huntington's American literature collection until 1971.\autocite[312-3]{roger_e._stoddard_lyle_1981}

Over the course of his career Wright published his three volumes of \textit{American Fiction}. His first was \textit{American Fiction, 1774-1850} (Wright I), published in 1939, after he had spent nearly a decade as a professional bibliographer. He would revise the first volume in 1948, after returning from the War, and soon after begin working on the second volume, \textit{American Fiction, 1851-1875} (Wright II), which was published in 1957. By 1965, he had revised and expanded the second volume, and a year later published \textit{American Fiction, 1876-1900} (Wright III). He retired almost immediately after the publication of Wright III.

Wright's influence is not to be understated. While his name is not common in American literary scholarship, \textit{American Fiction} has helped to inform the development of literary scholarship and the way scholars are able to access and discuss American materials. Chapter two will discuss the way Wright was influenced by the institutions he visited, specifically the American Antiquarian Society, and how he, in turn, affected the Society with his bibliography. Chapter 3 will discuss one of his most notable errors, the inclusion of Harriet Jacobs' autobiographical \textit{Incidents in the Life of a Slave Girl}(1861) in his list of fiction, and the context that informed his decision. As well, this chapter will discuss the consequences, both good and bad, of Jacobs' inclusion in \textit{American Fiction}. Finally, the last chapter will address the way Wright has been adopted by others and how his influence continues into the digital age as his bibliography migrates from analog media to digital.

By exploring Wright and his \textit{American Fiction} my goal is to respond to the provocation of Allison. As a case study, \textit{American Fiction} presents a fascinating confluence of events that inform its creation and transmission. These events are not dissimilar in some ways to the study of an individual text. Reading Wright as an author of a text, and not just a bibliographer organizing information with sterile precision, allows for an understanding of the data he compiled in its social and subjective capacities. Furthermore, Wright's work presents an instance of an analog example to a digital problem. \textit{American Fiction} as a collection of texts was composed in the context of a field that was pre-occupied with the theories that undergird the creation of collections and lists. Modern day scholars engaged in the creation of digital datasets, or in the adoption of digital datasets for their work, are participating in the same tradition of bibliographers such as Wright. Thus, an understanding and analysis of his work serves to bring a critical eye to the myriad of datasets modern scholars use in their work. 
