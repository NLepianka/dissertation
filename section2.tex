%%%%%%%%%%%%%%%%%%%%%%%%%%%%%%%%%%%%%%%%%%%%%%%%%%%
%
%  New template code for TAMU Theses and Dissertations starting Fall 2016.  
%
%
%  Author: Sean Zachary Roberson
%  Version 3.17.09
%  Last Updated: 9/21/2017
%
%%%%%%%%%%%%%%%%%%%%%%%%%%%%%%%%%%%%%%%%%%%%%%%%%%%

%%%%%%%%%%%%%%%%%%%%%%%%%%%%%%%%%%%%%%%%%%%%%%%%%%%%%%%%%%%%%%%%%%%%%%%
%%%                           SECTION II
%%%%%%%%%%%%%%%%%%%%%%%%%%%%%%%%%%%%%%%%%%%%%%%%%%%%%%%%%%%%%%%%%%%%%%


\chapter{DATA, BIBLIOGRAPHY, AND INTERPRETATION}
\section{Preface}

Before we can begin to explore Wright's \textit{American Fiction}, it is necessary to put into context some of the central concepts that will guide my readings of Wright's work. Primarily, we must first understand the idea of data, specifically as it concerns literary materials, and place it into the context of bibliography that Wright was working within. Wright, as a bibliographer, arranged and modeled data he had collected from various libraries across the United States in order to populate \textit{American Fiction}. This process, however, involved interpretive decisions on his part that affected how the data he collected was modeled and how it was then delivered to readers who approached \textit{American Fiction}. While Wright was composing \textit{American Fiction} in the mid-twentieth century, bibliography as a field devoted to the study, analysis, and description of printed books was entering into a golden era, so to speak. It is while Wright is compiling his list that some of the formative names in bibliography were theorizing about their role as scholars who work with the data that describes literary materials. What would come to be called the New Bibliography would conceive of itself as both a science and an art. As a science, bibliography was responsible for asserting claims as to the physical nature of a text and its production via evidence and logic. But bibliographers during this period also understood themselves as performing interpretations of the text and not just stating objective facts. In this way, the voices of the New Bibliography knew that through their descriptions of a text's physical nature they invariably affected how that text was perceived and read by others. Furthermore, because bibliographies as resources contain a large number of texts, these resources also offered arguments about the relationships amongst the materials they listed.

Bibliographies are an attempt to recognize the vastness of text, primarily published text.\footnote{Bibliographies often privilege published texts because they are what stand the best chance of leaving behind evidence that can inform a bibliography. Unpublished works can be featured and described in bibliographies, especially if they were well-preserved or cataloged. Bibliographies are not as prepared for media such as oral histories, where physical evidence is lacking or difficult to access.} Every bibliography is an admission of the enormity of printed literature, regardless of how limited or broad the definition of that term. By means of example, we may compare Jacob Blanck's \textit{Bibliography of American Literature} (\textit{BAL}, 1955) to Wright's \textit{American Fiction}. The \textit{BAL} is far more limited in its scope than Lyle H. Wright's \textit{American Fiction}, which is not to suggest that the \textit{BAL} is a lesser work than \textit{American Fiction}. Despite covering similar subject matter, the \textit{BAL} has different aims from Wright's work. The \textit{BAL} fills nine volumes and lists more than 40,000 entries, but is more discriminate in the authors it includes, preferring to list the works of those deemed the most influential or relevant American writers. As stated in the preface to the \textit{BAL}: "What concerns us is that at one time these books were read and for a span held positions of sorts in American letters."\autocite[xi]{blanck_bibliography_1955} This, naturally, is a subjective statement, informed by the expertise of the committee that assisted Blanck in compiling the \textit{BAL}, but a consciously interpretative statement nonetheless. Wright, however, was unconcerned with status, and concerned only with the question of whether or not a text was fictional. That an author only appeared once, or was relatively unknown even in their time, did not affect its inclusion in \textit{American Fiction.} 

In addition to the interpretive decisions that define inclusion into a list of titles, the way those titles are described by bibliographers also show discrepancies between what is considered important enough to tell the reader or what is within the scope of the bibliography. Below we can see a citation for the same work, Hawthorne's \textit{Blithedale Romance} (1852), as taken from Blanck's \textit{BAL} and Wright's \textit{American Fiction}, respectively.

\begin{displayquote}
\textsc{7611. The Blithedale Romance} \\ 
\textsc{Boston: Ticknor, Reed, and Fields. M DCCC LII}
 \\ First American edition. For prior publication \\ see preceding entry. \\ $\langle i \rangle$-vii, $\langle 9 \rangle$-288. 7 $1/8''$ x 4 $7/16''$ (edges plain). 6 $7/8''$ \\ full x 4  $3/8''$ (edges gilded). \\ $\langle 1 \rangle$-$18^8$.\autocite[11]{blanck_bibliography_1955}
\end{displayquote}

\begin{displayquote}
\textsc{Hawthorne, Nathaniel.} The Blithedale Romance . . . Bos- \\ \hspace*{1 pc} ton: Ticknor, Reed, and Fields, 1852.  288p. \begin{flushright}
\vspace*{-5mm} \textsc{AAS, AP, BA, BP, BU, H, HEH, LC, N, NYH, NYP, UC, UM, UP, UVB, Y}\autocite[153]{wright_american_1957}\end{flushright}
\end{displayquote}

The descriptions Blanck appends to his entries are substantial and specialized. The \textit{BAL}'s descriptions include title-page information, i.e. the publisher and year of publication (in Roman numerals, no less). Blanck includes as well the pagination system, which differentiates between introductory pagination (the $\langle i \rangle$-vii) and body pagination ($\langle 9 \rangle$-288). The paper size, and its properties, plain and gilded, are also described, with, finally, a collation formula ($\langle 1 \rangle$-$18^8$) that describes how the book was bound. What should become clear from the Blanck description is how precise it is in its attempt to describe the physical copy of the \textit{Blithedale Romance}, and in doing so must use a specialized discourse that is difficult to understand for those outside the field. Wright, on the other hand, describes texts more simply and in a manner that resembles modern citation practices. Wright provides the author, title, place, publisher, date, and a cumulative page count, all listed in a manner that is easy to comprehend without formalized bibliographic training. The only piece of information that may need explanation for Wright's description are the library codes that appear under the book description, which list the sixteen libraries where that specific copy of the \textit{Blithedale Romance} may be found (or could be in 1957). For both Wright and Blanck there is a question of perceived audience that informs how they model their information. Blanck has composed his bibliography primarily for librarians and bibliographers who would be able to read the specialized format of his data, while Wright appeals to a broader audience through his work, even when presenting the same information as Blanck; i.e. the 288 pages of \textit{Blithedale Romance} are given by both bibliographers, though Blanck goes further in presenting, via a bibliographic formula, \textit{how} those pages are numbered. 

What Wright, Blanck, and other bibliographers attempt to do in their bibliographies can not be said to simply list books they find, but rather to organize them into a schema, or a logically coherent model informed by a set of standards, that is understood by the bibliographer to best aid the reader of the list. For what purpose the reader may be viewing the bibliography is subjective and susceptible to change as scholars pursue different questions and conversations, but must nonetheless be considered in the creation of the bibliography. But while the bibliographer must consider their audience in the creation of a bibliography, they are also subject to what resources and institutions are available to aid them in their mission. This helps us to expand the textual nature of bibliography to a place beyond that of a simple bibliographer-reader relationship. The texts a bibliographer describe \textit{are} somewhere, with curators, catalogers, and archivists that preside over them and make a text's data accessible and able to be described by the bibliographer.

A helpful term that has become common within textual scholarship to describe this relationship is the concept of the assemblage. I use the term \textit{assemblage} because it concisely points to the wider social nature behind the construction of complex aggregations of data such as bibliographies. Ryan Cordell places the term within the sphere a textual criticism through his Viral Texts Project.\autocite{cordell_reprinting_2015} For Cordell, an \textit{assemblage} is a text which is ``defined by circulation and mutability,'' and is not beholden to its author, or even its editors and publishers, but can be liberated, reimagined and reinterpreted, by its readers. \autocite{cordell_two_2016} In this context, Cordell admits, it is an extension of D. F. McKenzie's ``sociology of the text,'' but with a focus on the life of the text after publication rather than before.\footnote{At the ``What is Critical Bibliography?'' panel at MLA 2017 in Philadelphia, MA, the term was discussed, though its meanings, the panel concluded, were variable.} While McKenzie's theory of textuality gave necessary attention to the stages of a text's life beyond that of the writer, it was primarily concerned with a text's life in the process of production, including printers, paper makers, and publishers. Cordell's use of \textit{assemblage} extends the sociology of the text to beyond production and into reception (not, however, at the cost of knowledge of production). 

Bibliographies are assemblages; they are texts like any other that are reliant upon a network, a system of relationships, of persons each with their own intervention in the process of composition, circulation, and transmission. My use of \textit{assemblage} is perhaps counter to Cordell's more progressive use in its application to textual criticism and bibliography, but that is because of the properties of a bibliography's composition and publication that influence not just the way readers interact with a text, but in fact, whether or not readers have access to the text at all.\footnote{The term \textit{assemblage} is rooted in discussions by Gilles Deleuze, Felix Guattari, Bruno Latour, and ``Actor Network Theory,'' a means of discussing the constructed nature of scientific classification. Cordell more directly has borrowed the term from Elizabeth Maddock Dillon, who employed \textit{assemblage} in a literary context to describe the state of agency amongst colonized subjects amidst the Haitian Revolution, remarking on the fact that lack of sovereignty did not necessarily mean a lack of agency; for Dillon, the concept of the assemblage was that which defined how meaning was made not just by colonizers, but allowed for meaning to be constructed by colonized subjects as well. Cordell's use of the term retains that sense of allowing for a text's nature to be influenced by its readers. See \autocite{dillon_obi_2013}; \autocite{deleuze_thousand_1987}; \autocite{latour_reassembling_2005}} A bibliography, as an assemblage, is not just the work of the bibliographer who compiled the lists, but a part of a larger network of connections between people and institutions. Certainly, Wright put great effort into compiling the three volumes of \textit{American Fiction}, libraries and archives in the process of creating \textit{American Fiction}; a total of thirty-one different institutional names can be found in each of the prefatory listing of libraries across the bibliography's three volumes. His descriptions, in turn, list for each entry the institutions in which a title can be found, presenting what Wright would call a census that informs the reader roughly how they might obtain a physical copy of a text. Wright's census not only reveals the network that informed his work, but additionally relies upon it as a testament to the truth of his entries. The presence of a text at multiple locations helps to verify Wright's claims that a given title exists, and can be viewed, if one should travel to the text's location. But Wright's explicit mentioning of the institutions involved in his work points to a level beyond the immediate relationship between the bibliographer and the reader and brings into focus the ways in which texts are preserved, cataloged, and made accessible. Finding a copy of a work listed by Wright does not involve Wright solely, but instead brings someone to a physical location where a library and its staff have cataloged and stored the text.\footnote{My definition and conceptualization of \textit{assemblage} in this context is partially informed by Darnton's communication circuit, which describes the various pathways that influence a text's existence as an object, from author to publisher to reader. In the creation of a bibliography, events such as the visiting of libraries for research or the holdings of a specific library could be seen as an act which constitutes another "node" on Darnton's circuit. See \autocite{darnton_what_1982}} 

Furthermore, I wish to bring attention to the fact that bibliographies such as Wright's function in part as arguments for how texts can be combined despite their different associations and relationships. Each act in the process of assembling a collection such as a bibliography requires conscious decisions. In the case of bibliography, these actions can be the choice of institution to visit, collection to browse, information given and recorded in the description, and naturally the parameters for what texts are admitted into the bibliography. At the same time, the bibliographer is subject to external contingencies; one can only record what has survived the sometimes hundreds of years between a text's composition and the bibliographer's description. Archives and libraries must actively choose what texts to purchase and catalog. These decisions will ultimately affect the bibliographer.

This rationale is what drives the process of assembling the assemblage. As stated, enumerative bibliographies represent one of the ways in which scholars have attempted to organize and represent aggregated information for use by other scholars. This argument, however, will also extend to the ways in which bibliographies and other collections of texts inform humanities scholarship, especially as scholars have increasingly resorted to digital tools for research. Aggregated collections of texts have become commonplace in modern literary studies as databases and repositories such as \textit{Early English Books Online}, \textit{Making of America}, and \textit{Eighteenth Century Collections Online}. Digital humanities scholars have led to initiatives to not only use and promote these sorts of collections but also in creating them and interrogating their place as scholarly objects. Scholars originally sheltered in the realms of textual criticism and bibliography have found fresh topics of discussion in the digital possibilities for texts, and so often find themselves confronting the rationale of aggregation when attempting to discuss how databases of texts function as works. Others have taken advantage of the affordances of digital collections and begun to read texts at distance, to use Franco Moretti's term, at a scale far past what is humanly practical.\autocite{moretti_graphs_2005} These scholars, however, are not necessarily approaching anything that we have not seen before. Enumerative bibliographies helped to create "big data" before the digital humanities. Contemporary work in the digital realm that finds value in considering the aggregate is informed by the work of the bibliographers that have helped to build that aggregate.

\section{Data}

To help further explain the interpretive capacities of bibliographies and bibliogrpahic descriptions, it is necessary to spend time with the components that inform the bibliographers work; that is, the data of texts. My preferred definition  of the term is that offered by the Open Archival Information System (OAIS) for its clarity, breadth, and nuance.\footnote{OAIS was adopted as ISO standard 14721 in 2002. Originally a product of the Consultative Committee for Space Data Systems, the OAIS prescribes a system for archival workflows and digital preservation. While I am primarily concerned with the way it has defined the term data and how this may help us approach bibliographical descripton, the OAIS covers a large, interdisciplinary and complex model of archives, preservation, and access. \autocite{consultative_committee_for_space_data_systems_reference_2002} To reference the ISO standard, see \autocite{noauthor_iso_2012}. For the OAIS in a humanities context and its relevance to the archives, see \autocite{kirschenbaum_.txtual_2013}} The OAIS definitions declares data as:
\begin{displayquote}
A reinterpretable representation of information in a formalized manner suitable for communication, interpretation, or processing. Examples of data include a sequence of bits, a table of numbers, the characters on a page, the recording of sounds made by a person speaking, or a moon rock specimen.\autocite[1-10]{consultative_committee_for_space_data_systems_reference_2002}
\end{displayquote}
This definition overtly states what are several important concepts for the term as they relate to bibliographical composition. First, the idea of data as "reinterpretable", rather than as statically informative, suggests a dynamic value to data. The use of "interpretitive" twice in the definition emphasizes the process of encountering data as reader-centric, as having its value determined and subsequently defined by the observer. The definition does not suggest data as  singularly objective, as possessing a finite amount of truth or factuality, as it assumes reinterpretability as inherent to the objects termed data, and so assumes multiple observers, each with their own means of interpretation that can produce variable outcomes. The word "processing" to some degree can be understood as nearly synonymous with interpretation, though with a distinction that processing refers to a method of interpretation and re-representation from a technical or computational point of view, rather than that of a human observer. The formalization of data and its ability to be communicated are codependent. In bibliographies, descriptions of titles obey a set sequence of details, determined by the bibliographer, but also often falling in line with a disciplinary consensus: author, title, place, date, with additional details that may be added to help facilitate the particular goals of the bibliography. The development of standards and conformity amongst representations of data is meant to allow the reader to both access individual units within the dataset, and see relationships amongst the units. As I will discuss in the New Bibliography section, while the characteristics of data defined here are meant to facilitate the reader's exploration and interpretation of the data, it does not necessarily always succeed in doing so.  

The OAIS definition, while it does hint at the unfixed nature of data, bears only a trace of the fact that data itself has already undergone interpretation when it has been sorted, arranged, pared down, etc. In its use of the term "representation", the prefix re-, or "again", suggests a state prior to that of data. Thus, we may say data is not innocent of human involvement, tampering, or subjectivity, and thus, data itself is not a neutral object, but susceptible to ideology via the the methods and practices that inform the person producing the data. Johanna Drucker has argued for more humanistic approaches to data in her recent work. Drucker has stated in multiple venues that "data", derived from the Latin \textit{datum} ("that which is given"), is taken, rather than given, and thus is \textit{not} \textit{data}, but \textit{capta}.\autocites[128-9]{drucker_graphesis_2014}[3]{drucker_humanities_2011} Drucker's point with the term \textit{capta} is to bring attention to the fact that data is "always interpreted," as summarized with her statement that "no data pre-exists its parameterization."\autocite[129]{drucker_graphesis_2014} Parameterization, according to Drucker, is a construction and an interpretation; the term \textit{capta} opens up the possibility of recognizing and "acknowledging the constructedness of the categories according to the uses and expectations for which they are put."\autocite[129]{drucker_graphesis_2014}  The parameterization that Drucker locates as inherent to data/capta is synonymous with the formalization and representation the OAIS definition prescribes. 

The way in which data is arranged, according to discrete categories--date, author, title, publication place, etc.--represents a process of assigning terms or classes to concepts based on an interpretation of a pattern that is observed. For Drucker, these classifications can distort and simplify the complexity of the phenomena being forced into a classification framework, while also erasing the ambiguity amongst the different items arranged and united under the same concept. Drucker's example in this case refers to nations, genders, populations, and time spans; all are politically determined and institutionalized concepts that constrain identities to fixed notions, even though they are not "self-evident, stable entities that exist a priori."\autocite[129]{drucker_graphesis_2014} The same can be said of bibliographical information, whose methods of description are institutionalized according to disciplinary expectations, the style guides of academic journals, and title-page printing conventions. Taxonomies that presume authorship as primary and as fitting into a set number of categories (i.e. pseudonym, autonym, anonymous) render the status of author to a position that may not be entirely indicative of the details that accompany the author's relationship to the work. In cases of anonymously published works, the state of a work being known to be by a certain author does not necessarily mean that the work itself is not anonymous, as it was actively published without the name, constituting a declaration of detachment from a source. An author may choose to detach themselves from a work for a variety of social, economic, or personal reasons. A bibliographer, however, may subvert those reasons in constructing a description of a text. Authorial attributions made to the work are performed due to the need for the bibliographical information to conform to a standard for reference by the reader of the bibliography. The work itself is filed under a certain protocol and so must be sought according to one's understanding of that protocol. 

It is not enough to point out the idea of the "constructedness" of data and its parameters. To take the concept a step further and explain how mutable data is there are two things we must understand. The first is how the data's signification---that is, the concept of accuracy as it refers to the idea of trueness or correctness of data---must ultimately also be constructed if we accept Drucker's claim. If data is meant to accurately depict a phenomenon, and the way the data is arranged is susceptible to subjective interpretation, then the ascription of accuracy must also be a qualitative value informed by interpretation. As will be discussed in a moment, the field of bibliography expresses a desire for accuracy in its descriptions and considerations of material texts. What is termed accuracy however is susceptible to the aims of the scholar compiling the bibliography, who determines the mode of description which then informs how observers of the bibliography can understand information. The bibliographical description must point to the reality of the text, but it is a reality understood by the bibliographer and assumed by any one who views the bibliography unquestionably. A humanistic approach to data, as Drucker calls for, would be aware of the variance, subjectivity, and capacities of data to signify more than what may be represented on a page or screen. In a bibliographical sense, such an approach would also take into account how the accuracy of the data's representation is understood, and by what measures the truth of a text's description corresponds to the text's publishing history, but also how much of the information is appended, by the bibliographer, onto the text with information that exists outside of the textual object. 

The second is that data has a history, it is transmissable, and thus it can be subject to editorial hands and their accompanying judgments, interpretations, and critical impulses as it moves among different representations and reformations. This is the primary way in which collections of texts demonstrate their properties of assemblage, as they represent the piecing together of information, some of which is found in the wild, so to speak, by the same hand that is arranging them, but also because the data may be found amongst the arrangements and collections of others. Wright, in his \textit{American Fiction}, consulted other bibliographies of early American publishing that preceded him, explicitly referring to such works as Oscar Wegelin's \textit{Early American Fiction, 1774-1830}, Merle Johnson's \textit{American First Editions}, P. K. Foley's \textit{American Authors, 1795-1895}, B. M. Fullerton's \textit{Selective Bibliography of American Literature, 1775-1900}, and the regional bibliographies of James Johnson and Lizzie Carter McVoy and Ruth Bates Campbell. This is in addition to more primary sources such as the \textit{Publisher's Weekly} digests.\autocite{wright_pursuit_1966} Each of these individual sources themselves arranged and described texts according to each compiler's own standards and interpretations; Wright, approaching these lists, would have had to confront how to change or otherwise adapt or omit the selections of those lists. 

The practice of attempting to classify literature is not innately bibliographical however. Literary works have been cataloged and organized according to their semantic content (that which bibliography attempts to avoid).\footnote{This sentiment is attributed to W. W. Greg: "To the bibliographer the literary contents of a book is irrelevant. This does not mean that special bibliographies should not be compiled, or that the merits of the works included, or somebody's opinion thereon, should not be recorded. It means that this is not the task of the bibliographer." \autocite[46]{greg_what_1913}} Genres, in particular, represent a means of understanding collections of texts according to their content as it pertains to following specific themes or stylistic conventions. Wai Chee Dimock, in her introduction to a special issue of \textit{PMLA} on genre, understands genres as "fields of knowledge," as she explains: 
\begin{displayquote}
Far from being a neat catalog of what exists and what is to come, genres are a vexed attempt to deal with material that might or might not fit into that catalog...The membership---of any genre---is an open rather than closed set, because there is always another instance, another empirical bit of evidence, to be added.\autocite[1378]{dimock_introduction:_2007}  
\end{displayquote}
Dimock would seem to agree with and anticipate Drucker in her conceptualizing of genre, which drifts into a more scientific rhetoric (empirical, open versus closed set, evidence). Dimock recognizes the problems of attempting to fit genres onto groups of texts. Dimock acknowledges the same interpretive nature of data through her thinking about genre: "The spilling over of phenomena from labels stands here as an ever-present likelihood, a challenge to any systemizing claim."\autocite[1378]{dimock_introduction:_2007} Dimock's argument represents a means of understanding the creation of collections (catalogs, to Dimock) and its effect in the literary realm, but through its relevance to semantic and aesthetic considerations of literature, rather than bibliographic. It is perhaps easier, however, to see how genres and literary traditions, which are more obviously and strongly connected to a literary object and its subjective qualities, represent aggregation efforts that can be restrictive in terms of the claims of what defines a genre, or easily upset when its expectations are not conformed to. Though what Dimock suggests is that the particular accuracy, or the fidelity, of texts to the genre they are assigned to is not the reason for the classification. It is instead, an attempt that can be productive, rather than restrictive, if considered in the light that it is only ever going to ``vexed,'' or failing to ever fully encapsulate literature as a whole. The vexed, or troubling and endlessly complicated process of trying to assign and understand the data of literary production was generative process that fueled early twentieth-century bibliographic scholarship.

\section{The New Bibliography and Wright}

To understand how our modern ideals of literary data manifest, we must first understand how they are in dialogue with, and in some capacity, still at the mercy of the bibliographic principles of the scholars who came before them. Bibliographic scholars who are referred to as the New Bibliographers discussed and codified the standards for the description of books and their arrangement in enumerated lists. These New Bibliographers emerged in the early twentieth century and systematized the process of book description, collection, and arrangement and prescribed set procedures for how other scholars should be able to view, find, and learn about different texts and works. It is at the same time that Wright would be composing \textit{American Fiction}, and while not as actively contributing his theories of bibliography in an explicit form, Wright nonetheless expressed his ideologies in the composition of \textit{American Fiction} that resonates with the New Bibliogrpahy. The process of systematization and prescription was not without its debates, despite aiming for objectivity, the standards and methods that New Bibliographers employed were sometimes embedded in interpretive and ideological decisions that nonetheless were codified as a standard practice that still persists today. Of particular interest in this section will be the interventions of those who considered the interpretive nature of bibliographic lists against those who dismissed enumeration as a lower form of bibliography due to its perceived lack of intellectual rigor. 

For W. W. Greg, an enumerated bibliography is a list of books described, organized, and compiled according to a "guiding principle."\autocite[41]{greg_what_1913} Theodore Besterman, on the other hand, offers a modified definition in his history of enumerative and systematic bibliography, preferring, rather than a "guiding principle," a "permanent principle." \autocite[2]{theodore_besterman_beginnings_1968} What the two definitions presuppose with the term \textit{principle} is that the bibliography is composed according to a deterministic framework that imposes simultaneously standardization and clarity unto the books described. Standardization and clarity are codependent in this case, as the conventions of bibliographical description offer a pattern of information representation that creates a comprehensible system for researchers and readers to understand the construction of the bibliography, and to therefore use it for its intended purposes. 

To more concretely define what is the referent of \textit{principle}, observe the following samples drawn from the listing for Herman Melville's works in the \textit{Epitome} of Jacob Blanck's \textit{Bibliography of American Literature} (\textit{BAL}, compiled by Michael Winship, Philip Eppard, and Rachel Howarth:\footnote{Blanck passed away in  December, 1974, before the final three volumes---seven (1983), eight (1990), and nine (1991)---were completed. The final three volumes were completed and edited by others, including Katherine Jarvis, Virginia L. Smyers, and Michael Winship. Winship is credited with the completion in all three volumes.}
\begin{displayquote}
13663. \textsc{THE WHALE}. Lon: Bentley, 1851.
\\
\hspace*{3 pc}\small 3v. For U.S. edition see next entry.
\\
13664. \textsc{MOBY-DICK; OR, THE WHALE.} NY: Harper \&  Bros., 1851.
\\
\hspace*{3 pc}\small Critical edition published in 1967 (No. 13711).
\\
13666. \textsc{PIERRE; OR, THE AMBIGUITIES.} NY: Harper \& Bros., 1852. \autocite[200-1]{winship_epitome_1995}
\end{displayquote}

Even from the sample of three listings, certain patterns that inform the construction of the bibliography become apparent even if they are perhaps unconsciously realized. Not shown in the above selection is Melville's name as a part of the description, which informs the larger organization of the list, as the titles of the \textit{Epitome} are all arranged alphabetically by author. Melville's entries appear between authors Cornelius Mathews (1817-1889) and Joaquin Miller (1837-1913). The author's name assumes the primary position for the bibliography's organization and thus is the first layer of an information hierarchy or taxonomy for conceiving of the texts listed. Beyond the author's name we find the titles of the texts included with the accompanying information of place of publication, publisher, and year. It is at the year that we see the next organizational pattern---chronological---emerge, as the  end of the description provides us with a means of knowing why the texts are listed in the order they are. The entry for \textit{Pierre} (1852) at the end of this extract is what provides us evidence that the first two entries referring to two separate editions of \textit{Moby-Dick} are in chronological order despite both listing 1851 as their publication year. In the case of Melville, this informational structure is useful in elucidating basic publication details about \textit{Moby-Dick}: its original title was not \textit{Moby-Dick}, but in fact \textit{The Whale}, the title under which it appeared first in London, England before its print run in America. The additional details provide further information, letting us know that while the British edition was first, it was the American edition that formed the basis of the first critical edition, described later in Melville's entry. 

From understanding the hierarchy of the information that has formed the bibliography, we may then surmise the supposed principles---guiding, permanent, or otherwise---Winship, Eppard, and Howarth assumed in their adaptation and interpretation of Blanck's bibliography.\footnote{Winship, Eppard, and Howarth explain in the introduction to the \textit{Epitome} the principles they assumed for their work in adapting and interpreting Blanck's bibliography: \begin{displayquote}"Our goal has been to provide a useful complement to the full \textit{BAL} rather than a replacement for it. We have followed the scope and style as set forth in Blanck's "Preface" in the first volume of \textit{BAL} and have limited ourselves to information contained in the published volumes. In particular we have not included editions of an author's works that have appeared since the publication of that author's list in \textit{BAL}, nor have we incorporated the few corrections or additions that have been discovered since publication of the original volumes."\autocite[vi]{winship_epitome_1995} \end{displayquote} 
The primary criticism one could leverage against this practice as a principle is that it does not turn the \textit{Epitome} into a reference that could point the reader to corrections over Blanck's errors, and would rather have the reader led to Blanck's entries by their own virtue, rather than seek to modify access to individual texts that Blanck does not describe, whether accurately or at all. In short, the \textit{Epitome} in this case represents bibliography for bibliography's sake.} The \textit{Epitome}, like the \textit{BAL}, privileges the author as the prime piece of information that forms a bibliographic entry. In doing so, it assumes, like any bibliography whose construction is primarily modeled by an alphabetical author listing, that the primary use of the bibliography will be in researching individual authors; that the first means of reference for the scholar is to locate the author whom has been judged responsible for a given text and proceed from there in other directions (i.e. referencing other authors or other texts within the author listing). The secondary characteristic, the chronological organization, is subservient to the primacy of the author. The chronological listing is framed by the author and so places the author's texts within a timeline, but only with texts associated with the same author. The wider world of publishing is not easily visible; Other texts published the same year as \textit{Moby-Dick} are not placed in connection with Melville's work. Due to the isolating effects of consigning titles under an author header, more difficulty, on the part of both the bibliographers and the readers, becomes apparent when attempting to obtain information that is not determined by authorial association. 

To return to Greg and Besterman's conception of principles, the distinction between Greg's use of \textit{guiding} and Besterman's \textit{permanent}, however, introduces complications into this supposition of how one constructs, and ultimately, obeys the principles the bibliographer lays out. For Besterman, \textit{permanent} indicates a sort of finity, constraint, and superiority on the part of the bibliographer compiling the bibliography that a term such as \textit{guiding} does not. What Besterman refers to with his idea of permanence is that which defines the pre-determined knowledge the compositor of the bibliography comes into the project possessing. This model of bibliography, that which is derived from what is ``known'' beforehand, demonstrates a disposition found among twentieth-century bibliography that imagines the discipline as a science, carrying the connotation of laws and facts, which are presumed to be stable, infallible, and observable (without considering who the observer is). When discussing St. Jerome's \textit{De Viris Illustribus} (fourth century CE), which Besterman deems an early example of systematic bibliography, albeit accidental, he claims Jerome "looked upon his compilation as a piece of theological propaganda. He did not put out his bibliography to to guide or to instruct, but to convert."\autocite[9]{theodore_besterman_beginnings_1968} Besterman reveals not just St. Jerome's predispositions here in terms of what a list of texts \textit{should} do, but his own as well in clarifying Jerome's motives. Besterman believes bibliographies should "guide" or "instruct", which in itself reveals the placement of bibliographers as curators of facts, as those who lead others to knowledge by which they can be informed according to the bibliographer's standards. 

Besterman's definition demonstrates a symptom of the New Bibliography in terms of its approach to the description, enumeration, and analysis of texts.\footnote{The \textit{Oxford Companion to the Book}'s entry for "New Bibliography" goes as follows:\begin{displayquote}
The New Bibliography involved 'the application of physical evidence to textual problems' (Tanselle) and one of its key achievements has been its systematic and rigorous methodology for describing such physical evidence...It's most enduring (and contentious) legacy has been in the field of editorial theory, where its intentionalist and eclectic editorial principles (refined first by Greg, later by Bowers, and more recently by Tanselle) long dominated the production of critical literary editions.
\end{displayquote}\autocite[963]{suarez_oxford_2010}. \autocite[40]{tanselle_bibliographical_1988}} Other scholars have explored the premise that New Bibliography conceived of itself as scientific in nature and constructed the field in accordance with mechanical and technical methodologies.\footnote{G. Thomas Tanselle has published a lengthy article on the subject of bibliography and science, where he revisits the debate. He argues that science was only ever an analogy to refer to the empirical or systematic nature of bibliography and editing and to contrast it with the bibliophilic tradition that preceded the New Bibliography. Tanselle does, however, also lay some of the blame for the debate on bibliographers themselves for continued use of scientific terminology. \autocite[57]{tanselle_bibliography_1974}. More recently, some scholars have also approached the debate by discussing it as a symptom of the cultural moment the New Bibliography emerged. Amanda Gailey discusses the Greg-Bowers method and the scientific rhetorical grounding of it as a response to the Cold War and the need to compete with scientific fields for federal funding and prestige. \autocite{gailey_proofs_2015}. Amy E. Earhart has tied the Greg-Bowers method and its focus on the "purity" and "corruption" of the text to both Bowers' interest in dog breeding and the resistance of textual scholarship to diversity issues. \autocite[36-7]{earhart_traces_2015}. These discussions are all almost entirely dedicated to discussing the debate in regards to textual criticism and editing, however, and not necessarily covering biliography as whole and its other subsidiary methods of description or enumeration.} New Bibliographers are not hesitant to affirm this claim. As Greg notes: 
\begin{displayquote}
Facts are observed and catalogued by the systematizers, and then suddenly, as if by chance, an idea is born that introduces order and logic into what was the mere chaos, and we are in possession of a guiding principle, of an instrument of thought and investigation, that may transform the whole of our relation to knowledge or alter the face of the physical globe.\autocite[41]{greg_what_1913}
\end{displayquote} 
Bibliographers certainly belong to the same population as those Greg calls ``systematizers.'' It is the responsibility of the bibliographer to order facts according to those "guiding principles" which in turn produce knowledge. For Greg, knowledge seems to only come out of "order and logic" once it has been applied to ``facts,'' or what may be termed in other circumstances as ``raw data.'' Greg, however, seems to consider enumerative bibliographies to be ``raw data'' that have not yet had ``order and logic'' applied to them, and so  are not necessarily productions of knowledge themselves, but only aids to its manufacture. Philip Gaskell is of a similar opinion; in his \textit{New Introduction to Bibliography}, appearing almost sixty years after Greg's above statement, Gaskell understates enumerative bibliographies as useful, but not the "purpose" of bibliography. Instead, their job is to aid in the study of literature as reference tools and to aid bibliography in deterministically proliferating "accurate" texts. \autocite{gaskell_new_1972} A. S. G. Edwards makes a similar claim in discussing enumerative bibliographies: "One accepts, I assume, that the aim of any enumerative bibliography is to achieve as close an approximation to definitiveness as is practicable." \autocite[331]{edwards_problems_1981-1} Terms such as "accuracy" or "definitive" depict a particular stance towards information that belies the ultimately subjective nature of texts and their production. Unironic or unqualified use of the word "accuracy," or even its corresponding term--precision--forgets that in the case of texts, what is "accurate" is ultimately subject to and defined by the whim of the bibliographer themselves.

Fredson Bowers is conscious of this detail when he attempts to conceive of the place of enumerative bibliographies---or, to him, catalogues, handlists, or checklists, terms which would likely not stand up to the scrutiny of formal librarians. He, like Gaskell, diminishes the role of enumerative bibliographies by making them subservient to descriptive bibliographies: \begin{displayquote}
Their primary purpose is to make available a listing of books in a certain collection or library, or else in a certain field, such as a specific period, a particular type of literature, a definite subject, or an individual author. Noting the existence of these books is the end-all and be-all of a catalogue, and under ordinary circumstances only the minimum of identifying details is provided, as author, title (abbreviated when necessary), the date and possibly the place of publication, and occasionally the format. Some catalogues may include the name of the printer or publisher, or both. The writer may compile his list partly from other catalogues and partly by personal examination of the books, supplemented by notes furnished by contributing libraries or scholars; but except in extraordinary cases he is not concerned with the textual history, circumstances of printing, or variation within issue (sometimes even within edition) of the books listed.\autocite[3]{bowers_principles_2005}
\end{displayquote}
What Bowers does note, however, is that, ultimately, these mere lists are subjective. While disparaging enumerative bibliographies whose "end-all and be-all" is simply stating their existence, he notes that details included in the lists are not always absolute. The compiler of the list is in control of what information is provided to the reader, and, in effect, can determine how the list is used as a research and reference tool. Bowers perhaps is seeing the nuance of Greg's "guiding principle" that Besterman's "permanent principle" discarded. To Greg's credit, he saw bibliography as empirical, based on evidence, but not rationalistic. In the ironically titled "Rationale of the Copy-Text," Greg argues that an editor's choice of a parent, or source, text which guides the creation of a new edition should be informed by the most "authoritative" copy of the text that can be located (by means, most likely, of an enumerative bibliography that goes uncredited). But Greg carefully notes that "authority" is always relative, never absolute. \autocite{greg_rationale_1950} For Greg, expertise---formal training and education in a subject---allows for the ability to determine "authority." But Greg does not necessarily turn his qualification of subjectivity onto himself. David F. Foxon, however, is apt to do so, as he targets both Bowers and Greg: "My researches suggest that some at least of our accepted conventions result from the idiosyncrasies of individual scholars; these were uncritically adopted by others and have finally come to be regarded as scientific."\footnote{\autocite[7]{david_f._foxon_thoughts_1970} In a memorial essay on Foxon, James McLaverty claims that Foxon's criticism against Greg, Bowers, and the enshrined practices of bibliography were derived from two sources: first, the development of technology in the 20th century, specifically the ability for scholars to xerox pages and compare them for both analysis and description. And second, that traditional methods were flawed in their ability to differentiate among editions.\autocite[103]{mclaverty_david_2001}}

Foxon's criticism, again like most New Bibliograpy discussions, is focused on description, as his primary example in this case refers to Greg and Bowers' apparent disagreement about labeling recto (i.e. the right side page in a printed book) and verso (i.e. the left side) pages of a printed leaf and its adaptation into collation formulas despite the convention being, to Foxon, obviously illogical.\footnote{To explain further, the convention Foxon is discussing holds that one should only ever explicitly reference verso pages, with either a \textit{b} or \textit{v} while recto pages go unremarked. Foxon addresses the fact the Greg and Bowers agree on the method but not its meaning. The lack of a marker that signifies recto or verso pages on a leaf, to Greg, says that the entire leaf should be considered, rather than a page, but Bowers claims that leaving the signature unmarked would suggest recto unless context implied differently. This moment of disagreement points to ambiguity that Foxon claims is antithetical to a scientific system. \autocite[8-9]{david_f._foxon_thoughts_1970}} Foxon's critique is not irrelevant to enumeration, however. Certain standards as to the arrangement of bibliographies are in place that obfuscate the interpretive nature of enumeration. Alfred W. Pollard lists three primary methods of organization that are prevalent in bibliographies:\autocite[133]{pollard_arrangment_1976}
	\begin{enumerate}[label=(\roman*), leftmargin=1in]
	\item Alphabetical
	\item Chronological
	\item Logical
	\end{enumerate}
Of relevance to this discussion are the first two points, which correspond to the major American literature bibliographies, including Wright.\footnote{Not discussed here is what Pollard means by the term "logical", where he considers the "natural sequence" of a subject. An  example here is a bibliography of mathematics, which Pollard states would have to be subdivided into smaller sub-topics (arithmetic, algebra, etc.) to be comprehensible and useful. \autocite[137]{pollard_arrangment_1976}} While stated in 1907, these methods of classification, present long before the New Bibliography, and enduring long after its heyday and the emergence of critical theory, have affected the way in which information is presented and, thus, interpreted. As argued by Pollard, the proximity of pieces of information suggests a relationship. A bibliography always possesses claims as to the relationship amongst the texts it lists, whether intentional of not. For Wright's \textit{American Fiction}, we can immediately assume a few qualities about the texts without reading them. First, the texts all share a genre: fiction. Second, the texts are produced within the same geopolitical area, i.e. America. For Pollard, the arrangement of the bibliography should help to acknowledge connections amongst the listed texts. Arundell Esdaile acknowledges this when he lambasts the concept of alphabetical organization: 

\begin{displayquote}
Some bibliographers have simply sorted the titles into the alphabetical order of authors; but that is mere intellectual laziness or want of imagination (perhaps the same thing); for while the alphabet enables the searcher to get access in a library to a particular book of whose existence he is aware, or, it may be, to refresh his memory as to a title or date, or other detail in the title, it serves no other purpose. The alphabet does nothing to collocate material bearing on the same or a closely allied side of the subject; it serves you up impartially the prunes and prisms together.\autocite[364]{esdaile_students_1954}
\end{displayquote}
Esdaile's mantra for the use of referential materials and bibliographies specifically is that they should be "illuminating" in their organizational method for the researcher.\autocite[35]{esdaile_students_1954} The bibliographer's task and thinking should be directed towards its organizational structure, as failure or indeterminacy in this area would mean the reader of the work, the student and would-be researcher, would "lose his way."\autocite[20]{esdaile_students_1954} Esdaile's tone is patriarchal and authoritative, not dissimilar to that of Besterman's, in that both place the bibliographer in a superior position over the reader. Esdaile does, however, hold the bibliographer accountable for their organization should it present poorly aggregated material that does not allow space for interpretation via its combination of bibliographic descriptions. The fundamental aspect of enumeration is to ensure an open-ended but well-mediated pathway between the subject of the bibliography and its reader. \textit{Collocation} is a favored term in Esdaile's manual as it hints at the interpretive nature of proximity. Esdaile's use of the term references patterns that should, ideally, be easily perceivable or possibly emergent when one entry is compared to many others, seemingly in the same physical space of a page, section, or chapter, as physical separation caused by a critically detached organizational scheme, such as alphabetical listing, hinders the reader's ability to recognize latent patterns. 

Esdaile agrees with Pollard, who also considers alphabetic listing as unfavorable. On the one hand, alphabetical order is, according to Pollard, the most fundamental organizational system that one can assume a literate person will recognize; it benefits from being inherent to the ability to read, and thus is the simplest. This though is precisely the reason why Esdaile declares it lazy and serving "impartially the prune and prisms together." Alphabetic organization is impartial because it is detached from the subject of bibliography and bearing no relation to the connections, in the case of literary bibliographies, among authors---the primary subject of alphabetization. A relationship inferred between Melville and his colleagues in the \textit{Epitome}, Cornelius Mathews and Joaquin Miller, purely on the basis of their proximity in an alphabetical listing would of course be erroneous. Any information that could emerge is at the mercy of chance. More consistently, research would require further digging into the entries for other details---year, publisher, collation, etc.---to form a more logically sound thesis. Thus, the the complexity of the bibliography's organization is increased without substantial aid to the information it provides\footnote{An exception is in the case of family relationships where patronymics remain static. For example, Amos Bronson Alcott and Louisa May Alcott are together in the \textit{Epitome}. A relationship could be inferred by a hypothetical unfamiliar scholar, though it would require confirmation. This is seemingly the only instance where alphabetization does help the reader understand a relationship between entries, but the limited application does not justify the lack of coherence amongst the other 279 authors in the \textit{Epitome}.} The simplicity, though, is why Esdaile also claims it is so widely used; its ease of access for the reader to grasp, despite the fact that it offers no new information and produces no knowledge. The "prunes and prisms" Esdaile mentions describe his ultimate opinion on the practice: it looks good, or has an aesthetic quality, but nothing more.

It is worth mentioning that when Pollard lays out his three common categories, he lists alphabetical as "according to the names of authors."\autocite[133]{pollard_arrangment_1976} Traditional print bibliographies often organize themselves as such to allow the author to occupy the prime position, both in their overall organization scheme and at the level of the individual description. Author name however, while seemingly objective and certain, can introduce a variety of interpretive outcomes in the arrangement of entries, and those outcomes are informed by the bibliographers' own position and thinking about the purpose of their list. Recalling the example at the beginning of the chapter, The \textit{BAL} is entirely guided by the author as a central figure for the way it presents its information and how it divides its volumes. Meanwhile, Wright's \textit{American Fiction} is divided into volumes by chronology (1774-1850, 1851-1875, and 1876-1900), but within each volume the listed titles are arranged alphabetically by author. Wright provides a more meaningful organizational structure (e.g. chronological) above the level of author, but still defers to the author within a single volume. While Pollard and Esdaile do criticize alphabetical listings, they both allow the concept of the author as first and foremost in a description go with little comment. Pollard does claim the "reader who already knows the book which he wants will be able to find it at once under the name of its authors," but this only hints at the root issue of the arrangement: a reliance upon the assumptions that guide the bibliography.\autocite[134]{pollard_arrangment_1976} The primary assumption being that a scholar seeks information based on a larger, overarching question that presumes the author or name listed as relevant to the conversation, when it is not necessarily always the case. Nineteenth-century American publishing, of course, did not lack its share of anonymous and pseudonymous titles. An alphabetical list according to author is immediately troubled by this fact. Wright's first volume of \textit{American Fiction} (1774-1850) begins with an anonymous entry: 

\begin{displayquote}
\textsc{An Account of the Marvelous Doings} of Prince Alcohol, as Seen \\ \hspace*{1 pc} by One of His Enemies, in Dreams.  [N.p.]  1847.  72 p.  12mo \begin{flushright}
\vspace*{-5mm} \textsc{NYP}\autocite[1]{wright_american_1939}\end{flushright}
\end{displayquote}

When an author is absent, Wright naturally defaults to a title, but this not only disrupts the inherent order of the bibliography Wright wished to enforce, but also serves to bury anonymous titles in odd places that makes tracing them more difficult, especially when titles have been shortened in the description.\footnote{Wright periodically does shorten the titles of works he describes when their titles are considered too long to practically list. For his explanation on this matter, see \autocite[ix]{wright_american_1939}.} Further complicating the scheme, the listing is interjected with notes that reference known pseudonyms of authors. When looking for Mark Twain in the second volume (1851-1870), readers will find in between title entries a note to find the works of Twain under Samuel Langhorne Clemens.\autocite[341]{wright_american_1957} Similarly, searching for Fanny Fern will direct the reader to instead look under Sarah Payson Willis Parton.\footnote{\autocite[120]{wright_american_1957}. Wright does, however, denote names absent from the title page with square brackets ([]), which does include autonymous names that supplant pseudonyms; he physically places distance between the title of the work and the listed author name in this case. Blanck in the \textit{BAL} also defaults to autonyms for the authors he lists, though there is little to help the reader realize this. Mark Twain is found under Samuel Clemens in volume 2. This introduces some difficulty for the reader at a practical level who may not go into the bibliography with the knowledge of Blanck's organization, and so goes searching volume 8 for Twain, rather than the correct volume.} For a print bibliography, this can mean searching through the text to find information in a different place than one suspects. In a bibliography that spans multiple volumes, this can pose issues of accessibility or impracticality if a researcher must sift through multiple printed copies to find the information they desire. \textit{American Fiction} is not wholly consistent in this case, however. For a text such as \textit{Incidents in the Life of a Slave Girl}, which was published by Harriet Jacobs pseudonymously as Linda Brent, Wright instead conflates the two names into one description: 

\begin{displayquote}
\textsc{[Jacobs, Mrs. Harriet (Brent).]} Incidents in the Life of a \\ \hspace*{1 pc} Slave Girl. Written by Herself... Edited by L. Maria Child. \\ \hspace*{1 pc} Boston: published for the author, 1861.  306 p.\begin{flushright}
\vspace*{-5mm} \textsc{AAS, BP, H, HEH, LC}\footnote{\autocite[179]{wright_american_1957} More about the oddities of the inclusion and description of Harriet Jacobs in a bibliography of fiction will be explored in a future chapter.}\end{flushright}
\end{displayquote}

The decisions for how authors are designated in the bibliographical descriptions represent a moment where the principles of aggregation, here inclined towards authorship and alphabetization, produce friction against the concept of bibliographies seeking to produce accurate and definitive lists. On the one hand, the preference of Wright, and other bibliographers, for autonymous author entries for their listing is deferential to the work of the person who created the work. The practice, however, also represents a process of interpretation of data that runs counter to more dominant discourse of these authors by literary scholars who are supposedly meant to be served by the bibliography. Mark Twain, rather than Samuel Clemens, it can be argued, is the dominant name attached to the works of \textit{Huckleberry Finn} or \textit{Connecticut Yankee}, and the suggestion of Clemens as the author dissociates these texts from the character of Twain and the standards of literary scholarship. The same can also be said of the less canonical authors such as Fanny Fern, the name attached to even the most recent editions of her works.\footnote{See Susan Belasco's edition of \textit{Ruth Hall}. \autocite{fern_ruth_1997}} Bibliographical descriptions of the texts that abandon the common discourse of author reference offer a competing claim as to the creator of the work. This is an act of interpretation that demonstrates the gap between the the way the literary scholar conceives of a work versus that of the bibliographer. Wright inherently recognizes this by providing the signposts in his bibliography that point researchers from the pseudonyms, which are assumed to be sought first, and their place in the listing to Wright's preferred method of classification---the autonym. 

This issue, I would argue, is bound to the New Bibliographical desire for accuracy, but enforces that these concepts are subjective when it comes to attempting to codify and arrange a subject such as early American writing. Wright's bibliography demonstrates an ideal that understands the author as synonymous with the person writing and publishing the work, rather than observing and adhering to what the text claims about its author. This makes sense in cases such as Washington Irving, whom is used as the authorial reference in regards to his pseudonymously written works. For example, \textit{The Sketchbook} (1819) is composed by Geoffrey Crayon, and the \textit{History of New York} (1809) is written by Diedrich Knickerbocker, but these author figures are themselves fictional characters created by Irving.\footnote{In the works of Irving mentioned, Knickerbocker and Crayon are narrators, but are also telling stories about themselves, implicating their own fictionality in the course of the narratives.} However, in the cases of Twain and Fern, these issues are less clear. Neither Fern nor Twain are fictional characters within the texts that bear their names as is the case with Crayon and Knickerbocker. The writings of Fern, including her earliest published monograph, \textit{Fern Leaves from Fanny's Portfolio} (1853), does not suppose Fern to be a fictional entity, but a pseudonym that gives the author an alternate identity, a different means of referencing the real author. The distinction, however, is that Fern and Twain adopt the literary status their counterparts, Clemens and Parton,  names that are leave behind in the course of both their publishing and circulation, wherein the names printed alongside the text become inherent to how the text is received and read. Scholars working with the texts continue to hold Twain, rather than Clemens, as the author figure. Thus, when \textit{American Fiction} demotes Twain and Fern to pseudonyms while promoting Clemens and Parton to author figure in its organization, it does so counter to literary scholarship's standards. Instead, it has located its concept of accuracy in a discourse outside of the realm of the subject it is responsible for enumerating.

When a bibliography is running in opposition to its audience, it ends up making the process of turning data into information more difficult. Pollard prefers, over the alphabetical listing that, in conjunction with a fixation on the author figure, complicate research, the chronological format such as that used by Wright as the dividing line between the three volumes of \textit{American Fiction}. For Pollard, chronology is easily comprehensible, similar to the alphabet in terms of the capabilities of the assumed reader, but is in general more generative and expressive as an organizing principle. Wright covers the years 1774 to 1900 in \textit{American Fiction}, with each volume containing a division of that timeframe: 1774-1850, 1851-1875, and 1876-1900. This timeline is seemingly arbitrary to the reader, as Wright does not offer comment as to the \textit{why} of his demarcation. The most Wright provides is some meditation as to what he perceives as the general overarching themes of the books included in the bibliography. From the preface of Wright II (1851-1875):
\begin{displayquote}
The momentous events that occurred during this quarter century are reflected in the fiction of the period. The slavery question, pro and con, was the theme of scores of novels, and as many more covered the Civil War, a national catastrophe that induced authors to attempt to be more realistic in their writing. The westward flow of the population was not overlooked...During the 1850's the sentimental novel reached its peak in popularity, aided and abetted by the large increase in women writers. And the woman's rights movement gained impetus through the numerous novels and short stories which presented it in a sympathetic vein. Religion, including controversies between denominations, was also a favorite subject with authors.\autocite[vii]{wright_american_1957}
\end{displayquote}
These themes, however, are porous and not bound solely to the second volume of \textit{American Fiction}, as several example titles that align with Wright's framing can be found in Wright's first volume. Let us return to the first anonymous entry in \textit{American Fiction} I mentioned previously, \textit{An Account of the Marvelous Adventures of Prince Alcohol} by "One of His Enemies," a temperance novel addressing the religious controversies Wright describes.\autocite[1]{wright_american_1939} Similarly, an 1849 novel, \textit{Amelia Sherwood; or, Bloody Scenes at the California Gold Mines}, references westward expansion.\autocite[5]{wright_american_1939} J. Elizabeth Jones' \textit{The Young Abolitionists; or, Conversations on Slavery} (1848), published by the Boston Anti-Slavery Office is obviously in dialogue with the "slavery question, pro and con."\autocite[156]{wright_american_1939} Since the only apparent justification for Wright's chronological division are his thematic considerations, it is worth addressing how that division in the case of American history here, and of chronological classification systems generally, are to some degree always arbitrary and reliant upon judgment. Wright's thematic reasoning offered in the prefaces of his three volumes provide a general sense of contents, but do not clearly elucidate with specificity the nature of the time period bound by each volume.  

Rationalizations attempting to explain Wright's demarcations meet inevitable walls in their logic but further show the divergence from literary history and criticism that a bibliographical work demonstrates. The second volume beginning at 1851 breaks from F. O. Matthiessen's pinpointing of 1850 as the first year of his "American Rennaissance" (1850-1855).\footnote{Matthiessen first published \textit{American Rennaissance} in 1941. Wright could have known Matthiessen's work by the time the second volume was being compiled. \autocite{f._o._matthiessen_american_1980}} The division between volumes by year that Wright draws causes the individual author bibliographies to become split, with works by Hawthorne and Melville included in all three volumes.\footnote{Posthumous works by Hawthorne, who died in 1864, are located in the third volume. Both \textit{Doctor Grimshawe's Secret} (1883) and \textit{The Dolliver Romance, and Other Pieces} (1876) are described there.} Book history complicates Wright's chronology significantly by moving events and innovations with drastic effects on American publishing earlier than 1851. Copyright (1790 with revisions in 1802 and 1831), the postal service (1792), continental railroad (1869), telegraph (1844 with its transcontinental implementation occurring in 1862), and dominance of the machine press over the hand press by the 1840s shift bibliographical qualities of texts outside of Wright's own parameters.\footnote{\autocite{robert_a._gross_history_2010}. See essays by Meredith McGill on copyright and the postal service; Richard R. John for a discussion of early postal service operations, and their relationship to the stagecoach industry; and Robert A. Gross' introduction for discussions of the telegraph and railroads. See Scott E. Casper's introduction where describes the gigantic shift in the amount of railroad coverage began in the 1840s.  \autocite{scott_e._casper_history_2007}. See \autocite{jen_a._huntley-smith_print_2002} for a discussion of book history in the American west as result of the telegraph and railroad.} The bibliography's chronological ordering principally based on theme and disconnected from a more specific historical perspective means that Wright has taken the most interpretive step in organizing the publication dates of the titles he lists. This puts a constraint upon the researcher, whose ability to evaluate events, phenomena, writers, or subjects that span years outside the scope of single bibliography find themselves facing, at a practical level, more difficulty in searching and comparing the information Wright offers than those who find their questions approachable with a single volume.\footnote{Amusingly enough, Wright's division of the three volumes does show an interesting pattern. Volume 1 contains just under 2200 entries (Wright does not provide exact numbering, but an estimate); volume 2 contains 2832 numbered titles; and volume 3 enumerates 6175 titles. Physically, each subsequent volume is larger than the other, visually suggesting that as time moved forward, more and more American titles were published. This a base assumption that could be more complicated; there may be better preservation efforts for more recent titles, or they have had easier times surviving. The institutions Wright traveled to may have privileged later titles over earlier ones, generally. Or, Wright's own standards and selection principles grew more relaxed.} 

Lastly, the selection principles of a bibliographer are informed by the biases of the bibliographer (and the institutions the bibliographer visits). The reader may be unaware of these biases, yet they will affect the text. Bibliographic theorists such as Pollard and Esdaile do not consider bias in their philosophies of the arranagement of bibliographies, yet it is an unavoidable component of the process. Again, Wright is a useful example as one whose goal of compiling a list of early American fiction is not as fundamentally clear as it may first appear. In the preface   for each volume, Wright presents a similarly worded statement in an attempt to clarify his aggregation process: "The design of this bibliography is to list the American editions of novels, novelettes, tales, romances, short stories, and allegories, in prose, written by Americans." \autocite[ix]{wright_american_1939} This statement is not as all-encompassing as a reader may suppose, however; Wright excludes, despite their relevance to the above system: "annuals and gift books, publications of the American Tract Society and the Sunday School Union, juveniles, fictitious Indian captivities, jestbooks, folklore, collections of anecdotes, periodicals, and extra numbers of periodicals," as well as essays.\autocite[ix]{wright_american_1939} Wright is honest in what this may mean for his entries, as he notes that his parameters cause some questionable exclusions even for canonical authors; Poe's "The Balloon Hoax" (1844) and Whitman's \textit{Franklin Evans} (his 1842 temperance novel) are consciously excluded because of Wright's decision to omit extra numbers of periodicals.\footnote{As reported by Wright, "The Balloon Hoax" was published in \textit{The Extra Sun}, April 13, 1844. \textit{Franklin Evans} appeared in \textit{The New World}, II, No. 10, Extra Ser., No. 34.} While some of his omissions seem to suggest themselves as obviously outside the bounds of fiction (essays and jestbooks), others raise more questions or are in need of an argument for exclusion that is not provided.

The categories of "juvenile" or "fictitious Indian captivities" raise a question as to why these would not be desirable in Wright's bibliography, and its effects can be clearly noted when browsing the listings. The exclusion of juveniles, for example, results in Louisa May Alcott's section of Wright II to be lacking in her most prominent works. Wright lists for Alcott five titles: \textit{Hospital Sketches} (1863), \textit{Hospital Sketches and Camp and Fireside Stories} (1869), \textit{Moods} (1865), \textit{On Picket Duty, and Other Tales} (1864), and \textit{Work: A Story of Experience} (1873). Absent are the arguably more popular and relevant works for reference to Alcott: \textit{Little Women} (1868), and its sequels   \textit{Little Men} (1871) and \textit{Jo's Boys} (1886, this text would be located in the third volume were it included). Wright's judgment, then, considers Alcott's major works as outside the bounds of more "adult" literature that he wishes to enumerate.\footnote{"Adult" becomes a term Wright adds to his preface in future volumes and revised editions.\autocite[vii]{wright_american_1957}} This is, again, an interpretive judgment that offers an argument both to claims as to what is valuable within the realm of American publishing and thus what should be accessed by researchers, with the consequence of limiting or obstructing access to a text based on a designation that would warrant exclusion. A genre such as the "juvenile" is less stable than the deterministic methods of arranging and presenting bibliographies acknowledge, in so far the methods operate on a completely binary system of either including or excluding an item in their arrangement. Wright would even admit at the end of his career that the category of juvenile was a never-ending headache for him, as the characteristics of the genre continued to evolve for him. A text being considered as a juvenile was the root of most of the revisions he made to \textit{American Fiction} in second editions as he continued to reinterpret the texts he lists.\autocite{wright_pursuit_1966}

Wright's other major oversight here is the omission of periodicals, which affects late-twentieth and twenty-first century scholars who have devoted increased attention to serial publication and in recovering works by marginalized writers who published their works in specialized serials. The expansion of communication technologies and the industrialized machine press gave rise to serial novels as a prevalent form of fiction and literary writing that original embodied such works as Edgar Allen Poe's \textit{Arthur Gordon Pym} (1838), Melville's \textit{Israel Potter} (1855), and even Harriet Beecher Stowe's \textit{Uncle Tom's Cabin} (1852). These works would eventually come to be found in book form that allowed them to be noted and described by Wright. Other serial publications were left out of this, however, including works by African-American authors, such as Martin Delany's \textit{Blake, or, the Huts of America} (serialized in the \textit{Anglo-African Magazine} in 1859), which is left undescribed by Wright. By ignoring serials, Wright also coincidentally ignores a prominent venue for black writers in the nineteenth century. As Eric Gardner has stated, black serials were "\textit{the} central publication outlet for many black writers---and especially for texts that were \textit{not} slave narratives."\autocite[10][emphasis Gardner's]{gardner_unexpected_2009} Additional casualties of Wright's method include Julia C. Collins' \textit{The Curse of Caste; or, The Slave Bride} (1865) as well as Frances Harper's \textit{Minnie's Sacrifice} (1869). 

On the other hand, Wright also reveals errors of judgment on his part, especially with regards to African-American literature. Wright offers in a bibliography of fiction, autobiographical works such as the previously mentioned Jacobs' \textit{Incidents in the Life of the Slave Girl}, and Solomon Northup's \textit{Twelve Years a Slave} (1853). The inclusion of these materials suggest that Wright did not believe the accounts he read as based in truth, perhaps due to the sentimental writing styles that aligned them more with mid-nineteenth-century fiction.\footnote{This subject will be pursued in depth in Chapter 3. For now, it is important to know that at the time of \textit{American Fiction}'s composition, the autobiographical nature of a work like \textit{Incidents} was not clear and there was no consensus until after the text's recovery in the 70s and 80s. Jean Fagan Yellin published her work authenticating \textit{Incidents} in 1981. See \autocite{yellin_written_1981}.} In addition to the exclusion of serials which barred black authors from receiving more descriptions, the inclusions also present problems when they become described and framed under an erroneous assumption that then paints how the reader will read the work. With the overarching claim that the texts such as \textit{Incidents} and \textit{Twelve Years} represent fiction, these slave narratives have their receptions altered in a way the discredits their experiences and stories by implying these works possess fantastic or unreal qualities. At the same time, their inclusion possesses the possibility of creating an errant scholar whose reading of those texts pulls them out of their historical context, or supplants autobiographical content and lived experience with aesthetic and literary motives. Cases such as Jacobs and Northup are easy to determine were this happening, given their recent prominence in literary studies. It is for the titles that have not yet received the same consideration as these that are in vulnerable positions. 

Several other oddities exist within the listings of Wright beyond the omissions and the gaps they produce. For example, an American edition of Charles Dickens'  \textit{The Mystery of Edwin Drood} is described; this edition of a work Dickens did not complete before he died in 1870 was completed by an American, Thomas Power James. According to Wright, the completion of \textit{Edwin Drood} by an American warrants inclusion into his collection of American fiction.\autocite[422]{wright_american_1957} And while Wright endeavored to exclude juveniles, and does in the case of Alcott, he decides to include Twain's \textit{Tom Sawyer} (1876) and \textit{Huckleberry Finn} (1885), texts that were marketed as juveniles when they were published.\autocite[109, 111]{wright_american_1966} Attempting to understand why Wright made such decisions can be difficult, especially should one try to investigate each and every unique case. For many of the odd exclusions and inclusions no published justification exists, and in general, it is only stated by Wright that he intended to exclude certain genres or forms of fiction from his bibliography. However, what should be clear is that this process of producing a resource for the study of American literature involved questions that, while at face value may seem objective, involves choices by the bibliographer, who is human, fallible, but also always interpreting. Wright's work, as with any bibliography, should be seen as a helpful resource, but not necessarily as immutable, as Wright himself would show in the numerous revisions he makes to his work, as well as with the subject of the next chapter, the friends and institutions that would help him in the process of creating \textit{American Fiction}.