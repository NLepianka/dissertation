%%%%%%%%%%%%%%%%%%%%%%%%%%%%%%%%%%%%%%%%%%%%%%%%%%%
%
%  New template code for TAMU Theses and Dissertations starting Fall 2016.  
%
%
%  Author: Sean Zachary Roberson
%  Version 3.17.09
%  Last Updated: 9/21/2017
%
%%%%%%%%%%%%%%%%%%%%%%%%%%%%%%%%%%%%%%%%%%%%%%%%%%%

%%%%%%%%%%%%%%%%%%%%%%%%%%%%%%%%%%%%%%%%%%%%%%%%%%%%%%%%%%%%%%%%%%%%%%
%%                           NOMENCLATURE
%%%%%%%%%%%%%%%%%%%%%%%%%%%%%%%%%%%%%%%%%%%%%%%%%%%%%%%%%%%%%%%%%%%%%

\chapter*{NOMENCLATURE}
\addcontentsline{toc}{chapter}{NOMENCLATURE}  % Needs to be set to part, so the TOC doesnt add 'CHAPTER ' prefix in the TOC.

%A note about aligning: These entries will align
%themselves according to the ampersand (&).
%No extra spaces are needed, as seen in some of
%the entries below.

\indent Below are a few frequently used terms and their definitions as they are understood in this dissertation. These terms are primarily drawn from and informed by the field of textual scholarship, which prescribes a very narrow sense to some of these terms. 

%Example of the longtable environment.
\hspace*{-1.25in}
\vspace{12pt}
\begin{spacing}{1.0}
	\begin{longtable}[htbp]{@{}p{0.35\textwidth} p{0.62\textwidth}@{}}
	   % \begin{tabular}{@{}p{0.33\textwidth} p{0.62\textwidth}@{}}
		Bibliography	&	This term can be used in two different senses and will be used in both within this dissertation. The first sense is as the name of a field and the systematic study of books as objects and their production. The second refers to the organized arrangement of bibliographical descriptions, generally in print, but not necessarily.\\	[2ex]
		Enumerative Bibliography		&	A precise term used to identify bibliographies that privilege listing of works rather than in-depth descriptions. Typically, these resources testify to a large number of materials. This genre of bibliography is referred to as enumerative because of the tendency to number the entries, but this is not required.\\	[2ex] %[2ex] provides double space between each row
		Wright I, II, III			&	The standard for how to reference the three volumes of Wright's \textit{American Fiction} bibliography is by use of Wright's name with a Roman numeral, I-III, that corresponds to the chronological volume Wright published. Thus, \textit{American Fiction, 1775-1850} is referred to as Wright I, \textit{American Fiction, 1851-1875} is Wright II, and \textit{American Fiction 1876-1900} is Wright III. This dissertation will employ the same standard for referencing \textit{American Fiction}.\\	[2ex]
		Description & The formal notation of bibliographical elements is generally referred as a \textit{bibliographical description}, or simply a \textit{description}. Likewise, the process of a bibliographer creating a description is referred to as \textit{describing}. Descriptions appear similar to citations commonly found in academic publications, in that they often record the same data, but more advanced bibliographical descriptions will include elements that not seen in citations, such as collation formulas, information about illustrations, typesetting, ornamentation, and binding.\\ [2ex]
		Work & The concept of a specific piece of literary writing. For example, Harriet Beecher Stowe's \textit{Uncle Tom's Cabin}. It should be noted as a distinctly different term from that of \textit{Text} in that is unattached to a specific physical manifestation of writing.\\ [2ex]
		Text & The physical incarnation of a work. The term will be used to refer to specific editions or copies of a work. For example, the 1853 John P. Jewett Illustrated Edition of \textit{Uncle Tom's Cabin}. \\ [2ex]
		%XXXXXXXX		&	This is an optional page. Random word to test how long the sentence can be? This is just for test purpose. The current setting aims to align left/right margin same as all other pages.\\	[2ex]
	   % \end{tabular}%
	\end{longtable}
\end{spacing}

\pagebreak{}