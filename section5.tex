%%%%%%%%%%%%%%%%%%%%%%%%%%%%%%%%%%%%%%%%%%%%%%%%%%%
%
%  New template code for TAMU Theses and Dissertations starting Fall 2016.  
%
%
%  Author: Sean Zachary Roberson
%  Version 3.17.09
%  Last Updated: 9/21/2017
%
%%%%%%%%%%%%%%%%%%%%%%%%%%%%%%%%%%%%%%%%%%%%%%%%%%%
%%%%%%%%%%%%%%%%%%%%%%%%%%%%%%%%%%%%%%%%%%%%%%%%%%%%%%%%%%%%%%%%%%%%%%
%%                           SECTION V
%%%%%%%%%%%%%%%%%%%%%%%%%%%%%%%%%%%%%%%%%%%%%%%%%%%%%%%%%%%%%%%%%%%%%

\chapter{CONCLUSION: THE "TRANSFORMISSION" OF WRIGHT AND \textit{AMERICAN FICTION}}

In 1966, Lyle Wright completed the third volume of \textit{American Fiction}, bringing the bibliography's range of coverage to 1900 and adding 6175 titles to \textit{American Fiction}'s corpus. In 1972, a second printing of Wright III was issued. In between Wright went back to the first volume and produced another revised edition in 1969 for the 1775-1850 range. These were the last few milestones of Wright as he neared the completion of his career as a bibliographer. By 1972, \textit{American Fiction} had amounted to three volumes with revised editions for the first two volumes: Wright I had three different editions (1938, 1948, and 1969), Wright II had two (1957 and 1965), and Wright III had two printings (1966 and 1972).\footnote{For Wright II, the additions and corrections made were published separately as well in 1965.} Over the course of these editions, Wright made substantial changes to the substance and arrangement of his bibliography. The revised editions, naturally, listed more titles as they were discovered, but also additional author attributions and emendations, or, in the case of Wright I, the assignment of numbers to each title for ease of reference.\footnote{The 1938 edition of Wright I was not enumerated, and thus relied purely on alphabetical order as its primary reference mechanism. The addition of enumeration for the revised edition carried over into Wright II and III's initial editions.} For Wright, it seemed that \textit{American Fiction} was never truly complete; in a talk delivered in 1966, after the publication of Wright III, Lyle Wright jokes that "retirement has provided a ready answer to any question of a volume four."\autocite[31]{wright_pursuit_1966} The acknowledgment that one could continue his work, as he indicates, and as he demonstrates through the revisions he made to previous volumes, shows how \textit{American Fiction} is as much a text as the narratives it lists in that it is always susceptible to editorial changes that alter the data these volumes retained.

It is unsurprising, after having completed such a fundamental resource for the study of American literature, that Wright's bibliography became the basis for others to use in developing further tools and resources for the study American fiction. Several commercial and academic groups have used Wright as the basis for their own work and their use of him demonstrates the ways in which Wright has been read by those who further his work, and how that in turn affects the way his work is read by others. As a text that can be adopted by others and changed to fit their own initiatives and desires for what the data can do for researchers, \textit{American Fiction} has been edited and its components either expanded upon, altered, or otherwise made different from the original way they were presented. Both commercial institutions such as Gale Cengage and Proquest, as well as academic ones including the University of Indiana Libraries and the University of Virginia, have taken up Wright as a guiding force in creating or curating collections of American fiction. Understanding the goals and interpretive claims made by these alterations to Wright's data reveals to us how data moves among people and institutions and the ways in which data are not only constructed but restructured to fulfill a new purpose, even in cases where that purpose may be at odds with the ideals of the compositor. 

This chapter explores the ways in which various adaptations of Wright have transmitted the data of \textit{American Fiction}. The original way in which Wright captured and presented his titles, "parameterized" to use Drucker's term, has been acknowledged but is not immutable when it comes to designing a corpus of American fiction that is intended to serve researchers and readers in a way that is different from that of an enumerative bibliography. Looking at the ways both commercial and academic institutions have repurposed Wright, using his data as a fundamental building block of their initiative, can demonstrate to us the ways in which datasets are both read and edited by others as a holistic text. 

Randall McLeod has argued that instances of transmission are always transformations as well, as that which contains and presents the text undergoes a change even if the text does not (though, of course, the text usually does as well). As McLeod says, "[a text's] structural redundancies are, crucially, both \textit{of} its text and \textit{about} it. Any lapse in them opens up contradiction at the heart of transmission" (emphasis McLeod's). McLeod coins the portmanteau "transformission" to encapsulate his idea.\autocite[246]{randall_mcleod_published_under_random_clod_information_1991}
Understanding transformission as a guiding framework, we can then understand that all attempts by others to use Wright's work inherently "transformits" the data, imbuing new meaning and contexts unto the data. McLeod's stance is embedded in the framework of the social text as pioneered by D. F. McKenzie and Jerome McGann, and the concept of "transformission" understands that individual actors upon the text will inform its transformation as these actors place the text into a new context and form. Bibliographies are no different than the dramatic and poetic texts McLeod analyzes, as a text such as a bibliography relies on a rigid structure to inform its arrangement, and therefore changes to that structure inherently affect the data itself and what can be taken from that data. Appending facsimiles of the enumerated texts or omitting a text for any reason suggests, at the least, a counterargument or desire for revision for the original work. The transformations that are undergone by the data of Wright's bibliography open up new readings of that data and make more clear the ways in which those who took \textit{American Fiction} as a basis for the creation of new collections of data viewed a collection of American fiction differently from the way that Wright conceived of it in his list. 


\section{Wright on Wright}

As a first step to understanding how the editing and adapting of Wright's work to other forms present interpretive judgments about texts, it is necessary to understand Wright's own approach to his work. Identifying the framework he employed in compiling his multiple volumes of American fiction helps to reveal some of the implicit arguments a work such as \textit{American Fiction} presents to its readers. Wright was a full-time employee at the Huntington Library for most of his career; he did, however, publish a few pieces of scholarship that were informed by his research and production of \textit{American Fiction}. His first essay, "A Statistical Survey of American Fiction, 1774-1850" displays a few key ideas the undergird the first volume of \textit{American Fiction}'s argument. The first of these ideas is the concern for contemporary taste that informs the population of his bibliography, rather than the designation of status and cultural value. As he begins in one section of his analysis: "The writings of many of the forgotten authors, true enough, may not be literary masterpieces, but the point so often overlooked is the contemporary taste for such literature" \autocite[312]{wright_statistical_1939}. Key early American writers, in Wright's terms, are Timothy Shay Arthur and Joseph Holt Ingraham, both of whom published prolifically between 1830 and 1850, with Arthur publishing 50 works in that time and Ingraham producing 79 works. These are not canonical authors, but authors who were successful in the marketplace, in producing, publishing, and seemingly in selling. What Wright recognizes is that their ability to fill out his list so substantially is a byproduct of that success and not their literary value. 

The question of literary value is antithetical to \textit{American Fiction}, and this stance dovetails into the second of Wright's ideals: the ability for a list of works to potentially (though not realistically) include everything. In his essays, Wright demonstrates concern for how those he calls "literary historians" treat the wider world of American publishing beyond the canon. At the conclusion of his discussion of Arthur and Ingraham, Wright asks, "How many literary histories even mention their names?"\autocite[312]{wright_statistical_1939} This sort of subtle jab at literary scholars is characteristic of Wright in his other writings.\footnote{Wright's primary term to reference those who study literature is "literary historian" and he, to my knowledge, does not deviate from that term. I interpret his usage of the term as broadly applicable to not just those interested in literary history, but also literary criticism since his idea of their work includes not just historical scholarship but interpretive and analytic scholarship of literature.} In an essay derived from his work on the second volume of \textit{American Fiction}, Wright states, "Literary historians will say, I am sure, that some of these titles were better forgotten, but that is a bibliographical impossibility." \autocite[75]{wright_few_1955} This statement erects a dividing line between Wright's conception of literary scholarship and bibliography. Bibliography is meant to compile and bear witness to all the print publishing that it can; to Wright, bibliography is perfectly egalitarian with its ideal of proclaiming everything worthy of being listed. Literary scholarship, on the other hand, is exclusionary. 

One of the arguments we might then assume about the three volumes of \textit{American Fiction} is then in the ability for the list to testify as to the abundance of American fiction in any capacity, rather than just the aesthetic and cultural value of the items listed. This appeals to the natural affordance of bibliographies and why they exist in the first place: they provide easy reference to a large amount of information. But we can also view this as a ideological principal that guides a reader's approach the subject of American fiction. Wright's view of the topic of American fiction is more aware of market success, contemporary taste, and those who numerically contributed to the total sum of the American literary tradition, rather than to the canonical figures whose presence fills out Wright's bibliography to a lesser degree than the noncanonical works. The information that \textit{American Fiction} provides, as Wright implies, is useful not for the possible literary merit that could be discovered, but because of what it will provide access to the holistic knowledge of American culture. As Wright observes, the noncanonical works he presents are useful beyond the question of literary merit:
\begin{displayquote}
From these tales of varying degree of literary merit, and I do not consider all of them literary outcasts, a great deal can be learned about the way of life of the people, the clothes they wore, the food they ate, and their daily gossip. \autocite[77]{wright_few_1955}
\end{displayquote}
The culture of early America as displayed by its fiction is not necessarily congruent with the way that literary history has been constructed by scholars, but a reference work that makes more accessible the ability to see beyond the mainstay titles of American literature can help to alleviate that problem. Wright's position as the compiler of American fiction titles places him outside of the position of the literary historian who closely reads a small subset of those titles. 

From his vantage point of being able to, by the end of his career, view over 10,000 titles of American fiction, Wright's readings of literature come from what could be referred to as a "distant" view.\footnote{To clarify, Wright's method and point of view here resonates with contemporary digital humanities scholars. Franco Moretti, in \textit{Graphs, Maps, Trees} argues that distant reading brings awareness to a level of literary scholarship that was previously ignored or inaccessible; that is, the cycle, a temporary structure, smaller than the span of centuries but larger than the individual text, two chronological levels literary scholars are more familiar with. See \autocite[13-4]{moretti_graphs_2005}. Similarly, Matthew Jockers argues for a macroscopic, or distant, view via quantitative evidence to in turn help both understand the contexts literary texts emerge from and to better understand specific literary texts. See \autocite[26-7]{jockers_macroanalysis:_2013}.} As evident from the titles, "A Statistical Survey of American Fiction" and "A Few Observations on American Fiction," it is evident that Wright is interested in what his bibliography can holistically tell scholars about American fiction. In his "Statistical Survey," Wright catalogs the number of separately published titles per decade from 1770 to 1850, totaling 1377 individual titles. Wright then further classifies the texts based on a genre-designating term found on the title-page. In this he asserts the popularity of the term of "novel", "tale", and "romance" as they emerge in early America, encapsulating a full third of the 1377 titles Wright had counted.\autocite[309-11]{wright_statistical_1939} This example is interesting in both the claim it makes, but also in how Wright has chosen to make his argument an extension of his bibliographic work, which necessitated the accumulation of a large quantity of data. Wright compares his own statistical work with that of the literary historian's narrow view of literary history, as the primary thesis of his statistical work claims: "An analysis of all the titles yields results that vary considerably from those obtained after an examination of a select few."\autocite[309]{wright_statistical_1939}

Similarly, though less mathematically, Wright's "Observations" and "In Pursuit of American Fiction," both contribute to large, overarching judgments and generalizations about American literary culture based on his experience in compiling \textit{American Fiction.} In "Observations" Wright holds a similar stance as he is explaining his work in compiling \textit{American Fiction}'s second volume. With a resource such as \textit{American Fiction} on hand, Wright explains, "the scholar has a much broader view of the literary activity in this field during the third quarter of the nineteenth century."\autocite[75]{wright_few_1955} The subject of what is considered to have merit is irrelevant next to the cultural information that can be compiled; the trends and patterns that emerge can be, to Wright, just as informative and capable of expanding the purview of literary study. At the close of Wright's career, in his lecture "In Pursuit of American Fiction," Wright finishes by saying that, even after the expansion of American literary study in the academy, and after this own extensive work in creating a three volume listing of titles, the field is "still far from exhausted." Wright explains, "...many of the novels of minor authors have yet to be examined, even superficially, for any share they may have contributed to the enrichment of our literature."\autocite[48]{wright_pursuit_1966} 

Wright's insistence on the still-widening expanse of scholarly possibility in his published work gets to the heart of what it means to compile a bibliography. In Wright's own view, \textit{American Fiction} is a tool from which insight can be gained, and it can improve the field. Part of the construction of the bibliography is designed around facilitating research and encouraging the expanding of American literary study that Wright speaks to. As a reference work, Wright's bibliography takes upon itself to not only list works and testify, via their descriptions, that these works exist but to also facilitate access to these texts. A significant part of the descriptions Wright includes is which libraries possess a title, based on the libraries Wright surveyed and visited while he was researching and compiling the volumes. In all three volumes, each description is appended with codes of libraries where Wright was able to physically locate the texts. The census of where the works are available lends both a provenance and argument as to the work that went into the compilation of the bibliography, but also works to help the reader. For an early to mid-twentieth work, this was the most a work could do in pointing researchers in the direction of the wealth of American fiction that exists, but still endeavored to aid in the access to rare or lesser-known texts. The presence of the census reinforces the overall trajectory of Wright's work in terms of its commitment to showing the broadest cultural context of American literary writing. Reading the entries in the census itself, however, provides an addendum to the data beyond the innately bibliographical information (i.e. title-page, year of publication, etc.) that speaks to the influence of certain titles in terms of their physical location in various academic libraries. Unsurprisingly, a first edition of a title such as \textit{Moby-Dick} (1851) was found in thirteen of the nineteen collections Wright perused. Titles with less or even no cultural and institutional backing appear in sometimes only one library. 

The issue of access becomes central to the way that Wright's work is used and re-imagined in subsequent generations of scholarship and projects that use Wright's work as a foundation. Wright's own work was a significant undertaking that has yet to be replicated or redone, even with the expansion of multiple libraries and the arrival of digital repositories and databases that might turn up new discoveries that fit Wright's definition of "American Fiction." Rather than a vertical expansion of Wright's list to increase the number of titles between 1776 to 1900, the transmission of Wright instead relied on a more horizontal expansion, taking the titles and furthering the data and information associated with those already listed by Wright. The most obvious and primary way of doing this was to make more accessible the texts of the titles Wright enumerated, particularly by finding, scanning, and marketing a collection of images of the texts via microfilm in the 60s and 70s, and then in digital form by the twenty-first century. When the data that Wright had presented to the world moved from his hands into the hand of others interested in the possibilities \textit{American Fiction} presented, Wright's data became the subject of the inevitable changes that occurs to any text in the process of transmission. 

\section{Research Publications and Microfilm}
It was not long after the publication of \textit{American Fiction} that it was used to facilitate the process of creating research materials for scholars. As a bibliography, \textit{American Fiction} was a reference tool that attested to the existence and location of titles, but did not provide the text of those titles. Libraries and archives, such as the \textit{American Antiquarian Society} as discussed in a previous chapter, would use Wright as an aid in assembling materials and expanding their collections, but a formidable wall still existed that hindered access to some of the more rare materials Wright listed. A company known as Research Publications undertook the task of creating a microfilm collection of every title Wright listed.\footnote{Research Publications is now known as Primary Source Media and operates as a branch of Gale Cengage.} By 1974, the task had been completed and a set of microfilms that provided facsimiles of 10,827 titles was available for purchase for institutions interested in having the texts listed by Wright for considerably less effort than acquiring physical copies of every text. In addition to the microfilms, Research Publications published a cumulative index of the texts, "in keeping with [their] policy of providing the best possible bibliographic tools for use" \autocite[n.p.]{wright_american_1974}. 

Research Publications' project represents a significant textual moment for \textit{American Fiction} because it demonstrates one of the first moments of transmission to a different format. This transmission of Wright's work from a bibliography to a microfilm collection incorporated Wright and American fiction into library settings in a way that had not been feasible before. The packaging of the entirety of the unique titles listed by Wright into a single product meant drastically improving a library's American fiction holdings upon acquisition of these microfilms, especially for libraries that could not obtain first editions or copies of rare titles. The process was not without its own modifications to Wright's work that affected the way one would approach the Wright titles. As a physical collection of microfilms, the medium afforded the ability to approach individual texts, but not to look holistically at the entire corpus without significant effort. Comparison among texts, even those by the same author, would involve changing multiple reels to view them, or at worst, using multiple machines.\footnote{It is common now for microfilm machines to have the ability to scan the microfilm images to save them as a .pdf or .jpeg file, but in the 60s and 70s, this ability would have been far more limited in allowing one to approach multiple images.} While the microfilm collection gave scholars a means to access the texts they could not before, it did not as easily allow for the holistic view that Wright saw as potential in composing \textit{American Fiction}. This fact is further complicated when it came to how Research Publications treated description and inclusion into their microfilm collection and the differences between their corpus and Wright's.

The microfilm project was not a complete replication of the bibliography in itself. The focus was on the facsimiles of the text rather than the descriptions. The microfilm did not need advanced bibliographical information to help scholars reference or locate the text, because they were supplying the text itself. The printed index to the microfilm (a monograph in itself) presented only the slimmest information in reference to the titles, rather than Wright's title-page descriptions or complete bibliographical descriptions. Any publication information available upon viewing the scanned page images that related to the bibliographic data was still available, but any information that required research beyond what was overtly visible (i.e. the census information, related publications, reprintings, etc.) was not included. As well the index did not follow Wright's structure for \textit{American Fiction}, that is, the chronological organization. The concept of the distinct volumes, demarcated by time period, was discarded and instead the titles were arranged alphabetically as a whole, with their volumes and Wright number therein indicated. As an example, let us examine the below text, a representation of the index entry for Susanna Rowson's \textit{Charlotte Temple} (1794):\footnote{The microfilm's texts and the data associated with them were derived from the 1948 revised edition of Wright I, the 1965 revised edition of Wright II, and Wright III's 1966 copy.}

\begin{displayquote}
ROWSON, Susanna (Haswell). Charlotte. 2d ed. Philadelphia, Printed for M. Carey, 1794. I.2159, Reel R-5 \footnote{\autocite[318]{wright_american_1974}. The second edition of \textit{Charlotte Temple} was published with the more simple title \textit{Charlotte}.}
\end{displayquote}

The codes at the end of the index reference represent, first, the Wright volume number and entry number (thus, \textit{Charlotte Temple} is to be found in the first volume, as entry number 2159. This assumes the 2nd edition of Wright I, since the first was not enumerated. The second code refers to the microfilm collection's own reel. Any additional bibliographical information helpfully points to the exact place in Wright where a scholar could find it, and helps to locate the place in the microfilm for the scholar to find the text, but it is not a full description. This is not the chief point of interest in the microfilm's collection of texts versus Wright's bibliography, however. The instance of \textit{Charlotte Temple} demonstrates an important condition of the microfilm collection's creation. The index only lists one instance of \textit{Charlotte Temple}, and the microfilm contains only one facsimile for the text. That is, the second edition as it clearly denotes in the index. Wright I, however, enumerates 82 different editions of the \textit{Charlotte Temple} published in America alone, all with unique bibliographical descriptions that point to the fact that the 82 entries are considered by Wright distinctly different texts.\autocite[232-238]{wright_american_1948} These descriptions point to different years of publication, publishers, printings, etc. of \textit{Charlotte Temple} that distinguishes them from one another. The Research Publications' microfilm, however, omits all other entries in favor of simply supplying one text for reference. This choice suggests a reasoning that helps us understand what the goals and aims of the microfilm collection are in comparison with the Wright bibliography. The numerous editions of \textit{Charlotte Temple} are of interest to collectors, librarians, editors, and bibliographers interested in knowing the publication history of the text, but the literary critic may not care or know to care about the importance. Instead, one copy of the text should suffice, lest it suddenly become an overbearing labor to wade through over a hundred different scans of the same novel. The cost of such work would also present an appealing incentive to discard the idea of scanning entries with multiple editions listed.\footnote{\textit{Charlotte Temple} is not the only one to be treated this way. Other notable entries would include James Fenimore Cooper's various novels, which also have quite a few entries per novel. Other authors are not as widely reprinted as these two and may only have an additional one or two editions that Wright also lists.}

\begin{figure}
\includegraphics[scale=.23]{charlotte}
\caption{The cover page of the second edition of \textit{Charlotte Temple} as found on the Research Publications "American Fiction" microfilms, reel R-5.}
\end{figure}

This sort of change demonstrates where the most important additions to Wright's work could be made, and to whom the work was intended to help. Those interested in comparing editions of \textit{Charlotte Temple} gain little, unless they had something other than the second edition available already and sought out the microfilm collection. The microfilm itself, while alluding to Wright, erases the work he (and others) have done in compiling the list, as well as hides the expansive publication history of the novel, barring access to one pivotal piece of information and the culture in which the text of \textit{Charlotte Temple} is circulating--a culture which could not get enough of the novel for decades--while simultaneously attempting to enable access to the text in order to provide researchers and readers with a firsthand artifact that observed that same culture.

While the microfilm collection does cut what it sees as excess editions from Wright's list for the purpose of providing access, the collection does include some repetition. Such as Hawthorne's \textit{Scarlet Letter}, listed three times in the 1948 edition of Wright I (nos. 1146-8). Two editions of \textit{The Scarlet Letter} were scanned for microfilm--the first and second edition, both printed in 1850, a rare moment when one work has multiple copies present in the collection. It is likely no mistake that one of American literature's most canonical novels received this treatment. Wright's own notes in his bibliography give away the fact that the \textit{Scarlet Letter} has received closer attention than many other texts. Wright appends the following note under the first edition's description: "Page 21, line 20, 'reduplicate.' For other differences between this entry and the following one, see Cathcart."\footnote{Wright refers here to W. H. Cathcart's \textit{Bibliography of the Works of Nathaniel Hawthorne, published in 1905.}} For the second edition, he notes for page 21, line 20, "repudiate."\autocite[122]{wright_american_1948} Noting the variants between the two texts, as well as the reference to the more in-depth work of Cathcart's work with Hawthorne, Wright signals to the wider context of Hawthorne's work, and declares there to be a literally noteworthy mention of the changes between the two \textit{Scarlet Letters}. In addition to lexical variants, the second edition appends an introduction by Hawthorne to the text, which is viewable on the microfilm. 

\begin{figure}
\includegraphics[width=\textwidth,keepaspectratio]{scarlet}
\caption{The back of the cover page and the first page of Hawthorne's preface found in the second edition of \textit{The Scarlet Letter} as found on the Research Publications "American Fiction" microfilms, reel H-6.}
\end{figure}

The presence of the two \textit{Scarlet Letters} is redundant if the logic of the cases of \textit{Charlotte Temple} and others are considered, as the extensive number of editions of \textit{Charlotte Temple} would of course have variants just as Hawthorne's romance would.\footnote{Robert G. Vail's descriptive bibliography of Susanna Rowson's works attests to this. See \autocite{vail_susanna_1933}} But the decision as to why the space, labor, and resources would be spent on the two editions of the \textit{Scarlet Letter} brings attention to the novel's canonicity, and implicitly argues that while multiple copies of \textit{Charlotte Temple} may not be worth viewing, or worth the labor, two copies of the \textit{Scarlet Letter} certainly are. Whether it is due to the the fact that there are significantly fewer editions of the \textit{Scarlet Letter} than of \textit{Charlotte Temple}, or the fact that Wright's notes directly draw attention to variations among the editions, whereas the notes for \textit{Charlotte} do not, the microfilm collection does value Hawthorne's work more than that of Rowson's. By the 1970s, Rowson had become a woman writer in need of recovery, despite her popularity in early America. Cathy Davidson explains in her introduction to the Oxford University Press edition of the text: "After World War I, tastes changed and \textit{Charlotte Temple} lost its popular appeal. And, later, academic reading habits shifted in response to the emerging New Critics who focused on, say, the levels of ambiguity in \textit{Moby Dick} or \textit{The Scarlet Letter}..."\footnote{\autocite[xxxii]{rowson_charlotte_1986}. In 1964, Clara M. and Rudolf Kirk's edition with College and University Press was published in order to help bring \textit{Charlotte Temple} back into classrooms. Davidson's 1986 edition was printed for Oxford University Press' Early American Women Writers series, which provided four titles in an effort to help recover and reincorporate several significant works: \textit{Charlotte Temple}, Tabitha Gilman Tenney's \textit{Female Quixotism} (1801), Hannah W. Foster's \textit{The Coquette} (1797), and Catherine Maria Sedgwick's \textit{A New-England Tale} (1822).} Davidson's point here provides another explanation for the curious composition of the microfilm collection in that it provides a means of understanding how the microfilms were scanned in response to academic trends. The microfilms were meant for research institutions and libraries, and since the culture of the academy was invested in New Critical work (though also in transition to post-structuralism by the time the scanning was complete), then the resources would be more marketable and conceivably useful if they responded to the increased focus on more often studied texts such as the \textit{The Scarlet Letter}. Following this point, we can see further how Research Publications' aims with their microfilms differs from that of Wright's and his ideology. In pursuing access to the texts of Wright, the collection also considers what texts are most likely to be accessed, anticipating that \textit{The Scarlet Letter} would be subject to more traffic than most of the other texts, and attempts to meet the increased demand with a perceived idea of what scholars may desire. 

While Research Publications endeavored to scan and present every title that Wright listed, the status of a given text inflected the nature of the construction of the collection. In appearing to privilege specific texts above others, the microfilm collection reveals the way it has adapted Wright's bibliography to suit the purpose of its producers. Within the context of Wright I, all of the texts, whether or not they are a version of a single work, are worth including. A researcher looking at \textit{The Scarlet Letter} on this microfilm collection is simply treated to more information than a researcher looking at \textit{Charlotte Temple}, despite the fact that the publication history for the latter is more extensive. The logic of the corpus of American fiction as a whole, as represented in its scanned collection, demonstrates an understanding of the canon and the subjective and differential treatment texts within and outside of that canon. In attempting to translate a bibliography constructed with a sort of egalitarian logic in mind (at least in theory if not in practice) to a form for scholars to access the texts, the transmission has been affected by the exclusionary logic of literary history. Such a model for translating Wright's corpus to other forms beyond an enumerative bibliography inevitably come to similar conclusions in practice. Even after the substantial recovery work for marginalized and forgotten authors that began to occur in the 1970s, collections based on \textit{American Fiction} would continue to privilege certain texts above others in the same collection. 

\section{The Wright American Fiction Project}
Moving from microfilm to the emergence of the web and digital technologies, Wright's work saw continued use when Indiana University introduced the \textit{Wright American Fiction Project}. The project gives scholars a repository of both .pdf facsimiles of the Wright titles found in volume two (1851-1875) as well as some XML-encoded text versions alongside some of the original metadata Wright initially described.\footnote{XML, or Extensible Markup Language, is a manner of appending information to digital text to inform machine learning or reading of the information. In the context of digitized texts, the Text Encoding Initiative (TEI) is the institution that publishes the standards for encoding information. The TEI guidelines largely prescribe methods for marking bibliographic information (page numbers, chapters, paragraphs, etc. but some semantic guidelines are also present. XML is largely used to aid in searching texts and extracting or otherwise organizing textual information and is not commonly engaged with by typical readers.} In one way, the project continues with some of the same ideals that Research Publication's microfilm collection did, but as well, the project updated the concept to fit with both modern scholars' expectations and the affordances of the digital environment. Moreso than the microfilm copies of the texts presented, the \textit{Wright American Fiction Project} is a task of editing and demonstrates an interpretation of the items that compose the project as a whole by means of presenting texts in a way that the project's leads assume scholars will find useful. 

To begin, the project was pushed with only the second volume of Wright viewable. While the other two volumes are labeled as eventual goals, the project is seemingly dormant, with no major updates since it released the digital texts of the Wright II titles. The choice of Wright II is significant as a first choice for providing digital records because it coincides with literary movements and historical moments that emerged after Wright's time. Wright II encapsulates F. O. Matthiessen's \textit{American Renaissance}, the publication of \textit{Uncle Tom's Cabin} and its imitators and contractors, westward expansion and imperialism, and of course the Civil War. On the subject of F. O. Matthiessen, his concept of the American Renaissance defined much of the way American literary scholars conceptualized and taught American literary history; Matthiessen's \textit{American Renaissance}is one of the primary influences for the centrality of Melville, Hawthorne, Emerson, and Thoreau in the American literary canon. Wright published Wright II in 1949, 8 years after \textit{American Renaissance} was published (1841), but Wright does not make any explicit mention of Matthiessen in \textit{American Fiction}, so it is indeterminent whether or not he was aware of or familiar with the Matthiessen's arguments therein. Given that Wright II follows naturally from Wright I, beginning in 1851, and ends in 1875, a date not particularly relevant to Matthiessen's thesis, the overlap is certainly coincidental from a composition perspective, but nonetheless significant from the perspective of readers and researchers. To prioritize Wright II is to prioritize the likes of Melville, Stowe, and Twain; the project is not shy about this as its website's home page shows portraits of the aforementioned authors alongside Bret Harte, William Dean Howells, and Nathaniel Hawthorne that are hyperlinked to search results of these authors' works. The project itself understands and anticipates who the most prominent members of its corpus are and what texts will be the biggest hits, so to speak. This is in direct conflict with Wright's ideal of the egalitarianism of bibliography. The issue is only further compounded when one looks at the state of more popular and canonical texts when compared to those that are virtually unknown. 

The \textit{Wright American Fiction Project} provides XML-encoded files of the texts enumerated by Wright II in addition to .pdf and plain-text formats. Details about the status of the collection's encoding can be found in the site's "Encoding Overview," which reveals a division in the way certain texts were treated:
\begin{displayquote}
The online collection of nearly 3,000 volumes consists of two different groups of texts. The larger group of approximately 1,800 electronic texts was created by Prime Recognition Optical Character Recognition (OCR) software. These texts are minimally encoded and largely unedited, and rely on the facsimile page images as the main access point. The other group of approximately 1,200 texts has been fully edited and encoded, and also includes facsimile page images. In addition to being corrected, these files allow for better document-centric navigation by identifying chapter or story divisions within each work and having a hypertext linked "Table of Contents." Both groups of texts are available for bibliographic and full-text searching as well as browsing.\autocite{noauthor_encoding_nodate}\footnote{In addition to these two collections, the encoding has changed, from the original SGML to the various iterations of the TEI guidlines P3 through P5 in order to keep in it in line with standards and best practices at the time.}
\end{displayquote}
The choices of which texts belong to the "largely unedited" texts and the group of edited and encoded is not entirely random. The selection process for the more robustly edited and encoded texts is largely opaque to the viewer, but some of the choices are obvious as to why they were chosen for advanced encoding. Again, familiar names pop up as to which texts are beneficiaries of the resources and labor required for electronic editing. 

Let us compare \textit{Moby-Dick} to the text that is found directly after it in \textit{American Fiction}, Matthew Merchant's \textit{How Bennie Did It} (1869). For this comparison, I would like to begin with the text of Merchant's work and the associated XML the \textit{Wright American Fiction Project} presents to readers who may encounter this text. As the project's "Encoding Overview" states, the digital text was created via OCR, rather than transcription. This process, depending on various factors including how the software was trained, the quality of the print scanned, among other contingencies, can produce errors in the text files that are created. \textit{How Bennie Did It} is no exception. Below is the opening paragraph of the body of the text and its XML encoding:

\begin{quote}
\&\#xD;<pb n="0" xml:id="VAC8367-00000005"/>   HOW BENNIE DID IT. CHAPTER L IT'S a singular story,     think you will say, before having rea it through;. Singular, however, though, it may be, we hope there are none who may read it but will do so with both pleasure and profit. The BENNIE STOUT Of the story was a youth, the record of whose life, peculiarly interesting and eventful as it is, might well be accepted. as more of a study than a story: for, connectedy with, or in fact giving rise to, the very features of his history which will doubtless interest us most, there is something deeper than story, and more earnest than entertaing. ment. It will not only be the youthful reader,  ho will stop, and wonder, and probably aask met   twl  o  nyb teyuhu    edr   \&\#xD;<pb n="8-9" 
xml:id="VAC8367-00000006"/>\autocite{matthew_merchant_how_nodate}
\end{quote}

What becomes apparent from this section is how minimal the "minimal encoding" is. The text here is lacking in more common elements found in even lightly encoded texts. Standard TEI markup of a text would include bibliographical information that demonstrates the structure of a page, i.e. the distinction between a chapter heading and the body of the text. This chunk of text, however, is only marked up to refer to a page break, i.e. the $\langle pb \rangle$ tag which simultaneously refers to the page image that is associated with the text. The rest of the text is largely unmarked, leading to a messy chunk of text that does not delineate between the body of the text and its paratext. The all-caps "HOW BENNIE DID IT" is seen after every $\langle pb \rangle$ tag in the XML, indicating the running head of the page being scanned and picked up by the OCR, but its physical location on the page and the meaning inherent to that being lost. A similar phenomena occurs with the "CHAPTER L," which is both an OCR error produced by reading an I with a full stop as an L, and a phrase that has been dissociated from its bibliographic function as what demarcates separate textual sections.\footnote{XML and the TEI guidelines specifically makes space for these textual characteristics to be easily marked. Chapter headings and running heads for pages are both standard tags included in TEI's P5 guidelines.} The end of the quoted passage additionally shows some of the difficulties that come with OCR. The final words of the page itself are "probably ask" before it continues the sentence on the next page. However, the OCR, and thus the XML file and text visible on the project's site, include the gibberish "met   twl  o  nyb teyuhu    edr." No words follow the "probably ask" of the page image, and so the machine has had text introduced to it that is not apparent in its source, and the project has pushed this text to the public with these additions. 

\textit{How Bennie Did It} has been updated or revised if the changelogs of the XML file are to be believed, but still the errors found within the first paragraph are present. The level of effort given to Merchant's text pales in comparison to that of \textit{Moby-Dick}. Viewers of the \textit{Wright American Fiction Project} version of \textit{Moby-Dick} would a find a document that is heavily encoded: its table of contents is hyperlinked to the body of the text, the initial "Etymologies" section is formatted as a table and presented in an organized fashion that resembles the same way it is presented in print, and the body of the work is arranged as much as one would expect a novel to be in XML. But of course what makes \textit{Moby-Dick} a prime choice for comparison here is not just its canonicity, but the way it shifts into other modes of presentation periodically, such as the "Midnight, Forecastle" chapter, wherein the text is set as if it were a stageplay, or the "Cetology" chapter that replicates a geneological catalog. These moments of the text reveal the effort required and given to this particular text in order to present a satisfactory digital edition of the work. In the "Midnight, Forecastle" chapter, the initial lines of the dramatic performance belong to the 1st Nantucket Sailor, whose words are not only presented as dialogue but also contain quoted verse. 

\begin{displayquote}
<sp>
     
     <speaker>1ST NANTUCKET SAILOR.</speaker>
     
     <p>Oh, boys, don't be sentimental; it's bad 
     for the digestion! Take a tonic,
     	follow me! <stage>(Sings, and all follow.)</stage>
        
        <q>
        	
        	<lg type="quotation">

              <l>Our captain stood upon the deck,</l>

              <l rend="ti-1">A spy-glass in his hand,</l>

              <l>A viewing of those gallant whales</l>

              <l rend="ti-1">That blew at every strand.</l>

              <l>Oh, your tubs in your boats, my boys,</l>

              <l rend="ti-1">And by your braces stand,</l>

              <l>And we'll have one of those fine whales,</l>

              <l rend="ti-1">Hand, boys, over hand!</l>

              <l>So, be cheery, my lads! may your hearts 
              never fail!</l>

              <l>While the bold harpooneer is striking 
              the whale!</l>

             </lg>

          </q>

       </p>

	   </sp>

<sp> \autocite{herman_melville_moby_nodate}
\end{displayquote}

Compared to the encoding of \textit{How Bennie Did It}, the words of the 1st Nantucket Sailor show more than just a minimal level of encoding, but, in fact, close attention to reading the text. This level of encoding goes beyond presenting just a page break, but the representation of lines, line groups, and the speaker, with both the text that signifies the subject and the words spoken marked with their own distinct tags. The text of \textit{Moby-Dick} is much cleaner compared to \textit{How Bennie Did It}. The readability of the text is improved both by the advanced markup applied to \textit{Moby-Dick} as well as the proofreading the XML file reveals the text to have undergone.\footnote{Specifically, the XML claims as a change that occurred July 31, 2003: "Finished final proofreading." This was done by Maggie Hermes. The XML file of \textit{How Bennie Did It} does not include a similar note.} This more significant level of attention given to the text of \textit{Moby-Dick} marks the difference between the two groups of the text that the \textit{Wright American Fiction Project} describes, and thus informs us how the approaches to Wright appear to differ from Wright's vision. 

While the \textit{Wright American Fiction Project} uses Wright's initial bibliographic work to populate its database, the extended services and means of accessing those documents troubles some of initial ideals Wright had about the literary documents that composed his bibliography. The perspective that all the materials enumerated by a bibliography are equal by the virtue of their having been listed in the first place becomes more troubled when we begin to see how the steps beyond listing come into play, either through the limitation of resources or the decisions of the editor/s. For both Research Publications and the \textit{Wright American Fiction Project}, the major move these two initiatives made in expanding Wright was in connecting the bibliographic data to the material texts that data pointed to and in providing the body of the text itself, that which made the texts American fiction to be listed in the first place. In neither case was this done without affecting the titles and materials of Wright's work and thus reconstituting the Wright bibliography as a holistic text and concept. In seeking to provide access to the texts of Wright's titles, both projects made decisions to either limit or restrict other texts, whether it be the multiple editions of \textit{Charlotte Temple} or other reprinted works, or in the amount of effort put into editing and presented particular texts in digital reproductions.

\section{Wright as Dataset}
Wright's shift to the digital age is not located solely in the realm of public scholarship, but as the microfilm collection Research Publications suggests, commercial groups have taken notice of Wright's work as a viable resource from which to construct collections to be sold to academic institutions. This has continued to be true as digital collections have taken the place of media such as microfilms. In this section, I wish to discuss two digital collections created by commercial enterprises: Gale Cengage's \textit{American Fiction, 1774-1920} and ProQuest's \textit{Early American Fiction, 1789-1875} (EAF). These two different collections, which have each been informed by Wright, though to far different degrees, present a moment of comparison for how two collections of the same topic, presented in a similar mode, can vary widely in how they interpret their source. As a result, the overall effect on their collections is strongly inflected by that interpretive stance, and is not just a replication, nor even an attempt at replication, of Wright's bibliography. 

To begin, I will discuss Gale Cengage's \textit{American Fiction, 1774-1920} and the way in which they have used Wright as a guiding principle in developing a corpus of American fiction. This collection contains over 17,500 titles including those found in Wright. Those who access this collection would find, as expected, facsimiles and page images of the texts available, but text files created via OCR and an interface that provides several visualization and search tools that enable access to the texts beyond the level of simply viewing the titles. Gale Cengage advertises the collection as a " landmark digital collection" that "is based on authoritative bibliographies including Lyle H. Wright’s American Fiction: A Contribution Toward a Bibliography, widely considered the most comprehensive bibliography of American adult prose fiction of the eighteenth and nineteenth centuries, and Geoffrey D. Smith’s American Fiction, 1901-1925: A Bibliography." Gale Cengage, the company that eventually gained control of Research Publications, can be seen as expanding upon and revising the microfilm collection discussed in a previous section. However, with the movement to a digital platform as a scholarly database, comes more affordances in terms of approaching Wright's corpus as a whole, rather than the more limited microfilm. However, as made obvious by the presence of Smith and the extension of the years of coverage to 1920, this database has also attempted to expand \textit{American Fiction}'s coverage beyond its original parameters. 

The corpus of Gale Cengage's collection is iterative, in that it continues the work done by Research Publications' original microform collection, yet it does not lose sight of Wright as its foundation for both the titles list and its organizing principles. Before the collection was extended to 1920, Gale Cengage's previous version of the collection spanned to 1910 and was available not as a digital repository for American fiction titles, but as a microform collection. Their subsidiary company, Primary Source Media, formerly Research Publications, expanded upon their previous work with the microfilms by adding the titles from the \textit{Library of Congress Shelf List of American Adult Fiction} to cover the years from 1901 to 1910. Primary Source Media had a similar microform collection that spanned 1911 to 1920 that was created using the \textit{William Charvat Collection of American Fiction} at the Ohio State Libraries. It was in the move to the digital that Smith's bibliography became incorporated into the collection and the range of Wright's \textit{American Fiction} was associated with titles up to 1920.\autocite{noauthor_american_nodate}

Smith's \textit{American Fiction 1901-1925} (1997) is heavily influenced by and based on Wright. In his preface, Smith notes that his selection criteria and means of description were both modeled after Wright's, with a few exceptions. Like Wright, he excludes juveniles, folklore, periodicals and series, etc, with a preferences for "novels, novelettes, romances, short stories, tall tales, tract-like tales, allegories, and fictitious biographies and travels, in prose."\autocite[ix]{smith_american_1997} Smith does include more information in his descriptions in some cases, as he notes, to "accommodate current scholarly research." To that extent, Smith includes publisher and illustrator indices in response to the popularity of book history. Smith's bibliography, though, is imagined as a successor or continuation of Wright's work, though it remains distinguished from it through its changes to description and its choice of publisher (Cambridge instead of the Huntington Library). Gale Cengage made a calculated choice in combining the two bibliographies given Smith's proclivity for Wright's work, as this would ensure, for the most part, a cohesive set of standards for the types of fiction that would be included in the corpus. However, the process of expanding Wright is more complicated than simply adding Smith's titles to Wright's.   

As explained on the web page information about Gale Cengage's \textit{American Fiction, 1774-1920} database, the collection is more diverse in its source material. "Nearly all" of the works listed by Wright from 1774 to 1900 are included; the qualifier of "nearly" is telling but honest in that is does acknowledge where the data collection has come up short. The titles from 1901 to 1910, however, do not come from Smith's bibliography, as one may assume, however, but instead from "major American fiction collections" and the Library of Congress list of adult fiction that informed the microforms that preceded the digital version of this collection. The titles that fall between 1901 and 1910, Gale Cengage declares, adhere to Wright's parameters, but the provenance of the titles remains unclear. Only the years of 1911-1920 are drawn from Smith, whose work for the last five years that his bibliography covers are excluded for an also unstated reason, though the issues of copyright for texts published after 1920 provide a likely explanation.\footnote{\autocite{noauthor_american_nodate}. Copyright laws in the U.S. have covered texts published in 1923 and afterwards. In 2019, works published in 1923 will enter the public domain.} While there 17,500 texts available for exploring within this corpus, the composition of these texts is less certain than previous iterations of the \textit{American Fiction} corpus. Of note, however, is the fact that the statements about the text's composition, even when obfuscated, make sure to identify Wright as a guiding force of the collection, even for texts he did not himself list. The 1901-1910 texts are selected in accordance with Wright, or phrased another way, are an imagining of \textit{what Wright would have listed} should he have endeavored to craft a fourth volume of \textit{American Fiction}. At the same time, however, Gale Cengage implicitly acknowledges that Wright's work was limited in some capacity; by extending their available texts to 1920, they demonstrate that Wright's three volumes not only could, of course, be expanded upon, but that they should be in order to encapsulate and present a larger view of American fiction. Their rationale for the year 1920 is unclear, but the blurb on the page mentions World War I as a point of historical reference that provides some grounding that is seemingly less arbitrary than either Wright's 1900 or Smith's 1925 as endpoints for their bibliographies. 

Gale Cengage's digital form comes with some improvements to one's ability to examine the texts in a comparative or aggregate manner than its previous microfilm incarnation did, and to an extent, allows these functionalities more easily than Indiana's \textit{Wright American Fiction} project. In a 2016 press release about the digital \textit{American Fiction}, the company says:
\begin{displayquote}
As a part of the \textit{Gale Primary Sources} program, the content within \textit{American Fiction, 1774-1920} is fueled by technology which gives researchers the ability to cross-search with other Gale digital archives, as well as analyze results using graphing and search visualization tools. In addition, all works included in the archive are fully-indexed and full-text searchable, and the metadata and data are available to support text and data mining and digital humanities research.\autocite{noauthor_gale_2016-1}
\end{displayquote}
The company advertises several features built into their own interface that allows for digital analytic explorations of their \textit{American Fiction} corpus. Listed as "Features and Tools" is the ability to find key terms and their frequencies, visualize clusters of terms that frequently co-occur, and, most simply, the ability to search and cross-reference the entire corpus. These functions demonstrate an awareness of the shift towards cultural analytics and distant reading, on the one hand, and, on the other, shows how \textit{American Fiction} is transformed in yet another instance to fulfill a different function beyond its original role as a reference work. The collection and the interface associated with it are presented within the context of improving digital humanities work, bringing Wright's influence to bear on a field that had not been codified at the point of \textit{American Fiction}'s composition. On the other hand, given Wright's own attempts at presenting statistical but general arguments about the nature of American fiction based on his research, the tools and functions enabled by Gale Cengage align with one of the original ideas Wright suggested about his work, that is, the ability to grant a broad view of early American writing over the individual, limited readings of literary scholars.

The Gale Cengage corpus still allows individual access to specific texts, and the advertisements still use the celebrity of individual authors to appeal to scholars. For example, "The Wright bibliography offers first and hard-to-find editions of major writers including Louisa May Alcott, Harriet Beecher Stowe, Stephen Crane, Mark Twain, Henry James, and other well-known authors."\autocite{noauthor_american_nodate} Similar to Indiana's incarnation of Wright, Gale Cengage advertises these authors without offering the details, such as the absence of Alcott's most prominent works (i.e. \textit{Little Women}, \textit{Little Men}, and \textit{Jo's Boys}) from Wright's bibliography. And like the projects that came before it, the errors persists, such as the continued inclusion of Jacobs and Northup's narratives. But Gale Cengage's platform encourages more intimate exploration of individual texts even while it allows for quantitative, distant approaches, and in this way it expands upon the original affordances of the microfilm collections that preceded the digital collection. Users of the database are able to tag and append their own metadata to individual texts, enabling specific scholarly knowledge that is not considered to be of general interest, but could be relevant to one's own project. 

As a last example of Wright's continued circulation and transmission, I want to discuss one of the last major projects that have used Wright, among others, as a basis for their data. ProQuest's \textit{Early American Fiction, 1789-1875}(EAF) is a digital collection of texts created originally by the Chadwyck-Healey Group in partnership with the University of Virginia Libraries.\footnote{The Chadwyck-Healey Group created several reference databases and scholarly resources. Scholars would likely be most familiar with their LION(Literature Online) database, which ProQuest continues to provide.} The original company, Chadwyck-Healey was acquired by ProQuest in 1999, and ProQuest is now the main provider of access to this collection. Like the Gale Cengage collections, users can access facsimiles of page images and a searchable text of the included titles. And also like the Gale Cengage collection, the creation of the EAF was an iterative process. The first version of the collection, created in 2000, was \textit{Early American Fiction, 1789-1850}, twenty-five years short of its current accessible version. The newer EAF was made possible with grants by the Andrew W. Mellon Foundation that allowed for the extension of the year range and the inclusion of more titles.\autocite{noauthor_early_nodate-2}

Unlike Gale Cengage, however, Proquest's version of a digital collection based on Wright is far more limited. While Gale Cengage's corpus includes 17,500 titles, the Proquest version includes only around 730 (or about four percent of the total number found in Gale Cengage).\autocite{noauthor_early_nodate-4} Even though Wright listed approximately 10,000 titles himself, ProQuest's stance on creating a collection of American fiction is to heavily limit what is considered worthwhile to scan, make searchable, and provide access to. The editorial policies of the project provide some insight into how they arrived as such a significantly smaller number of titles:
\begin{displayquote}
Two standard bibliographies describe classic American literature: Wright's \textit{American Fiction 1774–1850} lists all works of fiction published from the first story up to 1850. \textit{The Bibliography of American Literature} (BAL) lists the original editions of the most important authors of American literature, as chosen by a committee of the Modern Language Association of America. The University of Virginia Library is fortunate to have two of the world's major collections of rare first editions of American fiction in its Barrett and Taylor collections. In these collections most of the first editions in Wright and BAL are available. In some cases the University of Virginia Library has one of the few existing copies of the edition. In the EAF project, therefore, we are using first editions from the University of Virginia Library that meet the following criteria:
\begin{itemize}
\item the author is in BAL, or
\item the edition is listed in Wright;
\item the University of Virginia Library has a first edition of the work.\autocite{noauthor_early_nodate-4}
\end{itemize} 
\end{displayquote}
The policy for inclusion is compounded not just with Wright's original parameters, but those of the BAL, and by Virginia's own collections.\footnote{A bibliography of the titles available in the EAF can be found at \autocite{noauthor_early_nodate-3}. Interestingly, the bibliographical information does include a description of the software and technology used to produce the facsimiles of the editions scanned.} As a result, we see a less replete collection of American fiction than we know to be possible. The editorial policy additionally frames the discussion of its titles as "masterpieces," with references to canonical figures, including Poe, Cooper, Melville, Twain, and Hawthorne. The writers of this blurb include some gestures to William Wells Brown's \textit{Clotelle} (1853) and Jacobs' \textit{Incidents in the Life of Slave Girl}.\footnote{The project includes the first U.S. edition of \textit{Clotelle} published in 1864, rather than its first edition published in London in 1853. The text of the blurb makes no mention of the autobiographical nature of Jacobs' \textit{Incidents}.}

By constraining the parameters of what fiction is to be included in the EAF, Wright's own voice and the stance of the "bibliographical impossibility" of a text not being worth listing, is irrelevant. Instead, Wright's work informs a bibliography that is less concerned with the ideology implicit in his bibliography, but instead concerned with a collection that highlights a single university's collections and their correspondence with two prominent bibliographies. This particular adaptation of Wright is contradictory to Gale Cengage's, in that its listings it has not sought to expand and to consider Wright as an authority to which all other additions to the collection must be informed, but instead Wright's own authority as a bibliography is subsidiary to another, and used as a means to provide credibility to a text, but not as the sole guiding principal for the EAF. 

I do not suggest that Wright or \textit{American Fiction} is somehow wronged in this manner, but that the case of ProQuest's EAF as compared to the case of Gale Cengage's \textit{American Fiction} demonstrates that there are multiple ways in which data can be adapted and used to inform the construction of future datasets. In each case, there are different interpretive judgments being made as to how Wright's work can be used to provide scholars with resources they need to further literary study, as well as how Wright can be perhaps improved, expanded, or otherwise added to in order to help accomplish this goal. The same has been seen in the previous examples of the Research Publications' microfilm and the \textit{Wright American Fiction} project at Indiana University. While all of these cases share a similarity in their seeming appreciation of Wright and acknowledgment of his contribution to bibliography and the study of American fiction, they do not receive Wright's data in the same ways, nor do they attempt to adapt or transmit it without change. Regardless of how objective and straightforward the data of \textit{American Fiction} may seem, it has been shown to be productive of different ways of representing and providing data to readers. Every instance of Wright's work beyond the volumes of \textit{American Fiction}, is instead transformitted as its handlers apply more information, more context, and more ways of reading that data than was initially provided. 

\section{Conclusion}

In some ways Wright's work was inevitable. It was certain that someone would appear who had the skill and resources to create as expansive a list of American fiction as Wright did. As the study of American literature was emerging and formalizing in the early twentieth-century, the demand for such a resource as \textit{American Fiction} was growing. Wright's work was not an inhuman feat, though it was difficult and time-consuming. It is not improbable that, had Wright not "staked his claim," as Vail had told him to, someone else would have endeavored to compile a bibliography of American fiction

But, the fact remains that what we have is Lyle H. Wright's \textit{American Fiction}. The production of Wright's bibliography entailed a unique set of events that made his bibliography, its comprehensive coverage, its omissions, and its errors unique. Another bibliographer would have made different decisions than Wright did, and thus, we would have a different set of titles in the list. Another bibliographer may have decided to include gift books, tracts, and annuals--texts that Wright excludes. They may have not had the friendship of Robert Vail, and so would have no reason to exclude the numerous captivity narratives Wright did. They may have chosen to be less critical about juvenile literature, and included \textit{Little Women} under Alcott. They may even have decided, upon reading \textit{Incidents in the Life of a Slave Girl}, that the text was nonfiction and thus not to be included. The effect of all these hypotheticals, then, would of course mean a difference in how those who adopted \textit{American Fiction} would transmit the bibliography, and the texts they carry forward into the digital age. 

Many of the current electronic resources we have available to us for the study of American literature are a product of Wright's work, and the decisions he made. Scholars are capable of accessing the texts within these resources because of unique events and decisions of the person who compiled the original list. Coming to understand this fact is what it means to research and examine the history and culture of data, to expose the personal, social, and cultural elements that inform the creation of a collection. In this light, a collection of data is little different from that of an individual work of literature. Textual scholars and critics have been aware of how such elements can affect the composition of a narrative, can affect its publication, its circulation. It is no surprise then that same can be said of a work that endeavors to present and organize literature for those same scholars. 

As the digital humanities progress, and computational research on large numbers of digital texts becomes more commonplace, understanding the ways data has been shaped becomes an important part of refining the methods, questions, and results of such research. Knowing what we now do about Wright we can begin to further question how computational research using Wright's collection, or collections created using Wright, can be improved. Is a dataset based on Wright made objectively better by the removal of \textit{Incidents} because it respects the authority of Jacobs? Does the removal of a single text amongst a collection of 10,000 make a statistical difference to results of large-scale computational analysis? Or, would such a decision simply mean fulfilling an ethical responsibility to recognize Jacobs and ensure as much accuracy as possible in the dataset regardless of how insignificant its overall effect? 

In the case of a single work, it may be a purely ethical consideration. But what of the hundreds of titles Wright excluded because they were printed in gift books, annuals, or serial extras? The same could be asked of the many captivity narratives Wright did not include because Vail was intending to place them in his bibliography. These entries would likely have larger effects on the possible results of computational analysis. Understanding the nature of the data one uses would show how ultimately human the process of constructing data is, and thus can always be revisited and revised to further refine the research that is produced using it. Data that is adequate for computational research is constructed with both ethical considerations in mind, but as well as acknowledges how historical instances of exclusion can have effects at scale. That our datasets are derived from historical moments that held certain ideas of what constituted American literature and fiction, more specifically, which are no longer defensible in contemporary scholarship presents to us a need to acknowledge how limited earlier attempts at aggregating data were, and thus, how much they can be improved to respond to current scholarly concerns. In this light, what I suggest here is little different from the work of scholarly editing. To borrow Peter Shillingsburg's definition, scholarly editing refers to "editorial efforts designed to make available for scholarly use works not ordinarily available or available only in corrupt or inadequate forms." \autocite[2]{shillingsburg_scholarly_1996} Shillingsburg's definition denotes that editing is focused on providing a scholastic resource in a form that is adequate. The issue, of course, is that the measure for adequacy requires interpretation, and thus is more personal than Shillingsburg describes here. 

By way of conclusion, I wish to compare Shillingsburg's statement here with one of Johanna Drucker's, discussed in the first chapter. When Drucker mentions that data are already interpreted, and that they do not "pre-exist their parameterization," I believe we may also decide to understand Drucker's point in the context of textual editing.\autocite[129]{drucker_graphesis_2014} The act of parameterizing itself necessitates making adequate for an audience the information one has found. Just as an editor takes up a text and reproduces it in a form to aid in research, education, and entertainment, the same is said of the bibliographer and the digital humanist. Wright took what was available to him in libraries and literary scholarship of his day in order to make it suitable for an audience that needed more resources to develop the field of American literature. In much the same way, computational work needs its data to be made "adequate." That in the case of American literature, Wright forms a connecting line between bibliography, data, literary study, and digital humanities work, is not an accident. Instead, it shows how interconnected the fields are, and how current research and work, while not the bibliography of the 1930s, is part of the same line of transmission.

Scholars will continue to make use of data for research as big data methods and questions become more commonplace. But just as the formalization of literary scholarship necessitated critical editions of works, there is an occasion for inquiry and interest for those who are already familiar with the ways in which data are constructed and presented. Because the case of Wright shows how collections of textual materials have a history that extends back beyond the emergence of digital collections, and touches into areas textual criticism knows well, scholars can take a critical look at the ways in which a collection may be changed, expanded upon, altered, or corrected. These tasks would serve to make the data accessed by other scholars more replete in information while also taking into account the idea of collections as substantive textual objects themselves. At the same time, knowing that the construction of data is itself similar to if not heavily informed by the methods of bibliography, what I have argued is not entirely new but a means of engaging with our intellectual ancestors. Those such as Greg, Pollard, etc. (and I would include Wright himself) whose systematic modes of enumeration and description have impacted the work of big data in the humanities because they formalized and disseminated the standards of bibliographical description and critical editing. Because data in the humanities has been gathered through a process of retrieval from library catalogs, collections, and bibliographies, textual scholars and editors are equipped to deal with the byproducts of data, because they are also likely familiar with the ideas that originally, to use Drucker's wording, "parameterized" the data in the first place.