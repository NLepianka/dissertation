%%%%%%%%%%%%%%%%%%%%%%%%%%%%%%%%%%%%%%%%%%%%%%%%%%%
%
%  New template code for TAMU Theses and Dissertations starting Fall 2016.  
%
%
%  Author: Sean Zachary Roberson
%  Version 3.17.09
%  Last Updated: 9/21/2017
%
%%%%%%%%%%%%%%%%%%%%%%%%%%%%%%%%%%%%%%%%%%%%%%%%%%%
%%%%%%%%%%%%%%%%%%%%%%%%%%%%%%%%%%%%%%%%%%%%%%%%%%%%%%%%%%%%%%%%%%%%%
%%                           ABSTRACT 
%%%%%%%%%%%%%%%%%%%%%%%%%%%%%%%%%%%%%%%%%%%%%%%%%%%%%%%%%%%%%%%%%%%%%

\chapter*{ABSTRACT}
\addcontentsline{toc}{chapter}{ABSTRACT} % Needs to be set to part, so the TOC doesnt add 'CHAPTER ' prefix in the TOC.

\pagestyle{plain} % No headers, just page numbers
\pagenumbering{roman} % Roman numerals
\setcounter{page}{2}

\indent This project investigates the case of a prominent bibliography and dataset of American fiction, the Wright \textit{American Fiction} bibliography, and traces how the discrete items within that set come to compose a part of the whole as a result of human decisions, circumstances, and interpretations. Lyle Wright created a three-volume bibliography of American fiction from 1776 to 1900  in which he described over 10,000 texts. Wright’s work became a guide for libraries and archives, but it has also informed the creation of digital datasets of American literature, including Indiana University’s \textit{Wright American Fiction Project} or Gale Cengage’s \textit{American Fiction 1774-1920} collection, which provide digital facsimiles and plain text versions of the titles listed by Wright for scholars. The bibliography’s corpus has been invaluable for big data scholars desiring access to early American texts, but its use does not come without consequences. Minority authors, particularly Indigenous American authors are excluded. Some works are erroneously included, such as Harriet Jacobs’ autobiographical \textit{Incidents in the Life of a Slave Girl} (1861). Canonical works are sometimes omitted, such as Walt Whitman’s novel \textit{Franklin Evans} (1842) or Louisa May Alcott’s \textit{Little Women} (1868). The projects that use Wright as their basis reproduce these errors and decisions in their digitizing of Wright’s original list, ultimately affecting the datasets used by scholars. This work demonstrates how these idiosyncrasies of the Wright \textit{American Fiction} bibliography come into existence and the effects Wright’s decisions have had on work that relies on his list. As the humanities become increasingly interested in data, and the use of computational methods of analysis become more prominent, research such as mine is positioned to affect the ways in which scholars view the objects from which they derive their arguments. This work demonstrates how a list of American fiction titles is assembled, and reveals the process to be an interpretive and debatable process. 


\pagebreak{}
