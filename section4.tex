%%%%%%%%%%%%%%%%%%%%%%%%%%%%%%%%%%%%%%%%%%%%%%%%%%%
%
%  New template code for TAMU Theses and Dissertations starting Fall 2016.  
%
%
%  Author: Sean Zachary Roberson
%  Version 3.17.09
%  Last Updated: 9/21/2017
%
%%%%%%%%%%%%%%%%%%%%%%%%%%%%%%%%%%%%%%%%%%%%%%%%%%%
%%%%%%%%%%%%%%%%%%%%%%%%%%%%%%%%%%%%%%%%%%%%%%%%%%%%%%%%%%%%%%%%%%%%%%
%%                           SECTION IV
%%%%%%%%%%%%%%%%%%%%%%%%%%%%%%%%%%%%%%%%%%%%%%%%%%%%%%%%%%%%%%%%%%%%%

\chapter{BIBLIOGRAPHICAL INCIDENTS IN THE LIFE OF HARRIET JACOBS}

\section{"No Slave Wrote That One"}
On February 3, 1955, Clarence Brigham, the head librarian of the American Antiquarian Society (AAS) who succeeded Robert G. Vail, sent a letter to Lyle Wright inquiring as to whether or not several titles would be included in Wright's bibliography.\autocite{clarence_s._brigham_letter_1955} As an aside, Brigham requests a clarification of Wright on the subject of a book titled \textit{Autobiography of a Female Slave} (1856), which had been described as authored by Mattie Browne. What Brigham needed clarified was whether or not this Mattie Browne was the "Mrs. Martha (Griffith) Browne" indexed in the Library of Congress Catalog. In this letter, Brigham seems to be asking about the nature of several texts that present themselves as nonfiction in their titles.\footnote{The six titles Brigham requests information on are as follows: C. Nodhoff, \textit{Whaling \& Fishing} (1856); Chas. L. Newhall, \textit{Adventures of Jack} (1859); Frank Munsell, \textit{Chips for the Chimney} (1871); Mary D. Wallis, \textit{Life in Feejee} (1851); Ada Isaacs Menken, \textit{Infelicia} (1868); and William N. Griggs, \textit{The Celebrated "Moon Story"} (1852). \autocite{clarence_s._brigham_letter_1955}} A title such as \textit{Autobiography of a Female Slave} immediately suggests that it is, in fact, the opposite of a fictional tale. However, Wright determines that this is not the case. In his February 5 response to Brigham, his handwritten note responds quite indirectly to Brigham's question: "Mrs. Martha (Griffith) Browne's \underline{Autobiography of a Female Slave} will be listed by me. No slave wrote that one."\autocite{lyle_h._wright_letter_1955}

The wording is particularly odd, as stated before, because Brigham asks for a clarification about the author's name. He gets this answer, yes, but in addition he is told that Wright has deemed the \textit{Autobiography} as fiction because "no slave" wrote it, thereby refuting the autobiographical claim. The identity of Mattie Browne, or sometimes Mattie Griffiths or Martha Griffiths, is not explained further other than this simple denotation that she was "no slave." Wright is not incorrect here; Browne was indeed a white woman who wrote the so-called \textit{Autobiography} with its misleading claim of the text's genre.\footnote{Browne was born in Kentucky to a slave-owning family, and at one point Browne herself owned slaves. She however became an abolitionist, moved north, and eventually published both the \textit{Autobiography} in addition to a few other abolitionist writings. She died in 1906.\autocite{browne_autobiography_1998}} The fact that Wright and Brigham had Griffiths' name on hand as a piece of bibliographical information suggests that the text was already being catalogued and acknowledged as fiction, even though the text was published anonymously. Griffith, in the year following the novel's publication, began to take ownership of it within the abolitionist circle she moved within, a circle that included Elizabeth Peabody and Lydia Marie Child.\footnote{Lydia Marie Child would likewise publish a few fictional antislavery tales in a collection titled \textit{The Liberty Bell} (1842).} \autocite[403-418]{browne_afterword_1998} The text is indeed a fictional work, but it is the odd "no slave" statement that deserves attention. This line can be read in several ways: it could be a poorly worded affirmation that Browne was white and that her race was at odds with the title of her work. In a more rhetorical sense, it acknowledges a set of standards for what an enslaved person could reasonably write, suggesting a reliance on some sort of cultural or educational benchmarks, derived either from historical probability at best and racist stereotypes at worst. Given the emphatic nature of Wright's wording here, rather than the declarative and direct response to Brigham that would have answered the initial question, I believe the latter is a more likely scenario.

What this letter reveals, however, is an acknowledgement of one of the more troubling aspects of the composition of Wright's \textit{American Fiction}, that of the slave narratives, abolitionist writings, and black literary production that appear in Wright's work. Many noteworthy texts appear in the list, including Frederick Douglass' \textit{The Heroic Slave} (1852) and William Wells Browns' \textit{Clotelle} (1853).\footnote{For Douglass, the Wright II no. is 1033; the text is given its own entry, but rather, this entry records the anthology \textit{Autographs for Freedom}, edited by Julia Griffiths. The Wright II nos. for \textit{Clotelle} are 390 and 391, where the London 1853 entry is recorded along with the significantly altered Boston 1867 edition.} Some texts do not appear, such as Martin Delany's \textit{Blake, or the Huts of America} (1859-62), which had been published serially but incompletely in both the \textit{Anglo-African Magazine} and \textit{Anglo-African Weekly}, and remains an incomplete narrative even in modern published editions.\footnote{The absence of \textit{Blake} in this case, may be because of both the lack of materials present in the libraries Wright visited, but also because it did not meet the criteria for inclusion. As the text was incomplete and not separately published, Wright may have been aware of it, but the work did not qualify for inclusion by Wright's standards.} What is more concerning, however, is the misguided inclusion of autobiographical narratives within the Wright \textit{Fiction} list: Solomon Northup's \textit{Twelve Years a Slave} (1853) and Harriet Jacobs' \textit{Incidents in the Life of a Slave Girl} (1861). 

Wright describes \textit{Incidents in the Life of a Slave Girl} and \textit{Twelve Years A Slave} in Wright II, covering years 1851-1875. This volume has both an original edition published in 1957, with a revised edition in 1965. In that time, the only changes that occur in the listings of these two nonfictional works is their numbering and the revision of the author name for \textit{Incidents}, in which [Brent, Linda.] becomes [Jacobs, Mrs. Harriet (Brent).]. This revision shows the increasing knowledge that was being produced about a text such as a \textit{Incidents}, as well as Wright's ability to find and record that knowledge in order to more faithfully describe the books he was listing. However, the fundamental quality of these two works, that is, their autobiographical nature which runs contrary to aims of the bibliography, has never been revised or updated. In this chapter, I wish to explore some questions pertaining to the inclusion of these two autobiographical narratives into the context of the Wright \textit{American Fiction} bibliography: 
\begin{enumerate}
\item How did the narratives of Jacobs and Northup end up being listed by Wright?
\item What are the consequences of these texts' placement within \textit{American Fiction}?
\end{enumerate}

My discussion will primarily privilege Harriet Jacobs and \textit{Incidents in the Life of a Slave Girl}, but Northup and \textit{Twelve Years} should not remain out of sight, as they periodically emerge in context alongside \textit{Incidents} as a narrative whose understanding as a fictional or nonfictional work was volatile and apparently necessary to authenticate. The goal of this chapter is to explore more directly the life of a single text and how the bibliographic record of it reflects on the nature of how this text has been read and positioned within literary history. The status of a text like \textit{Incidents} is not stable, from its early reception to its presence (or absence) in literary scholarship at a time contemporaneous with Wright, and further then to the early recovery and authentication work of literary scholars in the 1970s and 80s, which relied upon the bibliographical data that was available. As a work, \textit{Incidents} straddles multiple genres and discourses, and at the same time, multiple modes of describing and cataloging the work exist or have existed, troubling the way Jacobs' work is enshrined within the context of nineteenth-century American literary scholarship. This is not surprising. Jacobs was a writer and published a work outside of the typical standards that book description and cataloging were designed to handle. The mistakes made in the construction of \textit{Incidents}' data reflect upon how the text was subsequently read by scholars. 

There are two things to be said about Wright's decision to describe \textit{Incidents} in \textit{American Fiction}. First, Wright's inclusion of the text amongst the other works by black writers in pre-1900 America represents a progressive testament to black literature. Part of the survival of Jacobs' text and its accessibility outside of archives is due to Wright's description of \textit{Incidents}. Wright composed the second volume of \textit{American Fiction} in the 1940s, amidst the heights of segregation and Jim Crow laws, and on the cusp of the emergence of the Civil Rights Movement. In this context, \textit{American Fiction} may seem a small, scholarly protest against the larger cultural context of America.\footnote{It is worth reiterating here Wright's claim that there exists no book unworthy of description. This stance is egalitarian in concept, though in the context of this chapter also shows some problems that emerge when assumptions about texts, particularly a nonfictional text by a black woman, can affect the literary life and reception of a narrative despite good intentions.} However, for the second point, the argument made by \textit{American Fiction} contradicts the historical accuracy of \textit{Incidents} and removes some of the power of Jacobs's voice when her story is determined to be inauthentic as judged by a white bibliographer whose moment of encounter with the text is distant from Jacobs's own. This is further complicated by the contemporary methodologies of bibliography which would have approached Jacobs' document with assumptions and practices derived from Early Modern British publishing that would not anticipate the circumstances of a self-publishing black woman in ante-bellum Boston. \textit{Incidents} stands now as a canonical text, warranting well-edited editions and a robust amount of scholarship that has graduated beyond defending and authenticating the text, though those moments were also generative and important. However, at the same time, the errors of the past have not completely eroded away and arguably may be more susceptible to revival as bibliographic data comes into focus within the realms of digital humanities work that relies on library records or datasets derived from bibliographies such as Wright's. 

\section{"Too Orderly": Early Discourse Surrounding \textit{Incidents}}
We should first understand the way \textit{Incidents} has been recovered and the status of it as a text that straddles the boundary between fiction and nonfiction when scholars discuss it. The position of \textit{Incidents} as a text that has had to transition from being considered fiction to nonfiction has produced several arguments that attempt to read its perceived fictional qualities as fundamental to the truth and strategy of the text. However, these discussions also tend to share a quality in their arguments that not only think about the rhetoric and style of the fictional works which \textit{Incidents} were modeled from, but also how the physical form of the text necessarily informs the semantic contents and their reflection of the truth of Jacobs' narrative. The text's announcement of itself as an autobiography and a nonfictional work occurs just beyond the title-page; the first sentence of the preface by the author reads, "Reader, be assured this narrative is no fiction."\autocite[5]{linda_brent_harriet_jacobs_incidents_1861} The editor's introduction which follows the author's similarly uses the term autobiography in the first sentence: "The author of the following autobiography is personally known to me, and her conversation and manners inspire me with confidence."\autocite[7]{linda_brent_harriet_jacobs_incidents_1861} The appendix of the novel features a letter of endorsement from the abolitionist and Quaker Amy Post, with whom Jacobs was a close friend and communicated with while composing \textit{Incidents}; this letter ends with a validation of the work as nonfiction as well: "Her story, as written by herself, cannot fail to interest the reader. It is a sad illustration of the condition of this country, which boasts of its civilization, while it sanctions laws and customs which make the experience of the present more strange than any fictions of the past."\autocite[305]{linda_brent_harriet_jacobs_incidents_1861} Yet these testimonies did not appear adequate to either literary historians and scholars, nor to the bibliographers and catalogers of the libraries in the early twentieth century.  

Historical catalog records of \textit{Incidents} are not unanimous as to the status of the work, even though systems are in place to denote and clarify classificatory issues of genre. Autobiographies and fictional works warrant their own subject headings in the Library of Congress system, making clear a text's place within the system and to afford the locating of such texts. In the Schomburg collection at the New York Public Library, now under the umbrella of the Schomburg Center for Research in Black Culture, the cataloging record from of Griffith's \textit{Autobiography of a Female Slave} is given the subject heading "Slavery - Kentucky - Fiction". The record correctly demarcates the work's genre, with additional remarks as to the content (in this case, it is about slavery, though does not prescribe anything more precise than that, and it is geographically centered on the state of Kentucky).\footnote{It is worth pointing out however, the Schomburg catalog for \textit{Autobiography} does incorrectly note Mattie Griffith as an "American Negro author", however.} Similar designations are found in all the catalog cards for \textit{Uncle Tom's Cabin} in the Schomburg collection, which state "Slavery - U. S. - Fiction." The \textit{Key to Uncle Tom's Cabin}, on the other hand, which is comprised of research and nonfictional evidence that informed \textit{Uncle Tom's Cabin} in addition to Stowe's moral philosophizing, bears only "Slavery - U.S." as a heading. The Schomburg's record for \textit{Incidents} is similar to the \textit{Key} in that it contains the subject heading "Slavery - U.S." as its primary heading, with a secondary heading: "Child, Mrs. Lydia Marie (Francis) 1802-1880, ed." 

Without the "Fiction" marker, it would seem that the text was understood as nonfiction by the New York Public library, except this is complicated by the fact that other texts, such as Douglass' \textit{The Heroic Slave} (1852), his lone fictional work, are also unmarked as belonging to any generic category outside of "Slavery - U.S.", "Slavery - Virginia", and "Slavery - U.S. - Fugitives." Douglass' narratives, however, do get the subject heading that would classify them as nonfiction, with the "Autobiography" heading crediting Douglass himself and denoting its status as a historical document and attestation of Douglass' experiences. This does not necessarily separate Douglass' narratives from literary precedent, but rather asserts one of the most significant claims about Douglass' writing that would keep it from being enumerated in, for example, a list of fiction, or understood as fictional and thus the semantic value of the content of the text radically altered in its reception. That Jacobs is not treated with the same consideration suggests Jacobs' liminal state with less cohesive documents, such as the \textit{Key}, and ultimately plays out in Jacobs being left out of either box, as a true narrative or a work of fiction.\footnote{Modern online catalogs rectify this by assigning \textit{Incidents} the "Biography" tag in their MARC records, which still understates Jacobs as the primary producer of her work and invites the spectre of Lydia Marie Child's editorship into the foreground, but at the least, the text is now categorized as a testament and witness of Jacobs' life. See the American Antiquarian Society's record for \textit{Incidents} or the Library of Congress' record. \url{https://catalog.loc.gov/vwebv/search?searchCode=LCCN&searchArg=79170837&searchType=1&permalink=y} }

Literary scholars, whose readings of the text are more explicit and available, encounter some of the same problems in determining the place of \textit{Incidents} in literary culture and its accuracy as a slave narrative. John Blassingame bears the most singular blame as the metonymic stand-in for scholars who have relegated Jacobs' work to the fiction category. In his book, \textit{The Slave Community} (1972, revised edition 1979), Blassingame states the oft-cited comments that condemns Jacobs: 
\begin{displayquote}
...in spite of Lydia Marie Child's insistence that she had only revised the manuscript of Harriet Jacobs "mainly for the purpose of condensation and orderly arrangement," the work is not credible. In the first place, \textit{Incidents in the Life of a Slave Girl} (1861), is too orderly; too many of the major characters meet providentially after years of separation. Then, too, the story is too melodramatic: miscegenation and cruelty, outraged virtue, unrequited love, and planter licentiousness appear practically on every page. The virtuous Harriet sympathizes with her wretched mistress who has to look on all of the mulattoes fathered by her husband, she refuses to bow to the lascivious demands of her master, bears two children for another white man, and then runs away and hides in a garret in her grandmother's cabin for seven years until she is able to escape to New York. In the meantime, her white lover has acknowledged his paternity of her children, purchased their freedom, and been elected to Congress. In the end, all live happily ever after.\autocite[373]{blassingame_slave_1979}
\end{displayquote}
Blassingame's conviction stems from the novelized form of \textit{Incidents}'s narrative. His complaints of its orderliness, and the overt presence of its themes--the licentiousness of male slave-owners and the struggle to obtain freedom--and the all-too-convenient nature of the ending with a Congressman lover who can provide a sentimental and satisfying ending to the story all bear the guise of fiction for Blassingame. Credibility here belies orderliness, despite the fact that Child indicates her own editing was to arrange and thereby codify a form of order to the text.

The scholarly response to Blassingame's statements looks upon his criticism as reasonable but flawed when one takes into account the nature of why a text such as \textit{Incidents} could read as a novel. In essence, scholars such as William Andrews, Hazel V. Carby, Frances Smith Foster, Jacqueline Goldsby, and Jean Fagan Yellin have argued that the novelistic qualities of \textit{Incidents} are added for the purpose of readability and as a rhetorical move employed in order to persuade the intended audience of the text, that is, white women, who were also largely considered to be the primary audience of the sentimental and seduction novels of the time, and, hence, why such overlap between abolitionist writing and sentimental writing exists.\footnote{Notably, Joanne Braxton in \textit{Black Women Writing Autobiography} does not frame her discussion of Jacobs and \textit{Incidents} with Blassingame's infamous statement, instead taking as a matter of fact that \textit{Incidents} is an autobiography that makes use of the forms of discourse found in popular nineteenth-century novels, without entertaining the discussion of credibility. \autocite{braxton_black_1989}} Scholars who have discussed Jacobs then, while consistently citing Blassingame as the figure against whom they are arguing, use Blassingame as a way to explain why a text such as \textit{Incidents} possesses sentimental and novelistic content while positioning itself as a work of nonfiction. 

William Andrews centers his critique of Blassingame on the way Blassingame validates Northup's \textit{Twelve Years a Slave} in the paragraph after his condemnation of \textit{Incidents}. Andrews states: "...\textit{Twelve Years a Slave} is likely to sound more convincing than \textit{Incidents} because the fictionalizing of the former does not call attention to itself so much, nor does it make appeals to the kind of sentiment that often discomfits and annoys twentieth-century critics." \autocite[270]{andrews_tell_1986} For Andrews, the fictional elements of a text primarily center on the dialogue, an inherently unstable aspect of the text that attempts to recall moments from before the composition of the text and does so imperfectly, a point which he grounds in the theory of M. Bahktin.\footnote{Andrews is reading here Bahktin's \textit{The Dialogic Imagination} as translated by Caryl Emerson in 1981.\autocite[329]{andrews_tell_1986}} More importantly, however, Andrews applies the term "liminal autobiography" as a way to explain what appears in discourse as the strange case of \textit{Incidents} and Jacobs. Andrews' conceptualization of liminality refers to the marginalization of figures, such as Douglass and Jacobs, that develop in slave narratives between 1850 and 1865. What Andrews refers to is a category of autobiography that is written between the stages of social development, i.e. from slavery to freedom, but may also refer within the text to such moments, or "crises", that cause one to undergo a transition between developmental stages. Though additionally, Andrews says the liminal autobiography may also allow the narrator to "pass over various thresholds into a new relationship with his or her reading audience."\autocite[178]{andrews_tell_1986} Such thresholds suggest changes in genre or modes of writing, as \textit{Incidents} does.

The idea of liminality, however, is taken up by others, overtly or covertly, as a way to address not just Andrews' sense of the word, but a broader definition that applies to \textit{Incidents}. Jacqueline Goldsby cites Andrews' concept of liminality but does not rely on Andrews' definition of the term, as she is more concerned with how \textit{Incidents} as a text exists in a liminal space between fact and fiction, which becomes the mechanism for her defense of Jacobs' work as an inherently literary work: 
\begin{displayquote}
\textit{Incidents} proposes that conventional methods of historical investigation are themselves inadequate measures by which to determine what is "authentic" and what is not. Since, according to Jacobs, "truth" can be discovered only if it is left "concealed," rules of documentary evidence may not resolve the dilemma that \textit{Incidents}, as a slave narrative, confronts: how to preserve testimony of an experience that is itself beyond representation.\autocite[12]{jacqueline_goldsby_``i_1996}
\end{displayquote}
Goldsby's view of the text considers the horrors of slavery as beyond representation, and thus, conveyed in ways that do not match standard discourse employed in an autobiographical work that deterministically documents, rather than narrates, a story. 

Hazel Carby adopts a stance that attempts rather to define \textit{Incidents}'s liminality as revolutionary in developing a necessary and unique discourse for black women. Her contention with Blassingame lies in his consideration of a text such as Jacobs' within a patriarchal framework, rather than in one that evaluates the text as a narrative that is outside that framework: 
\begin{displayquote}
"The criteria for judgment that Blassingame advances here leave no room for a consideration of the specificity and uniqueness of the black female experience. An analogy can be made between Blassingame's criticism of \textit{Incidents} as melodrama and the frequency with which issues of miscegenation, unrequited love, outraged virtue, and planter licentiousness are found and foregrounded in diaries by Southern white women, while absent or in the background of the records of their planter husbands. Identifying such a difference should lead us to question and consider the significance of these issues in the lives of women as opposed to men, not to the conclusion that the diaries by women are not credible because they deviate from the conventions of male-authored texts."\autocite[46]{carby_reconstructing_1987}
\end{displayquote}
While invested in the semantics of the text, Carby mentions briefly the question that different forms of the text may exist in her construction of the man's documentary records, with their inherent authority, versus that of the woman's diary, seen as more superfluous, but as equally documentary as the records. Her argument is complemented by Goldsby's. Thinking about \textit{Incidents} as a different mode of narration outside of that of the typical male-dominated modes of writing and representation, to both Carby and Goldsby, opens up readings that allow for the text to be both nonfictional as well as engaged with the sentimental styles of writing the intended audience would expect at the time. What Carby demonstrates is that there is a necessity in the material properties of a text that can inform how the reading manifests and works. While Carby is certainly speaking of the expected contents of the records versus those of diaries it is worth noting that diaries and financial or business records also look different, and structure their contents differently. Her construction of records versus diaries denotes a means of reading that allows for an understanding of how the physical nature of a text asserts or projects an authority that determines its reading. In this instance, as Carby describes, the forms of writing seen as feminine, such as diaries, or even more broadly, any form of writing that deviates from an understandably male standard, thereby lacks a sense of authority because it does not \textit{appear} to be authoritative.

\textit{Incidents} exists not just in a liminal area in terms of its reception and the content of the narrative, which has been the focus of those recovering Jacobs, but the bibliographical history of the text which reinforces much of the same problem. As Jacqueline Goldsby points out, Jacobs's narrative did not follow the traditional route to printing as many other narratives. Many slave narratives had their beginnings in the oral addressing of abolitionist groups before their printing. This is the story told by Douglass that precedes the publication of his first biography, and Douglass' career was built as much on the lecture circuit as it was on the page. Jacobs, however, took her story directly to print, subverting the traditional protocol giving space to slave voices. \textit{Incidents} as a text existed solely as a manuscript, not a speech or address, before its printing. It was, in essence, designed to be printed.\autocite{jacqueline_goldsby_``i_1996} This point resonates with the previous section, in which the question of the text's sentimentality, orderliness, and novel structure convinced some readers of the text's status as a fictional work. Many of these discussions by Foster, Yellin, and others ignore, or only tangentially touch upon, the bibliographical questions that form the other side of \textit{Incidents}'s life, and in fact, as bibliographical work is meant to do, provides the foundation from which readings of the text might occur. The confusion in regards to the factual nature of \textit{Incidents} that surrounds the discrete moments of the text's acknowledgement and reception can find their basis in the confused nature of the way the text was rendered visible in the first place.

The liminality of \textit{Incidents} as a work is not solely the premise of the recovery-driven scholars who were responding to Blassingame's critique of the narrative. Scholars closer in time to Wright had similar thoughts, and mentioned \textit{Incidents}, though only in passing or minimal statements. In 1926, a decade before Wright, John Herbert Nelson, in \textit{The Negro Character in American Fiction}, would comment on the liminal state of \textit{Incidents} and, curiously, \textit{Autobiography of a Female Slave} as similar texts, declaring both of fiction:
\begin{displayquote}
"This last work, edited by the story-writer Mrs. Child and suspiciously like a sentimental novel, suggests that there was, in fact, another group of these books not genuine narratives at all, but wholly fictitious--romances masquerading as authentic autobiographies--a group related, on the one hand, to antislavery verse and on the other to antislavery fiction. In narratives of this class the hero is usually sentimental, super-refined in manner and feeling, more like philosophizing slave of the versifiers than the red-blooded fugitive of real life. On the other hand, this hero has also an affinity with Uncle Tom, in that both are purely fictional creations and both heroes of elaborate stories."\autocite[66]{nelson_negro_1926}
\end{displayquote}
Nelson's insight into Jacobs' writing relies on both his understanding that \textit{Uncle Tom's Cabin}, as the pinnacle of abolitionist writing, "overshadowed" all other works that came out in the same period.\autocite[67]{nelson_negro_1926} His classification of \textit{Incidents} suggests the text is more of a participant in a fad induced by the most popular American novel of the nineteenth-century rather than a true narrative. \textit{Incidents} is a replication rather than an authentic narrative in itself that is trying to perform the same function as \textit{Uncle Tom's Cabin.} For Nelson, the "red-blooded fugitive" seems to be the only possibility for authenticating the narrative, without which, the text is more of a hero's journey whose bias is clearly abolitionist in nature, at odds with lived experience. Or rather, lived experience and the genuine status as a former or fugitive slave is incompatible with being able to create sentimental and philosophical anti-slavery discourse, and is instead only capable of documenting. 

Nelson is not representative of a universal sentiment towards \textit{Incidents}, however. Vernon Loggins, in his 1931 book, \textit{The Negro Author: His Development in America to 1900}, mentions Harriet Jacobs and \textit{Incidents} alongside \textit{Twelve Years A Slave}.\footnote{Curiously, in both the text of discussion of \textit{Twelve Years a Slave} and in the bibliography in the back of the book, Loggins claims the work as anonymous.} His treatment of Jacobs and her work compares to Nelson's in that he too decides that there existed a category of slave narrative that, while presenting as autobiographical, was in fact fictional. Emblematic of this category is Griffiths' \textit{Autobiography of a Female Slave}. The existence of this category of texts creates a problem for readers, as he claims, "It is impossible to draw the line between the true and the fictional in even the most honest of the biographies."\autocite[229]{loggins_negro_1964} However, \textit{Incidents} (and \textit{Twelve Years}) are both outside of this classification, in contradiction to Nelson's hierarchy. Instead, Loggins does believe the texts to at the least be more factual than fictional, calling them biographies, and listing them as such in the bibliography at the end of his book. Loggins does not commit to a long or detailed discussion of \textit{Incidents}, as he quickly moves on, though not before crediting the text's readability to Lydia Marie Child's editing. Loggins' brief mention of Jacobs shows an interpretive approach that differs from the one suggested by Nelson. Even though the two agree on the general frame of reference for how to classify slave narratives, (i.e. that there exist fictional slave narratives) they disagree as to \textit{Incidents}' classification. What both scholars ultimately show, however, is that \textit{Incidents} was, even in its earliest appearances of literary scholarship, outside the sphere of its recovery and authentication, a controversial text in terms of its generic designation. 

Readings such as Nelson's indicate the frame of reference from which Wright and the various catalogers approaching \textit{Incidents} may have informed their own approaches to slave narratives. Literary scholars who believed \textit{Incidents} to be fiction classified it as such not on the basis of evidence or bibliographical information, but instead on the narrative structure and the rhetoric employed. An early reading such as Nelson's mirrors Blassingame's and reveals why it was necessary in the case of recovery scholars to rely on the material conditions of the narrative to set their arguments and comparisons against the claim of fiction. Loggins' counter to Nelson's argument is found in his crediting of Child as an editor. In recognizing Child's role, Loggins demonstrates his similarity with scholars such as Foster, Andrews, and Goldsby, wherein the status of the text as a true narrative is seemingly found in the circumstance that surround it rather than a pure text-only reading of the narrative. If in the case of the literary scholar, the bibliographic details of a text are able to divide between fact and fiction in the reading of \textit{Incidents}, why is it that catalogers and bibliographers were prone to categorizing the text as fictional instead, when their readings of the text were based on the bibliographical data they provided to those scholars? The answer would be in how little attention was paid to the conditions in which the text was published, and instead, on fitting the text of \textit{Incidents} into the dominant schematic by which books were described. A work like \textit{Incidents} was treated as fiction, because it was read as such, and the unique conditions of its publication that were marked by Harriet Jacobs' condition as a fugitive slave woman attempting to self-publish her story. These details were not ignored by scholars working to recover the text. The effect of this erasure of key details to the publishing of \textit{Incidents} and the how it effects the life of this text will be the focus of the next section. 

\section{\textit{Linda} and the Title-page}
One of the most interesting conundrums that is presented by \textit{Incidents} is its title, or rather, the phrase that has become its title. \textit{Incidents in the Life of a Slave Girl} is not necessarily the title of the text published in 1861, and it is not what is always claimed as the title by either contemporaries or early twentieth-century catalogs and lists. If one were to look at the spine of the first edition of the text, the title would read \textit{Linda}, but the text's title-page reads "Incidents in the Life of a Slave Girl." Foster in her article, "Resisting \textit{Incidents}" discusses this peculiar detail of the text, bemoaning the fact that the text is not widely called \textit{Linda}. Because of the text's sentimental qualities, Foster notes, the title \textit{Linda} would make Jacobs' work consistent with other notable sentimental tales: \textit{Pamela}, \textit{Clotel}, \textit{Ramona},  etc. What is even more troubling is how such a change in the naming practice is not concordant with those of other anti-slavery works; as Foster points out, \textit{Uncle Tom's Cabin} is not referred to as \textit{Life Among the Lowly} and yet for some reason the subtitle \textit{Incidents} replaces what would seem to be the proper title.\autocite[67]{frances_smith_foster_resisting_1996} But it is here that Foster gets closer to a bibliographical issue than anyone else, since her discussion, while still rooted in the attempt to understand the semantic content of the novel and its liminal place in literary history, takes notice of the discrepancies between the text's reception and its material properties. Foster's brief musing on the work's title relates to the concept of the data that has formed around \textit{Incidents} and permeated the way in which we are able to talk about it. That is to say, to speak of the title of \textit{Incidents} as data is to speak of the constructed and therefore already-interpreted information that has been assigned to it. It is necessary to mention that other scholars touch upon the data formed from or of the text when they engage in criticism of the work. For example, when scholars sometimes refer to Harriet Jacobs as Linda Brent (the name by which Jacobs herself signs the introduction) they are interpreting information placed in front of them by the text or out-of-date records that still retain Brent as the author. But literary scholars are not as affected by pseudonyms as by changes in titles, and for Foster, the ambiguity of the title represents a moment of resistance, or defiance of Jacobs and her story as it ignores Jacobs' voice in favor of modern critics, bibliographers, and readers. 

If we are to take the spine of the book as a source of information equal in authority to the title-page, the book itself suggests a logical title such as \textit{Linda; or, Incidents in the Life of a Slave Girl}. This would be Foster's way of composing the title as her discussion indicates. However, the fact that this does not happen and that the text has been deemed \textit{Incidents} rather than \textit{Linda} demonstrates an unequal authority between the two parts of the book. This is where bibliography has affected our outlook and codified a text as something that it is possibly not.\footnote{It is here that I would like to acknowledge the fact that I myself have been calling the text \textit{Incidents} here, because even though I will argue about the competing nature of the title \textit{Linda} for this work, I must defer to the dominant discourse of Jacobs scholars for readability.}  The purpose of this section will be to examine the way in which the title \textit{Incidents in the Life of Slavery}, as opposed to \textit{Linda}, emerged and became the dominant way to refer to Jacobs' work. As seen with Foster, the title itself does weigh on readings of the text, but we can ask how a title is assigned and realize that it may be the work of another, or may change over time as a work is cataloged and recorded. This discussion will distance us from Wright somewhat as the discussion of Jacobs' title and the absence of "Linda" in descriptions does not rest solely in his hands, but reflect instead on the wider bibliographic context in which Wright is working. 

\begin{figure}
\includegraphics[scale=.8,keepaspectratio]{linda}
\caption{The spine of the first American edition of \textit{Incident in the Life of a Slave Girl}. See "Harriet Ann Jacobs. Incidents in the life of a slavegirl.," accessed March 2, 2018, http://docsouth.unc.edu/fpn/jacobs.html.}
\end{figure}

\begin{figure}
\includegraphics[scale=.8,keepaspectratio]{incidents-title}
\caption{The title-page of the first American edition of \textit{Incidents in the Life of a Slave Girl}. See "Harriet Ann Jacobs. Incidents in the life of a slavegirl.," accessed March 2, 2018, http://docsouth.unc.edu/fpn/jacobs.html.}
\end{figure}

The title-page of the 1861 \textit{Incidents} is reprinted in many modern editions of \textit{Incidents} including the \textit{Norton Critical} edited by Foster, and the earliest critical editions by Harcourt Brace and Yellin's Harvard edition. This is far from uncommon in critical editions of works, where the original, or earliest available, title-page of the edited work is provided to readers. The practice carries with it an aura of historicity and a suggestion of fidelity to the text. The practice itself can be seen as a relatively simple feat, but also inherently privileges the title-page as a \textit{de facto} source of information when compared to other possible sources such as the spine. The title-page can be easily reprinted in facsimile in a way that the spine cannot. Microfilm copies of a book's title-page are easily accessed and scanned for printing. As well, a title-page is easily re-printed because it is a product of printing; it has already fit and made to be presented on a page, and can easily appear in black and white. A spine, however, would necessitate a photograph, which is expensive to print, especially should it appear in color, and its transition to paper, especially anything other than A1 copy paper, would distort and degrade the image. That the spine is an odd shape for a page layout also works against it being presented. A digital presentation of the text would not necessarily run into this issue, as the affordances of the digital allow for the inclusion of images that are not a page. As the \textit{Documenting the American South} (DAS) project shows on its page for \textit{Incidents} an image of the spine, clearly presenting the lone "Linda" as a title sans author.\autocite{noauthor_harriet_nodate} However, that a digital version represents the spine does not affect how the text's metadata is represented, and the authority of the title-page is maintained. The DAS copy of \textit{Incidents}, as a digital copy, is encoded in XML, which appends additional information and supplies a taxonomy to the text. It is here that the same information seen in a catalog record could be found. Within the XML code, the title is still rendered as \textit{Incidents in the Life of a Slave Girl}, however, ceding to the established standard of the text's name. The presence of the spine image, however, does attest to the peculiarities of the text's physical properties compared to the way it is discussed and read.

It is necessary to understand the significance of the title-page in bibliography, specifically in the early twentieth century and in Wright's own bibliographic work. At this moment, the title-page represents the \textit{prima facie} of bibliographic description. The title-page could be expected to have much necessary information contained in a single space, including the title and author, as expected, but also the imprint, where the place and publisher would be located. Examples such as \textit{Incidents} and its critical editions, show the continued reliance upon the aura of authority that surrounds the title-page. Facsimile reproductions of a text's title-page argue the authenticity and history of a text, but this is not necessarily relevant to what the bibliographer desires when looking at the title-page. 

Wright's reliance on the title-page in the process of creating \textit{American Fiction} is easily verifiable. In the preface of each volume, he explains his standards for his descriptions, including how he decided upon what the title of each entry will be. As Wright states: 
\begin{displayquote}
Some titles have been shortened, and authors' names and quotations appearing on the title-page have been omitted. Omissions are indicated by ellipses. The use of abbreviation "anon." for anonymous, following a title, indicates that the author's name did not appear on the title-page. The abbreviation "pseud.," following a name that is on the title-page as the author, is self-explanatory. These two abbreviations are given only in the first described editions and do not necessarily apply to later editions. Punctuation and capitalization have not followed the original titles."\footnote{See \autocite[ix]{wright_american_1939}. In Wright II, the volume in which Jacobs and \textit{Incidents} are listed, Wright additionally notes the intention behind his regularization of capitalization and punctuation, saying it was to ``avoid the eccentricities of nineteenth-century printing.'' See \autocite[x]{wright_american_1957}}
\end{displayquote}
The information Wright has compiled then, is structured via reference almost universally to the title-page of the text from which that information is derived. All other possibile sources of information are elided in deference to the title-page, including spine, running head, or other pre-existent modes of data, including catalogs (though, admittedly, in this case, most of these would have done little to change how \textit{Incidents} was described). Wright's insistence on the title-page leads us into two different strands of discussion. First, we must consider why the title-page of \textit{Incidents} does not align with the title printed on the spine of the text. The answer to this question necessitates discussion of the history and circumstances of \textit{Incidents}' printing. Secondly, we must ask why Wright's use of the title-page produces this discrepancy between the title-page and spine of \textit{Incidents}. The use of the title-page as the primary source of bibliographical information is standard, and in this way Wright's use of \textit{Incidents} over \textit{Linda} is not singular to him or \textit{American Fiction}. Instead, it represents how Wright's composition of \textit{American Fiction} is informed by the standards of bibliographical description, and how these standards do not always prescribe methods that work cleanly with every text. As a consequence, Wright's description of Jacobs' work demonstrates how bibliographical description can guide the way a text circulates and is read, and how the data is derived from the text described. Even when the purpose of the description is not meant to affect interpretation, the description may still inform a reader when considering how that information may hide key interpretive qualities, such as race and publication details, from the reader. 

While the title-page can be expected to be a general guide to the text it describes, there exist a few issues that affect how a book is then viewed and described. These problems are present when one considers \textit{Incidents}. First comes the tendency of early twentieth-century bibliographers to rely on the primacy of the title-page as the most authoritative source of information. This happens despite the fact that the title-page may not always be either in agreement with the other parts of the book (in this case the spine, in others the running head, etc.). The title-page itself may be the last page produced, and as such is the most susceptible to changes when a book is reprinted. These changes could be due to a new publisher, editor, resetting of the type, or other circumstances that may arise in the midst of publishing. These sorts of circumstances did not occur in the case of \textit{Incidents}, but instead, something even more drastic impacted the printing process: Jacobs' publisher went bankrupt. 

The original publishers of the manuscript of \textit{Incidents}, Thayer and Eldridge, went bankrupt in 1860 before the text could be published.\footnote{Thayer and Eldridge were also the last in a line of publishers Jacobs dealt with in order to secure publication of \textit{Incidents} according to Yellin, and they only agreed to the publication if the text came with a preface by Lydia Marie Child, who agreed to do so.\footnote{Child would also publish an excerpt of \textit{Incidents} in her anthology, \textit{The Freedman's Book} (1865), which would be titled "The Good Grandmother." The text would name Jacobs, and not Linda Brent, as the author, at the end of the excerpt, Child included a note advertising \textit{Incidents}, referring to it as \textit{Linda}.} Another potential publisher, Philips and Samson in Boston wanted a preface by either Harriet Beecher Stowe or her employer, Nathaniel Parker Willis, who was an apologist for slavery. Her multiple attempts to find a publisher in England were also unsuccessful until after the American edition was printed. See \autocite[137-53]{yellin_harriet_2004}} Thayer and Eldridge had proceeded as far as making stereotyped plates of the book, an expensive process that showed their faith in the narrative. With their failing,  Jacobs subsequently purchased the stereotyped plates and sought a way to have them printed. The title \textit{Incidents in the Life of a Slave Girl} was used by Thayer and Eldridge, as shown in a "coming soon" announcement that appeared in the \textit{Anti-Slavery Bugle}, Nov. 3, 1860, two months before the company's closure.\autocite[284]{jacobs_harriet_2008-1} At this point, it was likely that a title had been decided upon and was printed on the running head of the text's body. Thayer and Eldridge's bankruptcy gave Jacobs either control over the title that she did not have with a formal publisher or more time to revise and change the title. Knowing that the process of printing \textit{Incidents} was disrupted by the failing of her intended publisher and left with the plates that held the text, we can know for certain that the first printed copies of \textit{Incidents} were derived from molds and plates that pre-existed the title-page and the spine of the book, with their running head that read "Incidents in the Life of a Slave Girl." on the recto page sides and the chapter titles on the verso sides. The title is further complicated by the heading printed at the beginning of the narrative on the same page that begins the first chapter reads, "Incidents in the Life of a Slave Girl, Seven Years Concealed." Both the running head and heading before the chapter pre-exist the actual printing of the text, and therefore represent a former version of the text. At the time of the creation of the stereotype plates, they may have represented an original title as intended by Jacobs, or perhaps due to the influence of Lydia Marie Child. The title-page, however, is produced after the fact and only represents the title as seen by the running head and not the title that a reader would first encounter on the binding and cover.\footnote{Of additional note, and of evidence of a publisher's power in regards to titles, when the book was finally published in England, the publisher, Frederick W. Chesson, changed the title from both \textit{Linda} and \textit{Incidents} to be instead \textit{The Deeper Wrong}, the title by which it was known to any English reviewer.} 

The last title to be appended to the text, \textit{Linda}, conflicts with the titles that result from the stereotype plates. Given that at the point of binding Jacobs' had more control over her narrative's published form, the difference between the binding title and the title-page printed from plates produced under the direction of her publishers suggests resistance to the original direction the text was headed. The time in between the production of the plates and the eventual printing of the narrative gave Jacobs time to further think about her narrative, though not change much of it. The binding title, therefore, represents an instance of Jacobs as author making a final assertion about how her work will be received, made at the last possible moment before the text would be circulated. The movement of the text \textit{into} the hands of the author after the production of its plates is an atypical condition of many printed works that find their ways into bibliographies. The title on the binding, then, represents what is closest to the bibliographical ideal as proposed by the New Bibliography that emerged in the same period as Wright, the final intentions of the author. The process of relying on the title-page works against the authority of the writer of \textit{Incidents} in this case, and instead places that authority into the hands of the bibliographer, who determines what data describes the text, regardless of the circumstances that surround its publication. Similarly, the standard of the title-page as authority shows the issues that may emerge from a process that is applied consistently to a collection of objects that are not consistent in their production. Instead, the result is an interpretive practice that prescribes information \textit{en masse} and codifies that as truth. In an ironic turn, in the case of \textit{Incidents}, what we can see then is that the process, devised to be accurate, actually perverts the accuracy of the information in the case of a text that does not conform to the expected standards of creation. 

Returning to the wider idea of the title-page as the standard by which bibliographical description is derived, the circumstances that show the understanding of the title-page at the time of Wright will help to contextualize the issues observed in regards to \textit{Incidents}. As early as 1900, John Ferguson in \textit{Some Aspects of Bibliography} explicitly mentions the bibliographical hierarchy of description: "There are now included in book description the title-page, the author's name in full, the place, the printer, the date, the size, the signatures, the number of leaves or pages, the collation, the illustrations--if any--the style of printing, and any peculiarity the book may display."\autocite[10]{ferguson_aspects_1900} These standards resemble modern citations, with only small differences in what information is codified here. Most importantly, however, there is no ambiguity as to how the title of a work is obtained; there is no space given by Ferguson for alternate forms of deriving a title, but instead he notes that title-page as the source solely, as if it was an accurate stand-in for the title.\footnote{Ferguson's work is also notable where he declares that the content of a text is of no interest to the bibliographer. See \autocite[51-2]{ferguson_aspects_1900}} Similarly, Berwick Sayers in 1918 explains his own definition of description, or annotation in his words: "A descriptive extension of the title-page of a book in which the qualifications of the author, and the scope, purpose and place of the book are indicated."\autocite[2]{sayers_first_1918}. The method of describing a book is focused less on supplying the information of a title, rather than on a description of the physical object which may or may not identify that information; that the title of text matches what the title-page describes is, instead, incidental, and liable to be subsumed by the title-page's declaration. Lastly, Arundell Esdaile's \textit{A Student's Manual of Bibliography} (1931), one of the formative guides for bibliography in the early twentieth-century, says the data for description the title is "as found on the title-page, with necessary abbreviations..."\autocite[250]{arundell_esdaile_students_1931}. At this point it is clear how standardized, throughout the early decades of the 1900s, the idea of the title-page as the primary source of information about a text was, and its significance comes, most importantly, at the exclusion of any other possible source. This sort of standard relies on an institutional framework of publishing and production of textual materials that a work such as \textit{Incidents} deviated from.

Early bibliographers were not entirely blind to the problems such a standard could yield, however. In 1920, George Watson Cole discussed what he called "bibliographic ghosts," or editions of texts described by bibliographers that do not actually exist, but are derived from faulty information, errors sometimes passed from record to record. Cole suggests that these "ghosts" are the result of the over-reliance of bibliographers on the title-page for their information without considering the circumstances of their creation.\footnote{In Cole's situation, he is speaking of title-pages that are reappropriated to monographs from original pamphlet copies, and thus were either cropped or cut and lacking an imprint. Subsequent problems with identifying and misrecording the dates of the monograph copies with cut pages then were based on the actions of bibliographers who did not consider the printing process in full and instead relied only on what was singularly in front of them, which sometimes may have been someone else's description rather than the text itself. See \autocite[106-8]{cole_bibliographical_1920}} McKerrow mentions in his \textit{Introduction to Bibliography for Literary Students} (1927) that the use of the title-page for the creation of a bibliographic record requires some discretion, a warning Esdaile and others eschew. Particularly, he warns the difference in a title-page title and the running head of the body of the text may be different because they are created in different instances, and the header, coming first, represents the original author intention versus the input of the publisher. His primary point, however, even amongst others codifying the title-page as truth, is that various circumstances interrupt the process of book creation that may tamper with the set protocol of book description that underwrites what a reader may expect from a bibliographic description: an accurate portrayal not of the title-page but of the book's metadata, its title and author, and publisher, rather than what it presented within the text with its messy circumstances of printing.

The omission of the other facets of the physical text of \textit{Incidents} and its publishing, and the assumption that the work, and mostly the title-page, fit within the standard procedures of publication yet erases the fact that the text was an attempt to recover a work from a failure to engage in that typical process, and instead had a significantly different sort of labor applied to its creation, including the purchasing of its plates to be printed by another, and the decisions of the previous publisher that were in held in place even after their bankruptcy. These details reveal the social layer of the text that bibliographers did not see when looking at the title-page, even when seeing the "published for the author" in place of a traditional publisher in the imprint. The inconsistencies between the title-page and spine or chapter header open a space to where the text can be read materially, and speaks to the unconventional nature of the text's production. In the case of Wright's fiction, not to say anything of the myriad other bibliographers and catalogers that have replicated the same issues, the traditional methodology of description glossed over those gaps in attempt to codify the text in accordance with other, more typically published titles. 

Such a decision conflicts with how Harriet Jacobs chose to represent her text and how she even considered the work. As evidenced by her personal writings, Jacobs did not think of the work as \textit{Incidents}, but instead as either "my Book" and \textit{Linda}. During the process of composition, the narrative was mentioned often without a title, and usually just as "my Book", usually with a capital B.\footnote{\textit{Linda} as a possible title for the text of Jacobs's narrative has been so thoroughly obliterated in modern critical memory that the term does not even warrant an entry in the index of Yellin's collection of Jacobs's family letters, even though the text is referenced many times under that title.} After publication, it seems Jacobs does alter somewhat her manner in discussing it, at least formally, as indicated in a handwritten receipt for Francis Jackson, dated February 1, 1861, who purchased an unknown number of copies of "Linda" for one hundred dollars.\autocite[295]{jacobs_harriet_2008-1} Lydia Marie Child also refers to the text as "Linda," exclusively when not simply referring to it as Jacobs's narrative, helping to keep the author and narrative closely related. Her writings to friends and abolitionist colleagues, including the Quaker poet John Greenleaf Whittier would call the text "Linda."\autocite[335, 341-3]{jacobs_harriet_2008-1} Whittier, along with several other reviewer for abolitionist papers would name it as "Linda" in their writings. William C. Nell, contributing to William Lloyd Garrison's \textit{Liberator} would publicize it as "Linda: \textit{Incidents in the Life of a Slave Girl, seven years concealed in Slavery}" in the January 25, 1861 edition of the paper. Even the British publisher, Chesson, while he changed the title to the \textit{Deeper Wrong} for the British publication, printed on the same plates as the American edition, would refer to the American copies as "Linda" over his title or \textit{Incidents}.\autocite[719]{jacobs_harriet_2008-1}

These firsthand accounts give us not only a record of how early readers of the text, particularly those influential and close to Jacobs, as well as Jacobs herself, conceived of the work now known as \textit{Incidents}. A few early reviews of the American edition did refer to the text as \textit{Incidents in the Life of a Slave Girl}, including the \textit{Anti-Slavery Bugle}, whose contributors were not associated closely with Jacobs or her circle.\footnote{\autocite[327]{jacobs_harriet_2008-1} The \textit{Bugle} had previously published the announcement of the book for Thayer and Eldridge under that title. See note 31.} But others such as the \textit{Weekly Anglo-African} and the \textit{Anti-Slavery Advocate} held to "Linda" as the designated title.\autocite[349,351-2]{jacobs_harriet_2008-1} Confusingly, the \textit{National Anti-Slavery Standard}, with its close ties to Child and Garrison, recognized both titles, "Linda" and \textit{Incidents} in their publicity for the work.\autocite[328-35]{jacobs_harriet_2008-1} Reviewers who preferred \textit{Incidents} over "Linda" may have done what the bibliographers and catalogers in the succeeding century did, in that they deferred to the evidence of the title-page over that other parts of the text. Or, given how freely the \textit{Standard} would alternate the terms, it may also have not been important to note, as the book could be located, and seemed to be understood under both titles regardless, especially considering the name Linda Brent and Lydia Marie Child are so often mentioned so as to direct the work's seekers in the right direction regardless of the title. 

But Jacobs's own chosen title was "Linda," a title that was omitted in favor of the authority established by the title-page and the publishers and stereotyped plates that interpreted that title. The erasure of "Linda" for \textit{Incidents}, a subtitle that has taken the place of the main title, is, as Foster would say, a representation of a resistance to the text, in that the voice of Jacobs has been lost in this case. While her narrative itself remains and follows the title-page, the wrapper of that narrative has shown itself to be a more volatile object that determines the ways readers are able to, firs, locate that text and then read it. What an instance such as Wright's description of Jacobs's work does is further enshrine and proliferate this particular reading of \textit{Incidents}, in such a way as to preserve the dominant voice of the publishers, critics, and bibliographers rather than that of the author. It is no surprise, then, as the recovery of the text was underway, that scholars seeking what evidence they could find of the text would default to the title appended to the text by professionals. Whether approaching Wright, or other bibliographies including Sabin's \textit{Americana} or Blanck's \textit{Bibliography of American Literature}, or any number of catalogs at a library or archive, scholars attempting to reference or pull Jacobs's narrative would find \textit{Incidents} listed as the title, and build their citations as well as their entire readings upon that name while they tried to restore Jacobs's name and voice to the text.  

Bibliography, as a field, has conceived of itself as dedicated to accuracy and precision in the recording of printed information. However, the methods and standards the field developed fall short when a book is produced that does not follow the conventions imagined and set by bibliographers. \textit{Incidents} as a text, has its truth obscured by the practice of bibliography in an ironic attempt at preserving and disseminating awareness of the text's existence. A bibliographer such as Wright, who expresses good intentions in trying to include the diverse amount of work that comprises American fiction, also shows where his blind spots are. These blind spots are a result of the twentieth-century cultural assumptions and ideas that pertained to slave writing and publishing, and systematically deny authors who published under the conditions Jacobs did their ownership and control of the text. This is how the interpretive nature of data can manifest itself, by using the cultural values and judgments of the time as a filter through which the data must pass. The authority given to the bibliographer in these situations has consequences for both the texts that affected by bibliographic error, but as well for the research that looks upon these sources unquestionably. Modern resources that have utilized Wright's work in order to inform their own data adapt and organize Wright's work to suit their own needs, but they do not correct and revise Wright's work, even in such cases as Jacobs'. The chapter will cover this topic more in depth, but as closing note, it should be remembered that the case of Jacobs and \textit{Incidents} is not confined to a single volume, resource, or time period, but has been carried forward into the digital age where the bibliographical work of the early twentieth century has found a new purpose in providing aid to researchers. 